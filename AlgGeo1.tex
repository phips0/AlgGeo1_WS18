\documentclass[12pt,a4paper]{book}
\usepackage[T1]{fontenc}
\usepackage[utf8]{inputenc}
\usepackage{geometry}
\geometry{verbose,tmargin=2cm,bmargin=2cm,lmargin=2cm,rmargin=2cm}
\pagestyle{headings}
\usepackage[ngerman]{babel}
\usepackage{verbatim}
\usepackage{amsmath}
\usepackage{amsthm}
\usepackage{amssymb}
\usepackage{stmaryrd}
\usepackage{makeidx}
\makeindex
\usepackage{setspace}
\usepackage[all]{xy}
\onehalfspacing
\usepackage[bookmarks=true]{hyperref}

%% Theorems (numbered by Part)
\newtheorem{thm}{Theorem}[chapter]
\theoremstyle{definition}
\newtheorem{example}[thm]{Beispiel}
\theoremstyle{definition}
\newtheorem{defn}[thm]{Definition}
\theoremstyle{plain}
\newtheorem{prop}[thm]{Satz}
\theoremstyle{plain}
\newtheorem{cor}[thm]{Korollar}
\theoremstyle{plain}
\newtheorem{lem}[thm]{Lemma}
\theoremstyle{remark}
\newtheorem{rem}[thm]{Bemerkung}
\theoremstyle{plain}

%% Theorems (unnumbered)
\newtheorem*{question*}{Frage}
\theoremstyle{remark}
\newtheorem*{claim*}{Behauptung}
\theoremstyle{definition}
\newtheorem*{notation*}{Notation}
\theoremstyle{definition}
\newtheorem*{example*}{Beispiel}
\theoremstyle{plain}
\newtheorem*{rem*}{Bemerkung}
\theoremstyle{remark}

%% User-specified commands
\DeclareMathOperator{\rad}{rad}
\DeclareMathOperator{\Spec}{Spec}
\DeclareMathOperator{\Quot}{Quot}
\DeclareMathOperator{\im}{\mathrm{im}}
\DeclareMathOperator{\Hom}{\mathrm{Hom}}
\DeclareMathOperator{\Mor}{\mathrm{Mor}}
\DeclareMathOperator{\id}{\mathrm{id}}

%%sets
\DeclareMathOperator{\CC}{\mathbb{C}}
\DeclareMathOperator{\RR}{\mathbb{R}}
\DeclareMathOperator{\QQ}{\mathbb{Q}}
\DeclareMathOperator{\ZZ}{\mathbb{Z}}
\DeclareMathOperator{\NN}{\mathbb{N}}

%% categories
\DeclareMathOperator{\ouv}{\mathcal{O}uv}
\DeclareMathOperator{\set}{\underline{Set}}
\DeclareMathOperator{\ab}{\underline{Ab}}
\DeclareMathOperator{\cattop}{\underline{Top}}
\DeclareMathOperator{\cring}{\underline{CRing}}
\DeclareMathOperator{\psh}{\underline{\mathcal{PS}h}}
\DeclareMathOperator{\sh}{\underline{\mathcal{S}h}}

%% commands
\newcommand{\cat}[1]{\mathcal{#1}}
\newcommand{\sheaf}[1]{\mathcal{#1}}
\renewcommand{\labelenumi}{(\roman{enumi})}
\renewcommand{\labelenumii}{\arabic{enumii}.}

\begin{document}

%% Title page
\title{Algebraische Geometrie I}
\author{Prof. Dr. Venjakob}
\maketitle

\tableofcontents{}
\newpage{}

\section*{Literatur}
\begin{itemize}
\item Görtz, Wedhorn. \emph{Algebraic Geometry I}
\item Hartshorne. \emph{Algebraic Geometry}
\item Shafarevich. \emph{Basic Algebraic Geometry 1 \& 2}
\item Grothendieck. \emph{Eléments de géometrie algébrique, EGA I-IV}
\end{itemize}

\paragraph{Kommutative Algebra}
\begin{itemize}
\item Brüske, Ischebeck, Vogel. \emph{Kommutative Algebra}
\item Kunz. \emph{Einführung in die kommutative Algebra und algebraische Geometrie}
\end{itemize}

\chapter{Prä-Varietäten}
\label{chap:prae-varietaeten}

\include{Chapter1/AlgGeo1-Chapter1-1_Einfuehrung}

\include{Chapter1/AlgGeo1-Chapter1-2_Die-Zariski-Topologie}

\include{Chapter1/AlgGeo1-Chapter1-3_Affine-algebraische-Mengen}

\include{Chapter1/AlgGeo1-Chapter1-4_Der-Hilbertsche-Nullstellensatz}

\include{Chapter1/AlgGeo1-Chapter1-5_Korrespondenz-zwischen-Radikalidealen}

\include{Chapter1/AlgGeo1-Chapter1-6_Irreduzible-topologische-Raeume}

\include{Chapter1/AlgGeo1-Chapter1-7_Irreduzible-algebraische-Mengen}

\include{Chapter1/AlgGeo1-Chapter1-8_Quasikompakte-und-noethersche-topologische-Raeume}

\include{Chapter1/AlgGeo1-Chapter1-9_Morphismen-von-affinen-algebraischen-Mengen}

\include{Chapter1/AlgGeo1-Chapter1-10_Unzulaenglichkeiten}

\include{Chapter1/AlgGeo1-Chapter1-11_Der-affine-Koordinatenring}

\include{Chapter1/AlgGeo1-Chapter1-12_Funktorielle-Eigenschaften-des-Koordinatenrings}

\include{Chapter1/AlgGeo1-Chapter1-13_Raeume-mit-Funktionen}

\include{Chapter1/AlgGeo1-Chapter1-14_Der-Raum-mit-Funktionen-zu-einer-affin-algebraischen-Menge}

\include{Chapter1/AlgGeo1-Chapter1-15_Funktorialitaet-der-Konstruktion}

\include{Chapter1/AlgGeo1-Chapter1-16_Definition-von-Praevarietaeten}

\include{Chapter1/AlgGeo1-Chapter1-17_Vergleich-mit-differenzierbaren-komplexen-Mannigfaltigkeiten}

\include{Chapter1/AlgGeo1-Chapter1-18_Topologische-Eigenschaften-von-Praevarietaeten}

\include{Chapter1/AlgGeo1-Chapter1-19_Offene-Untervarietaeten}

\include{Chapter1/AlgGeo1-Chapter1-20_Funktionenkoerper-einer-Praevarietaet}

\include{Chapter1/AlgGeo1-Chapter1-21_Abgeschlossene-Unterpraevarietaeten}

\include{Chapter1/AlgGeo1-Chapter1-22_Homogene-Polynome}

\include{Chapter1/AlgGeo1-Chapter1-23_Definition-des-projektiven-Raumes}

\include{Chapter1/AlgGeo1-Chapter1-24_Projektive-Varietaeten}

\include{Chapter1/AlgGeo1-Chapter1-25_Koordinatenwechsel-im-projektiven-Raum}

\include{Chapter1/AlgGeo1-Chapter1-26_Lineare-Unterraeume-des-projektiven-Raums}

\include{Chapter1/AlgGeo1-Chapter1-27_Kegel}

\include{Chapter1/AlgGeo1-Chapter1-28_Quadriken}

\chapter{Das Ringspektrum}
\label{chap:das-ringspektrum}

\include{Chapter2/AlgGeo1-Chapter2-1_Definition-von-Spec-A}

\include{Chapter2/AlgGeo1-Chapter2-2_Topologische-Eigenschaften-von-Spec-A}

\include{Chapter2/AlgGeo1-Chapter2-3_Der-Funktor-A-Spec-A}

\include{Chapter2/AlgGeo1-Chapter2-4_Beispiele}

\include{Chapter2/AlgGeo1-Chapter2-5_Garben}

\include{Chapter2/AlgGeo1-Chapter2-6_Halme-von-Garben}

\include{Chapter2/AlgGeo1-Chapter2-7_Die-zu-einer-Praegarbe-assoziierte-Garbe}

%% TODO:
%% - Limiten einfügen in Garben
\section{Direktes und Inverses Bild von Garben}


\section{Lokal geringte Räume}


\section{Die Strukturgrabe auf $\Spec(A)$}


\section{Der Funktor $A \protect\mapsto(\Spec(A),\cat{O}_{\Spec(A)})$}


\section{Beispiele}


\section{Direktes und inverses Bild von Garben}
\label{sec:garben-direktes-inverses-bild}

Sei $f:X\rightarrow Y$ stetige Abbildung topologischer Räume, $\mathcal{F}$
eine Prägarbe auf $X$. Ziel: $f_{\ast}\mathcal{F}$ Prägarbe auf
$Y$, das direkte Bild von $\mathcal{F}$ unter $f$. Definiere $(f_{\ast}\mathcal{F})(V):=\mathcal{F}(f^{-1}(V))$
mit Restriktionsabbildung von $\mathcal{F}$ ($V_{1}\subseteq V_{2}:$
$s\in f_{\ast}\mathcal{F}(V_{2})\rightarrow s|_{V_{1}}=\mathcal{F}res_{f^{-1}(V_{1})}^{f^{-1}(V_{2})}$).
\begin{align*}
  f_{\ast}:PSh(X) & \longrightarrow PSh(Y)\\
  \mathcal{F} & \longmapsto f_{\ast}\mathcal{F}\\
  \mathcal{F}\overset{\varphi}{\rightarrow}\mathcal{G} & \longmapsto f_{\ast}(U):f_{\ast}\mathcal{F}\rightarrow f_{\ast}\mathcal{G}
\end{align*}

ist Funktor via $(f_{\ast}\varphi)_{V}=\varphi_{f^{-1}(V)}$.
\begin{rem}[28]
  \mbox{}
  \begin{enumerate}
  \item $\mathcal{F}$ Garbe auf $X$ $\Longrightarrow f_{\ast}\mathcal{F}$
    Garbe auf $X$, d.h. $f_{\ast}:Sh(X)\rightarrow Sh(Y)$.
  \item Ist $g:Y\rightarrow Z$ eine weitere stetige Abbildung topologischer
    Räume, so existiert ein offensichtlicher Isomorphismus $g_{\ast}\circ(f_{\ast}\mathcal{F})=(g\circ f)_{\ast}\mathcal{F}$,
    funktoriell in $\mathcal{F}$.
  \end{enumerate}
  \medskip{}
\end{rem}

Nun sei $\mathcal{G}$ eine Prägarbe auf $Y$.

\textbf{Ziel:} Definiere $f^{+}\mathcal{G}$ Prägarbe auf $X$. $f^{-1}\mathcal{G}=\widetilde{f^{+}\mathcal{G}}$
Garbe auf $X$, \textbf{Inverses Bild zu $\mathcal{G}$ unter $f$}
via 
\[
(f^{+}\mathcal{G})(U):=\underset{\underset{Y\supseteq V\supseteq f(U)}{\longrightarrow}}{\lim}\mathcal{G}(V)
\]

mit induzierte Restriktionsabbildung.\medskip{}

\textbf{Warnung:} $\mathcal{G}$ Garbe auf $Y$ $\leadsto f^{+}\mathcal{G}$
im Allgemeinen keine Garbe auf $X$. Falls $f:X\hookrightarrow Y$
Inklusion, $\mathcal{G}|_{X}:=f^{-1}\mathcal{G}$. Ist $X\subseteq Y$
offen stimmt $\mathcal{G}|_{X}$ mit der Einschränkung aus Beispiel
19 überein (cofinales Objekt). $\leadsto f^{-1}:PSh(Y)\rightarrow Sh(X)$
Funktor.

$g:Y\xrightarrow{\text{stetig}}Z$, $\mathcal{H}$ Prägarbe auf $Z$,
$U\subseteq X$ offen. 
\[
Z\underset{\text{offen}}{\supseteq}W\supseteq g(f(U))\Longleftrightarrow W\supseteq g(V)
\]

für ein $f(U)\subseteq V\subseteq Y$ offen. 
\begin{align*}
  \underset{\longrightarrow}{\lim}\underset{\longrightarrow}{\lim}=\underset{\longrightarrow}{\lim}\Longrightarrow f^{+}(g^{+}\mathcal{H}) & =(g\circ f)^{+}\mathcal{H}\quad(*)\\
  \Longrightarrow f^{-1}(g^{-1}\mathcal{H}) & =(g\circ f)^{-1}\mathcal{H}
\end{align*}

\begin{example}
  $\imath:\{x\}\rightarrow X$ Inklusion, $\mathcal{F}$ Prägarbe auf
  $X$. $\Longrightarrow\imath^{-1}(\mathcal{F})=\mathcal{F}_{x}$ per
  Definition. $(\ast)\Longrightarrow$
  \[
  \begin{array}{ccc}
    (f^{-1}\mathcal{G})_{x} & = & \mathcal{G}_{f(x)}\\
    \shortparallel &  & \shortparallel\\
    \imath^{-1}\circ(f^{-1}\mathcal{G}) & = & (f\circ\imath)^{-1}\mathcal{G}
  \end{array}
  \]
\end{example}

\begin{prop}[29]
  Für $f:X\rightarrow Y$ stetig sind die Funktionen $f_{\ast}$ und
  $f^{-1}$ zueinander adjungiert, d.h. für $\mathcal{F}$ Garbe auf
  $X$, $\mathcal{G}$ Prägarbe auf $Y$ existiert eine bijektion
  \begin{align*}
    \hom_{Sh(x)}(f^{-1}\mathcal{G},\mathcal{F}) & \longleftrightarrow\hom_{Psh(Y)}(\mathcal{G},f_{\ast}\mathcal{F})\\
    \varphi & \longmapsto\varphi^{\flat}\\
    \psi^{\sharp} & \longmapsfrom\psi
  \end{align*}

  funktoriell in $\mathcal{F}$ und $\mathcal{G}$.
\end{prop}

\begin{proof}
  $\varphi:f^{-1}\mathcal{G}\rightarrow\mathcal{F}$ Morphismus von
  Garben auf $X$. $t\in\mathcal{G}(V)$, $V\subseteq Y$ offen
  \begin{align*}
    \mathcal{G}(V) & \rightarrow f^{+}\mathcal{G}(f^{-1}(V))\xrightarrow{\imath_{f^{+}\mathcal{G}}}f^{-1}\mathcal{G}(f^{-1}(V))\xrightarrow{\varphi_{f^{-1}(V)}}\mathcal{F}(f^{-1}(V))=f_{\ast}\mathcal{F}(V)\\
    & \phantom{\rightarrow\ }\shortparallel\underset{\underset{Y\supseteq W\supseteq ff^{-1}(V)\subseteq V}{\longrightarrow}}{\lim}\\
    t & \mapsto\varphi_{V}^{\flat}(t)
  \end{align*}

  Definition von $\psi^{\#}$. $\mathcal{G}\xrightarrow{\psi}f_{\ast}\mathcal{F}$
  Morphismus von Prägarben auf . Wir definieren $\psi^{\#}:f^{+}\mathcal{G}\rightarrow\mathcal{F}$,
  welches dann $\psi^{\#}:f^{-1}\mathcal{G}\rightarrow\mathcal{F}$
  induziert. $U\subseteq X$ offen, $S\subseteq f^{+}\mathcal{G}(U)$,
  $s=[(V,s_{V})]$, $V\supseteq f(U)$, $s_{V}\in\mathcal{G}(V)$. $\Longrightarrow f^{-1}(V)\supseteq U$.
  \[
  \xymatrix{\psi_{V}(s_{V})\in f_{\ast}\mathcal{F}(V)\ar@{=}[r]\ar@{|->}[rd] & \mathcal{F}(f^{-1}(V))\ar[d]\\
    & \psi_{U}^{\#}(s)\in\mathcal{F}(U)
  }
  \]

  Überprüfe $\varphi^{\flat^{\#}}=\varphi$, $\psi^{\#^{\flat}}=\mathcal{H}$
  und Funktoriell.
\end{proof}
Definition + Proposition 29 verallgemeinern sich zu (Prä)Garben von
Ringen, $R$-Moduln, $R$-Algebren.

\textbf{Beschreibung} von:
\[
\mathcal{G}_{f(x)}=(f^{-1}\mathcal{G})_{x}\overset{\varphi_{x}}{\longmapsto}\mathcal{F}_{x},\ x\in X
\]

\[
\xymatrix{f(x)\in U\underset{\text{offen}}{\subseteq}Y & \mathcal{G}(U)\ar[r]^{\varphi_{U}^{\flat}}\ar@{-->}[d] & \mathcal{F}(f^{-1}(U))\ar[r] & \mathcal{F}_{x}\\
  \underset{\underset{U}{\longrightarrow}}{\lim} & \mathcal{G}_{f(x)}\ar@{-->}[rru]
}
\]


\section{Lokal geringte Räume}
\label{sec:lokal-geringte-raeume}
\begin{defn}
  Ein geringter Raum ist ein Paar $(X,\mathcal{O}_{X})$ bestehend aus
  einem topologischen Raum $X$ und einer Garbe $\mathcal{O}_{X}$ (kommutativer)
  Ringe. Ein Morphismus $(X,\mathcal{O}_{X})\rightarrow(Y,\mathcal{O}_{Y})$
  geringter Räume ist wiederum ein Paar $(f,f^{\flat})$ bestehend aus
  einer stetigen Abbildung $f:X\rightarrow Y$ und einem Homomorphismus
  $f^{\flat}:\mathcal{O}_{Y}\rightarrow f_{\ast}\mathcal{O}_{X}$ von
  Ringgarben auf $Y$. Dieses Datum ist gleichbedeutend (Proposition
  29) mit $(f,f^{\sharp})$, wobei nun $f^{\sharp}:f^{-1}\mathcal{O}_{Y}\rightarrow\mathcal{O}_{X}$
  ein Garbenhomomorphismus auf $X$ ist.

  Bezeichne: $f$ oder $(f,f^{\flat})$ oder $(f,f^{\sharp})$. Damit
  haben wir eine \textbf{Kategorie der geringten Räume}. $\mathcal{O}_{X}$
  heißt Strukturgarbe\index{Strukturgarbe} von $X$, oft schreiben
  wir $X$ für $\mathcal{O}_{X}$.

  Idee: $\mathcal{O}_{X}$ beschreibt die zulässigen Funktionen auf
  $U\subset X$, d.h. etwa stetige, differenzierbare, holomorphe, rigid
  analytische usw. Funktionen. Solche Funktionen auf $V\subset Y$ sollen
  beim ``Zurückziehen'' unter $f$ in dieselbe Klasse überführt werden.
  Dies wird formal durch das Datum $f^{\flat}$ sichergestellt.
\end{defn}

\begin{notation*}
  Wenn $A$ ein lokaler Ring ist, $\mathfrak{m}_{A}$ das maximale Ideal,
  und $\kappa(A)=A/\mathfrak{m}_{A}$ Restklassenkörper. Ein Homomorphismus
  $\varphi:A\rightarrow B$ lokaler Ringe heißt \textbf{lokal}, falls
  $\varphi(\mathfrak{m}_{A})\subset\mathfrak{m}_{B}$. $(f,f^{\flat})=(f,f^{\sharp})=X\rightarrow Y$
  Morphismus geringter Räume induziert:
  \begin{align*}
    & \mathcal{O}_{Y,f(x)}=(f^{-1}\mathcal{O}_{Y})_{x}\xrightarrow{f_{x}^{\sharp}}\mathcal{O}_{X,x}\\
    \text{oder } & \xymatrix{\mathcal{O}_{Y}(U)\ar[r]^{f_{U}^{\flat}}\ar[d] & \mathcal{O}_{X}(f^{-1}(U))\ar[d] & f(x)\in U\subset Y\text{ offen}\\
      \mathcal{O}_{Y,f(x)}=\lim\mathcal{O}_{Y}(U)\ar@{-->}[r] & \mathcal{O}_{X,x}
    }
  \end{align*}
\end{notation*}
\begin{defn}[orig. 31]
  Ein lokal geringter Raum ist ein geringter Raum $(X,\mathcal{O}_{X})$,
  für der $\mathcal{O}_{X,x}$ für alle $x\in X$ ein \emph{lokaler
    Ring} ist. Ein Morphismus $(X,\mathcal{O}_{X})\rightarrow(Y,\mathcal{O}_{Y})$
  lokal geringter Räume ist ein Morphismus geringter Räume $(f,f^{\flat})$,
  so dass die induzierte Abbildung 
  \[
  f_{x}^{\sharp}:\mathcal{O}_{Y,f(x)}\rightarrow\mathcal{O}_{X,x}
  \]

  ein lokaler Ringhomomorphismus ist für alle $x\in X$. Dies führt
  zu einer Unterkategorie der Kategorie geringter Räume, die im Allgemeinen
  \emph{nicht }voll ist, d.h. es gibt Morphismen $f$ geringter Räume
  zwischen lokal geringten Räume, die nicht lokal sind!
\end{defn}

Bezeichne: 
\begin{itemize}
\item $\mathcal{O}_{X,x}$ der ``lokale Ring von $X$ in $x$'';
\item $\mathfrak{m}_{x}$ maximales Ideal;
\item $\kappa(x):=\mathcal{O}_{X,x}/\mathfrak{m}_{x}$ Restklassenkörper
  (bei $x$). \texttt{
  \[
  \xymatrix@R=0pt{\mathcal{O}_{X}(U)\ar[r] & \mathcal{O}_{X,x}\ar[r] & \kappa(x)\\
    f\ar[rr] &  & f(x)
  }
  \]
}
\end{itemize}
Warum \textbf{lokal} geringte Räume? Heuristik:
\begin{align*}
  \mathcal{O}_{X}(U) & \leftrightarrow\text{Funktionen auf }U\\
  \mathcal{O}_{X,x} & \leftrightarrow\text{Funktionen auf Umgebung }U\text{ von }x
\end{align*}

\emph{Wunsch}: $f(x)\neq0\overset{!}{\Rightarrow}f$ ist invertierbar
auf einer kleinen Umgebung $V$ von $x$, d.h. 
\[
\mathcal{O}_{X,x}\backslash\underbrace{\{f\mid f(x)=0\}}_{=\mathfrak{m}_{x}}\subset\mathcal{O}_{X,x}^{\times},
\]

also $\mathcal{O}_{X,x}$ lokal. \emph{Ferner}: $g\mathcal{O}_{Y,f(x)}$
mit $g(f(x))=0$ sollte implizieren: $(g\circ f)(x)=0$. Übersetzt:
\[
f_{x}^{\sharp}(\mathfrak{m}_{f(x)})\subset\mathfrak{m}_{x},\quad f_{x}^{\sharp}(g)="g\circ f"
\]

\begin{example}[orig. 32]
  $\varphi_{X}$ Garbe der $\mathbb{R}$-wertiger stetiger Funktionen
  auf einem topologischen Raum $X$. $\varphi_{X,x}$ Ring der Keime
  $[s]$ stetiger Funktionen in einer Umgebung von $X$:
  \[
  \mathfrak{m}_{x}=\{[s]\in\varphi_{X,x}\mid0=s(x)\}
  \]

  ist einziges maximale Ideal, d.h. $(X,\varphi_{X})$ ist lokal geringter
  Raum. 

  \emph{Denn}: Sei $[s]\in\varphi_{X,x}\backslash\mathfrak{m}_{x}$
  gegeben.

  $\Rightarrow s(x)\neq0$ für alle $s\in[s]$. 

  $\Rightarrow(s$ stetig) $\exists x\in U\subset X$ offen mit $s(u)\neq0$
  für alle $u\in U$.

  $\Rightarrow\frac{1}{s|_{U}}\in\varphi_{X}(U)$ existiert.

  $\Rightarrow\varphi_{X,x}\backslash\mathfrak{m}_{x}=\varphi_{X,x}^{\times}$
  Einheitengruppe. Es ist: 
  \[
  \varphi_{X,x}\rightarrow\mathbb{R},\ [s]\mapsto s(x)
  \]

  ein surjektiver Ringhomomorphismus mit $\ker=\mathfrak{m}_{x}$.

  $\Rightarrow\kappa(x)\cong\mathbb{R}$. Sei $f:X\rightarrow Y$ stetig,
  $V\subset Y$ offen.
  \begin{align*}
    f_{x}^{\flat}:\varphi_{Y}(V) & \longrightarrow\varphi_{X}(f^{-1}(V))=f_{\ast}\varphi_{X}(V)\\
    t & \longmapsto t\circ f
  \end{align*}

  Es folgt:
  \begin{align*}
    \varphi_{Y,f(x)} & \longrightarrow\varphi_{X,x}\\{}
           [t] & \longmapsto[t\circ f]
  \end{align*}

  ist ein Morphismus lokal geringter Räume. Ebenso lassen sich Prävarietäten
  über lokal geringte Räume interpretieren!
\end{example}


\chapter*{Das Ringsprektrum als lokal geringter Raum}
\label{chap:ringspektrum-lokal-geringter-raum}
Ziel: volltreuer Funktor
\begin{align*}
  \text{Ringe} & \longrightarrow\text{Kategorie lokal geringter Räume}\\
  A & \longmapsto(\Spec A,\mathcal{O}_{\Spec A})
\end{align*}

\section{Die Strukturgarbe auf Spec A}
\label{sec:strukturgarbe-auf-spec-A}

Sei $X:=\Spec(A)$, $\mathcal{B}=\{D(f)\mid f\in A\}$ Basis der Topologie.

\textbf{Vorgegeben:} Definiere Prägarbe $\mathcal{O}_{X}$ auf $\mathcal{B}$,
die Garbenaxiome bzgl. $\mathcal{B}$ erfüllt.

\textbf{Wähle:} $\mathcal{O}_{X}(X)=A$ (vgl. Prävarietäten) bzw.
$\mathcal{O}_{X}(D(f))=A_{f}$, da
\begin{align*}
  \imath_{j}:A & \longrightarrow A_{f}\\
  a & \longmapsto\frac{a}{1}
\end{align*}

einen Homöomorphismus $D(f)\xrightarrow{\sim}\Spec(A_{f})$ induziert.
(``Funktionen mit möglichen Polen in $V(f)$).

\subsection{Wohldefiniertheit}
\label{subsec:strukturgarbe-wohldefiniertheit}

$D(f)=D(g)\Rightarrow A_{f}=A_{g}$ kanonisch. Dazu: 
\begin{align*}
  D(f)\subset D(g) & \Leftrightarrow\exists n\geq1\text{ d.d. }f^{n}\in A_{g}
\end{align*}


\subsection{Induzierte Abbildung}
\label{subsec:strukturgarbe-induzierte-abbildung}

\[
\mathcal{O}_{X}(D(g))\rightarrow\mathcal{O}_{X}(D(f)),\ \rho_{f,g}=:\text{res}_{D(f)}^{D(g)}
\]

Dies definiert eine Prägarbe auf $\mathcal{B}$.
\begin{thm}[orig. 33]
  Die Prägarbe $\mathcal{O}_{X}$ ist eine Garbe auf $\mathcal{B}$.
  Die induzierte Garbe auf $X$ (Proposition 20) werde auch mit $\mathcal{O}_{X}$
  bezeichnet. Da
  \[
  \mathcal{O}_{X,x}:=\underset{\underset{D(f)\ni x}{\longrightarrow}}{\lim}\mathcal{O}_{X}(D(f))=\underset{\underset{f\in\mathfrak{p}_{x}}{\longrightarrow}}{\lim}A_{f}=A_{\mathfrak{p}_{x}}
  \]
  mit $(X,\mathcal{O}_{X})=(\Spec A,\mathcal{O}_{\Spec A})$ (kurz $\Spec A$)
  ein lokal geringter Raum.
\end{thm}

\begin{proof}
  Sei $D(f)=\bigcup_{i\in I}D(f_{i})$ Überdeckung in $\mathcal{B}$.
  Zu zeigen:
  \begin{enumerate}
  \item $s\in\mathcal{O}_{X}(D(f))$ mit $s|_{D(f_{i})}=0$, $i\in I$.

    $\overset{!}{\Rightarrow}s=0$.
  \item $s_{i}\in\mathcal{O}_{X}(D(f_{i}))$, $i\in I$, mit $s_{i}|_{D(f_{i})\cap D(f_{j})}=s_{j}|_{D(f_{i})\cap D(f_{j})}$
    $\forall i,j\in I$.

    $\overset{!}{\Rightarrow}\exists s\in\mathcal{O}_{X}(D(f))$ mit $s|_{D(f_{i})}=s_{i}$
    $\forall i\in I$.
  \end{enumerate}
  Ohne Einschränkung:
  \begin{itemize}
  \item $I$ endlich, da $D(f)$ quasi-kompakt.
  \item $f=1$, $D(f)=X$ (mit $(A_{f},\mathcal{O}_{X}|_{D(f)})$ statt $(A,\mathcal{O}_{X})$
    betrachtet) 
    \[
    X=\bigcup_{i\in I}D(f_{i})\Leftrightarrow(f_{i}\mid i\in I)=A
    \]
    Es folgt: $b_{i}=b_{i}(n)\in A$ d.d. $\sum_{i\in I}b_{i}f_{i}^{n}=1$
    \textbf{Zerlegung der 1}. (z)
  \item[Zu 1.] Sei $s=a\in A$ d.d. $0=\frac{a}{1}\in A_{f}$, $\forall i\in I$.
    $I$ endlich, also $\exists n\geq1$ d.d. $f_{i}^{n}a=0$ $\forall i\in I$.
    Mit $(z)$ folgt
    \[
    a=\left(\sum_{i\in I}b_{i}f_{i}^{n}\right)a=0
    \]
  \item[Zu 2.] $s_{i}=\frac{a_{i}}{f_{i}^{n}}$ für $n$ geeignet, unabhängig von
    $i\in I$ (endlich). Nach Voraussetzung:
    \[
    \frac{a_{i}}{f_{i}^{n}}=\frac{a_{j}}{f_{j}^{n}}\in A_{f_{i}f_{j}},\quad D(f_{i})\cap D(f_{j})=D(f_{i}f_{j})
    \]
    Es folgt: $\exists m\geq1$ d.d. $(f_{i}f_{j})^{m}(f_{j}^{n}a_{i}-f_{i}^{n}a_{j})=0$
    $\forall i,j$.
    \begin{align*}
      \frac{a_{i}}{f_{i}^{n}} & =\frac{f_{i}^{m}a_{i}}{f_{i}^{n+m}}=:\frac{a_{i}'}{f_{i}^{n'}},\quad n'=n+m
    \end{align*}
    Ohne Einschränkung: $f_{j}^{n}a_{i}=f_{i}^{n}a_{j}$ $\forall i,j\in I$,
    ({*}) denn:
    \begin{align*}
      f_{j}^{m+n}f_{i}a_{i} & =f_{i}^{m+n}f_{j}^{m}a_{j}\\
      f_{j}^{n'}a_{i}' & =f_{i}^{n'}a_{j}'
    \end{align*}
    Setze $s:=\sum_{j\in I}b_{j}a_{j}\in A$ ($(z)$). Es folgt:
    \[
    f_{i}^{n}s=f_{i}^{n}\sum b_{i}a_{j}=\sum b_{j}(f_{i}^{n}a_{j})\overset{(*)}{=}\left(\sum b_{i}f_{i}^{n}\right)a_{i}\overset{(z)}{=}a_{i}
    \]
    also $\frac{s}{1}=\frac{a_{i}}{f^{n}}=s_{i}$.
  \end{itemize}
\end{proof}

\end{document}


\chapter{Schemata}
\label{chap:schemata}

\section{Schemata}

\begin{defn}
	Ein Schemata ist ein lokal geringter Raum $(X,\cat{O}_X)$, der eine offene Überdeckung $(U_i)_{i\in I}$
	besitz derart alle lokal geringter Räume $(U_i,\cat{O}_{X|U_i})$ affine Schemata sind.\\
	Für ein Schemata $S$ besitz $\textbf{Sch}/_{S}$ ~~ \textbf{Kategorie der Schemata über} $S$ oder $S$-\textbf{Schemata}\\
	\begin{itemize}
		\item Morphismen $X \rightarrow S$ von Schemata
	\end{itemize}
\end{defn}


\section{offene Unterschemata}


\section{Morphismen in affine Schemata}


\section{Morphismen der Form $\Spec(K)\protect\longrightarrow X$}


\section{Verkleben von Schemata und disjunkte Vereinigung}


\section{Der projektive Raum als Schemata}


\include{Chapter3/AlgGeo1-Chapter3-7_nullstellenmenge-in-projektiven-raeme}

\include{Chapter3/AlgGeo1-Chapter3-8_topologische-eigenschaften}

\section{Noethische Schemata}


\section{Generische Punkte}


\section{Reduzierte und ganze Schemata}


\section{Schemata von endlichem Type über $k$}


\section{Prävaritäten als Schemata}


\chapter*{Unterschemata und Immersion (Einbettung)}
\section{offene/abgeschlossen Einbettung}


\section{Reduzierte Unterschemata}



\chapter{Faserprodukte}
\label{chap:faserprod}

\include{Chapter4/AlgGeo1-Chapter4-1-Der-Punkte-Fuktor}

\include{Chapter4/AlGGeo1-Chapter4-2-Yoneda-Lemma}

\section{Faserprodukt in beliebiger Kategorie}

$\cat{C}$ Kategorie, $S \in \text{Oj}(\cat{C})$


\section{Faserprodukte von Schemata}

\underline{Ziel:} $X,Y$ $S$-Schemata


\section{Beispiele}



\section{Basiswechsel}

$\mathcal{C}$ belibige Kategorie


\section{Fasern von Morphismen}


\section{Eigenschaften von Schematamorphismen}


\section{Urbilder und Schema-theoretische-Durchschnitte}


\section{Affine/Projektive Räume über belibige Basen}


\section{Diagonale, Graph, und Kern in belibige Kategorien}


\section{Diagonal für Schemata}



\section{Seperite Morphismen}


\section{Eigentliche Morphismen}



\newpage{}
\printindex{}
\end{document}

\begin{document}

%% Title page
\title{Algebraische Geometrie I}
\author{Prof. Dr. Venjakob}
\maketitle

\tableofcontents{}
\newpage{}

\section*{Literatur}
\begin{itemize}
\item Görtz, Wedhorn. \emph{Algebraic Geometry I}
\item Hartshorne. \emph{Algebraic Geometry}
\item Shafarevich. \emph{Basic Algebraic Geometry 1 \& 2}
\item Grothendieck. \emph{Eléments de géometrie algébrique, EGA I-IV}
\end{itemize}

\paragraph{Kommutative Algebra}
\begin{itemize}
\item Brüske, Ischebeck, Vogel. \emph{Kommutative Algebra}
\item Kunz. \emph{Einführung in die kommutative Algebra und algebraische Geometrie}
\end{itemize}

\chapter{Prä-Varietäten}
\label{chap:prae-varietaeten}

\include{Chapter1/AlgGeo1-Chapter1-1_Einfuehrung}
\include{Chapter1/AlgGeo1-Chapter1-2_Die-Zariski-Topologie}
\include{Chapter1/AlgGeo1-Chapter1-3_Affine-algebraische-Mengen}
\include{Chapter1/AlgGeo1-Chapter1-4_Der-Hilbertsche-Nullstellensatz}
\include{Chapter1/AlgGeo1-Chapter1-5_Korrespondenz-zwischen-Radikalidealen}
\include{Chapter1/AlgGeo1-Chapter1-6_Irreduzible-topologische-Raeume}
\include{Chapter1/AlgGeo1-Chapter1-7_Irreduzible-algebraische-Mengen}
\include{Chapter1/AlgGeo1-Chapter1-8_Quasikompakte-und-noethersche-topologische-Raeume}
\include{Chapter1/AlgGeo1-Chapter1-9_Morphismen-von-affinen-algebraischen-Mengen}
\include{Chapter1/AlgGeo1-Chapter1-10_Unzulaenglichkeiten}
\include{Chapter1/AlgGeo1-Chapter1-11_Der-affine-Koordinatenring}
\include{Chapter1/AlgGeo1-Chapter1-12_Funktorielle-Eigenschaften-des-Koordinatenrings}
\include{Chapter1/AlgGeo1-Chapter1-13_Raeume-mit-Funktionen}
\include{Chapter1/AlgGeo1-Chapter1-14_Der-Raum-mit-Funktionen-zu-einer-affin-algebraischen-Menge}
\include{Chapter1/AlgGeo1-Chapter1-15_Funktorialitaet-der-Konstruktion}
\include{Chapter1/AlgGeo1-Chapter1-16_Definition-von-Praevarietaeten}
\include{Chapter1/AlgGeo1-Chapter1-17_Vergleich-mit-differenzierbaren-komplexen-Mannigfaltigkeiten}
\include{Chapter1/AlgGeo1-Chapter1-18_Topologische-Eigenschaften-von-Praevarietaeten}
\include{Chapter1/AlgGeo1-Chapter1-19_Offene-Untervarietaeten}
\include{Chapter1/AlgGeo1-Chapter1-20_Funktionenkoerper-einer-Praevarietaet}
\include{Chapter1/AlgGeo1-Chapter1-21_Abgeschlossene-Unterpraevarietaeten}
\include{Chapter1/AlgGeo1-Chapter1-22_Homogene-Polynome}
\include{Chapter1/AlgGeo1-Chapter1-23_Definition-des-projektiven-Raumes}
\include{Chapter1/AlgGeo1-Chapter1-24_Projektive-Varietaeten}
\include{Chapter1/AlgGeo1-Chapter1-25_Koordinatenwechsel-im-projektiven-Raum}
\include{Chapter1/AlgGeo1-Chapter1-26_Lineare-Unterraeume-des-projektiven-Raums}
\include{Chapter1/AlgGeo1-Chapter1-27_Kegel}
\include{Chapter1/AlgGeo1-Chapter1-28_Quadriken}

\chapter{Das Ringspektrum}
\label{chap:das-ringspektrum}

\include{Chapter2/AlgGeo1-Chapter2-1_Definition-von-Spec-A}
\include{Chapter2/AlgGeo1-Chapter2-2_Topologische-Eigenschaften-von-Spec-A}
\include{Chapter2/AlgGeo1-Chapter2-3_Der-Funktor-A-Spec-A}
\include{Chapter2/AlgGeo1-Chapter2-4_Beispiele}
\include{Chapter2/AlgGeo1-Chapter2-5_Garben}
\include{Chapter2/AlgGeo1-Chapter2-6_Halme-von-Garben}
\include{Chapter2/AlgGeo1-Chapter2-7_Die-zu-einer-Praegarbe-assoziierte-Garbe}
<<<<<<< HEAD

\section{Direktes und inverses Bild von Garben}
\label{sec:garben-direktes-inverses-bild}

Sei $f:X\rightarrow Y$ stetige Abbildung topologischer Räume, $\mathcal{F}$
eine Prägarbe auf $X$. Ziel: $f_{\ast}\mathcal{F}$ Prägarbe auf
$Y$, das direkte Bild von $\mathcal{F}$ unter $f$. Definiere $(f_{\ast}\mathcal{F})(V):=\mathcal{F}(f^{-1}(V))$
mit Restriktionsabbildung von $\mathcal{F}$ ($V_{1}\subseteq V_{2}:$
$s\in f_{\ast}\mathcal{F}(V_{2})\rightarrow s|_{V_{1}}=\mathcal{F}res_{f^{-1}(V_{1})}^{f^{-1}(V_{2})}$).
\begin{align*}
  f_{\ast}:PSh(X) & \longrightarrow PSh(Y)\\
  \mathcal{F} & \longmapsto f_{\ast}\mathcal{F}\\
  \mathcal{F}\overset{\varphi}{\rightarrow}\mathcal{G} & \longmapsto f_{\ast}(U):f_{\ast}\mathcal{F}\rightarrow f_{\ast}\mathcal{G}
\end{align*}

ist Funktor via $(f_{\ast}\varphi)_{V}=\varphi_{f^{-1}(V)}$.
\begin{rem}[28]
  \mbox{}
  \begin{enumerate}
  \item $\mathcal{F}$ Garbe auf $X$ $\Longrightarrow f_{\ast}\mathcal{F}$
    Garbe auf $X$, d.h. $f_{\ast}:Sh(X)\rightarrow Sh(Y)$.
  \item Ist $g:Y\rightarrow Z$ eine weitere stetige Abbildung topologischer
    Räume, so existiert ein offensichtlicher Isomorphismus $g_{\ast}\circ(f_{\ast}\mathcal{F})=(g\circ f)_{\ast}\mathcal{F}$,
    funktoriell in $\mathcal{F}$.
  \end{enumerate}
  \medskip{}
\end{rem}

Nun sei $\mathcal{G}$ eine Prägarbe auf $Y$.

\textbf{Ziel:} Definiere $f^{+}\mathcal{G}$ Prägarbe auf $X$. $f^{-1}\mathcal{G}=\widetilde{f^{+}\mathcal{G}}$
Garbe auf $X$, \textbf{Inverses Bild zu $\mathcal{G}$ unter $f$}
via 
\[
(f^{+}\mathcal{G})(U):=\underset{\underset{Y\supseteq V\supseteq f(U)}{\longrightarrow}}{\lim}\mathcal{G}(V)
\]

mit induzierte Restriktionsabbildung.\medskip{}

\textbf{Warnung:} $\mathcal{G}$ Garbe auf $Y$ $\leadsto f^{+}\mathcal{G}$
im Allgemeinen keine Garbe auf $X$. Falls $f:X\hookrightarrow Y$
Inklusion, $\mathcal{G}|_{X}:=f^{-1}\mathcal{G}$. Ist $X\subseteq Y$
offen stimmt $\mathcal{G}|_{X}$ mit der Einschränkung aus Beispiel
19 überein (cofinales Objekt). $\leadsto f^{-1}:PSh(Y)\rightarrow Sh(X)$
Funktor.

$g:Y\xrightarrow{\text{stetig}}Z$, $\mathcal{H}$ Prägarbe auf $Z$,
$U\subseteq X$ offen. 
\[
Z\underset{\text{offen}}{\supseteq}W\supseteq g(f(U))\Longleftrightarrow W\supseteq g(V)
\]

für ein $f(U)\subseteq V\subseteq Y$ offen. 
\begin{align*}
  \underset{\longrightarrow}{\lim}\underset{\longrightarrow}{\lim}=\underset{\longrightarrow}{\lim}\Longrightarrow f^{+}(g^{+}\mathcal{H}) & =(g\circ f)^{+}\mathcal{H}\quad(*)\\
  \Longrightarrow f^{-1}(g^{-1}\mathcal{H}) & =(g\circ f)^{-1}\mathcal{H}
\end{align*}

\begin{example}
  $\imath:\{x\}\rightarrow X$ Inklusion, $\mathcal{F}$ Prägarbe auf
  $X$. $\Longrightarrow\imath^{-1}(\mathcal{F})=\mathcal{F}_{x}$ per
  Definition. $(\ast)\Longrightarrow$
  \[
  \begin{array}{ccc}
    (f^{-1}\mathcal{G})_{x} & = & \mathcal{G}_{f(x)}\\
    \shortparallel &  & \shortparallel\\
    \imath^{-1}\circ(f^{-1}\mathcal{G}) & = & (f\circ\imath)^{-1}\mathcal{G}
  \end{array}
  \]
\end{example}

\begin{prop}[29]
  Für $f:X\rightarrow Y$ stetig sind die Funktionen $f_{\ast}$ und
  $f^{-1}$ zueinander adjungiert, d.h. für $\mathcal{F}$ Garbe auf
  $X$, $\mathcal{G}$ Prägarbe auf $Y$ existiert eine bijektion
  \begin{align*}
    \hom_{Sh(x)}(f^{-1}\mathcal{G},\mathcal{F}) & \longleftrightarrow\hom_{Psh(Y)}(\mathcal{G},f_{\ast}\mathcal{F})\\
    \varphi & \longmapsto\varphi^{\flat}\\
    \psi^{\sharp} & \longmapsfrom\psi
  \end{align*}

  funktoriell in $\mathcal{F}$ und $\mathcal{G}$.
\end{prop}

\begin{proof}
  $\varphi:f^{-1}\mathcal{G}\rightarrow\mathcal{F}$ Morphismus von
  Garben auf $X$. $t\in\mathcal{G}(V)$, $V\subseteq Y$ offen
  \begin{align*}
    \mathcal{G}(V) & \rightarrow f^{+}\mathcal{G}(f^{-1}(V))\xrightarrow{\imath_{f^{+}\mathcal{G}}}f^{-1}\mathcal{G}(f^{-1}(V))\xrightarrow{\varphi_{f^{-1}(V)}}\mathcal{F}(f^{-1}(V))=f_{\ast}\mathcal{F}(V)\\
    & \phantom{\rightarrow\ }\shortparallel\underset{\underset{Y\supseteq W\supseteq ff^{-1}(V)\subseteq V}{\longrightarrow}}{\lim}\\
    t & \mapsto\varphi_{V}^{\flat}(t)
  \end{align*}

  Definition von $\psi^{\#}$. $\mathcal{G}\xrightarrow{\psi}f_{\ast}\mathcal{F}$
  Morphismus von Prägarben auf . Wir definieren $\psi^{\#}:f^{+}\mathcal{G}\rightarrow\mathcal{F}$,
  welches dann $\psi^{\#}:f^{-1}\mathcal{G}\rightarrow\mathcal{F}$
  induziert. $U\subseteq X$ offen, $S\subseteq f^{+}\mathcal{G}(U)$,
  $s=[(V,s_{V})]$, $V\supseteq f(U)$, $s_{V}\in\mathcal{G}(V)$. $\Longrightarrow f^{-1}(V)\supseteq U$.
  \[
  \xymatrix{\psi_{V}(s_{V})\in f_{\ast}\mathcal{F}(V)\ar@{=}[r]\ar@{|->}[rd] & \mathcal{F}(f^{-1}(V))\ar[d]\\
    & \psi_{U}^{\#}(s)\in\mathcal{F}(U)
  }
  \]

  Überprüfe $\varphi^{\flat^{\#}}=\varphi$, $\psi^{\#^{\flat}}=\mathcal{H}$
  und Funktoriell.
\end{proof}
Definition + Proposition 29 verallgemeinern sich zu (Prä)Garben von
Ringen, $R$-Moduln, $R$-Algebren.

\textbf{Beschreibung} von:
\[
\mathcal{G}_{f(x)}=(f^{-1}\mathcal{G})_{x}\overset{\varphi_{x}}{\longmapsto}\mathcal{F}_{x},\ x\in X
\]

\[
\xymatrix{f(x)\in U\underset{\text{offen}}{\subseteq}Y & \mathcal{G}(U)\ar[r]^{\varphi_{U}^{\flat}}\ar@{-->}[d] & \mathcal{F}(f^{-1}(U))\ar[r] & \mathcal{F}_{x}\\
  \underset{\underset{U}{\longrightarrow}}{\lim} & \mathcal{G}_{f(x)}\ar@{-->}[rru]
}
\]


\section{Lokal geringte Räume}


\section{Die Strukturgrabe auf $\Spec(A)$}


\section{Der Funktor $A \protect\mapsto(\Spec(A),\cat{O}_{\Spec(A)})$}


\section{Beispiele}


=======
\section{Direktes und inverses Bild von Garben}
\label{sec:garben-direktes-inverses-bild}

Sei $f:X\rightarrow Y$ stetige Abbildung topologischer Räume, $\mathcal{F}$
eine Prägarbe auf $X$. Ziel: $f_{\ast}\mathcal{F}$ Prägarbe auf
$Y$, das direkte Bild von $\mathcal{F}$ unter $f$. Definiere $(f_{\ast}\mathcal{F})(V):=\mathcal{F}(f^{-1}(V))$
mit Restriktionsabbildung von $\mathcal{F}$ ($V_{1}\subseteq V_{2}:$
$s\in f_{\ast}\mathcal{F}(V_{2})\rightarrow s|_{V_{1}}=\mathcal{F}res_{f^{-1}(V_{1})}^{f^{-1}(V_{2})}$).
\begin{align*}
  f_{\ast}:PSh(X) & \longrightarrow PSh(Y)\\
  \mathcal{F} & \longmapsto f_{\ast}\mathcal{F}\\
  \mathcal{F}\overset{\varphi}{\rightarrow}\mathcal{G} & \longmapsto f_{\ast}(U):f_{\ast}\mathcal{F}\rightarrow f_{\ast}\mathcal{G}
\end{align*}

ist Funktor via $(f_{\ast}\varphi)_{V}=\varphi_{f^{-1}(V)}$.
\begin{rem}[28]
  \mbox{}
  \begin{enumerate}
  \item $\mathcal{F}$ Garbe auf $X$ $\Longrightarrow f_{\ast}\mathcal{F}$
    Garbe auf $X$, d.h. $f_{\ast}:Sh(X)\rightarrow Sh(Y)$.
  \item Ist $g:Y\rightarrow Z$ eine weitere stetige Abbildung topologischer
    Räume, so existiert ein offensichtlicher Isomorphismus $g_{\ast}\circ(f_{\ast}\mathcal{F})=(g\circ f)_{\ast}\mathcal{F}$,
    funktoriell in $\mathcal{F}$.
  \end{enumerate}
  \medskip{}
\end{rem}

Nun sei $\mathcal{G}$ eine Prägarbe auf $Y$.

\textbf{Ziel:} Definiere $f^{+}\mathcal{G}$ Prägarbe auf $X$. $f^{-1}\mathcal{G}=\widetilde{f^{+}\mathcal{G}}$
Garbe auf $X$, \textbf{Inverses Bild zu $\mathcal{G}$ unter $f$}
via 
\[
(f^{+}\mathcal{G})(U):=\underset{\underset{Y\supseteq V\supseteq f(U)}{\longrightarrow}}{\lim}\mathcal{G}(V)
\]

mit induzierte Restriktionsabbildung.\medskip{}

\textbf{Warnung:} $\mathcal{G}$ Garbe auf $Y$ $\leadsto f^{+}\mathcal{G}$
im Allgemeinen keine Garbe auf $X$. Falls $f:X\hookrightarrow Y$
Inklusion, $\mathcal{G}|_{X}:=f^{-1}\mathcal{G}$. Ist $X\subseteq Y$
offen stimmt $\mathcal{G}|_{X}$ mit der Einschränkung aus Beispiel
19 überein (cofinales Objekt). $\leadsto f^{-1}:PSh(Y)\rightarrow Sh(X)$
Funktor.

$g:Y\xrightarrow{\text{stetig}}Z$, $\mathcal{H}$ Prägarbe auf $Z$,
$U\subseteq X$ offen. 
\[
Z\underset{\text{offen}}{\supseteq}W\supseteq g(f(U))\Longleftrightarrow W\supseteq g(V)
\]

für ein $f(U)\subseteq V\subseteq Y$ offen. 
\begin{align*}
  \underset{\longrightarrow}{\lim}\underset{\longrightarrow}{\lim}=\underset{\longrightarrow}{\lim}\Longrightarrow f^{+}(g^{+}\mathcal{H}) & =(g\circ f)^{+}\mathcal{H}\quad(*)\\
  \Longrightarrow f^{-1}(g^{-1}\mathcal{H}) & =(g\circ f)^{-1}\mathcal{H}
\end{align*}

\begin{example}
  $\imath:\{x\}\rightarrow X$ Inklusion, $\mathcal{F}$ Prägarbe auf
  $X$. $\Longrightarrow\imath^{-1}(\mathcal{F})=\mathcal{F}_{x}$ per
  Definition. $(\ast)\Longrightarrow$
  \[
  \begin{array}{ccc}
    (f^{-1}\mathcal{G})_{x} & = & \mathcal{G}_{f(x)}\\
    \shortparallel &  & \shortparallel\\
    \imath^{-1}\circ(f^{-1}\mathcal{G}) & = & (f\circ\imath)^{-1}\mathcal{G}
  \end{array}
  \]
\end{example}

\begin{prop}[29]
  Für $f:X\rightarrow Y$ stetig sind die Funktionen $f_{\ast}$ und
  $f^{-1}$ zueinander adjungiert, d.h. für $\mathcal{F}$ Garbe auf
  $X$, $\mathcal{G}$ Prägarbe auf $Y$ existiert eine bijektion
  \begin{align*}
    \hom_{Sh(x)}(f^{-1}\mathcal{G},\mathcal{F}) & \longleftrightarrow\hom_{Psh(Y)}(\mathcal{G},f_{\ast}\mathcal{F})\\
    \varphi & \longmapsto\varphi^{\flat}\\
    \psi^{\sharp} & \longmapsfrom\psi
  \end{align*}

  funktoriell in $\mathcal{F}$ und $\mathcal{G}$.
\end{prop}

\begin{proof}
  $\varphi:f^{-1}\mathcal{G}\rightarrow\mathcal{F}$ Morphismus von
  Garben auf $X$. $t\in\mathcal{G}(V)$, $V\subseteq Y$ offen
  \begin{align*}
    \mathcal{G}(V) & \rightarrow f^{+}\mathcal{G}(f^{-1}(V))\xrightarrow{\imath_{f^{+}\mathcal{G}}}f^{-1}\mathcal{G}(f^{-1}(V))\xrightarrow{\varphi_{f^{-1}(V)}}\mathcal{F}(f^{-1}(V))=f_{\ast}\mathcal{F}(V)\\
    & \phantom{\rightarrow\ }\shortparallel\underset{\underset{Y\supseteq W\supseteq ff^{-1}(V)\subseteq V}{\longrightarrow}}{\lim}\\
    t & \mapsto\varphi_{V}^{\flat}(t)
  \end{align*}

  Definition von $\psi^{\#}$. $\mathcal{G}\xrightarrow{\psi}f_{\ast}\mathcal{F}$
  Morphismus von Prägarben auf . Wir definieren $\psi^{\#}:f^{+}\mathcal{G}\rightarrow\mathcal{F}$,
  welches dann $\psi^{\#}:f^{-1}\mathcal{G}\rightarrow\mathcal{F}$
  induziert. $U\subseteq X$ offen, $S\subseteq f^{+}\mathcal{G}(U)$,
  $s=[(V,s_{V})]$, $V\supseteq f(U)$, $s_{V}\in\mathcal{G}(V)$. $\Longrightarrow f^{-1}(V)\supseteq U$.
  \[
  \xymatrix{\psi_{V}(s_{V})\in f_{\ast}\mathcal{F}(V)\ar@{=}[r]\ar@{|->}[rd] & \mathcal{F}(f^{-1}(V))\ar[d]\\
    & \psi_{U}^{\#}(s)\in\mathcal{F}(U)
  }
  \]

  Überprüfe $\varphi^{\flat^{\#}}=\varphi$, $\psi^{\#^{\flat}}=\mathcal{H}$
  und Funktoriell.
\end{proof}
Definition + Proposition 29 verallgemeinern sich zu (Prä)Garben von
Ringen, $R$-Moduln, $R$-Algebren.

\textbf{Beschreibung} von:
\[
\mathcal{G}_{f(x)}=(f^{-1}\mathcal{G})_{x}\overset{\varphi_{x}}{\longmapsto}\mathcal{F}_{x},\ x\in X
\]

\[
\xymatrix{f(x)\in U\underset{\text{offen}}{\subseteq}Y & \mathcal{G}(U)\ar[r]^{\varphi_{U}^{\flat}}\ar@{-->}[d] & \mathcal{F}(f^{-1}(U))\ar[r] & \mathcal{F}_{x}\\
  \underset{\underset{U}{\longrightarrow}}{\lim} & \mathcal{G}_{f(x)}\ar@{-->}[rru]
}
\]

\section{Lokal geringte Räume}
\label{sec:lokal-geringte-raeume}
\begin{defn}
  Ein geringter Raum ist ein Paar $(X,\mathcal{O}_{X})$ bestehend aus
  einem topologischen Raum $X$ und einer Garbe $\mathcal{O}_{X}$ (kommutativer)
  Ringe. Ein Morphismus $(X,\mathcal{O}_{X})\rightarrow(Y,\mathcal{O}_{Y})$
  geringter Räume ist wiederum ein Paar $(f,f^{\flat})$ bestehend aus
  einer stetigen Abbildung $f:X\rightarrow Y$ und einem Homomorphismus
  $f^{\flat}:\mathcal{O}_{Y}\rightarrow f_{\ast}\mathcal{O}_{X}$ von
  Ringgarben auf $Y$. Dieses Datum ist gleichbedeutend (Proposition
  29) mit $(f,f^{\sharp})$, wobei nun $f^{\sharp}:f^{-1}\mathcal{O}_{Y}\rightarrow\mathcal{O}_{X}$
  ein Garbenhomomorphismus auf $X$ ist.

  Bezeichne: $f$ oder $(f,f^{\flat})$ oder $(f,f^{\sharp})$. Damit
  haben wir eine \textbf{Kategorie der geringten Räume}. $\mathcal{O}_{X}$
  heißt Strukturgarbe\index{Strukturgarbe} von $X$, oft schreiben
  wir $X$ für $\mathcal{O}_{X}$.

  Idee: $\mathcal{O}_{X}$ beschreibt die zulässigen Funktionen auf
  $U\subset X$, d.h. etwa stetige, differenzierbare, holomorphe, rigid
  analytische usw. Funktionen. Solche Funktionen auf $V\subset Y$ sollen
  beim ``Zurückziehen'' unter $f$ in dieselbe Klasse überführt werden.
  Dies wird formal durch das Datum $f^{\flat}$ sichergestellt.
\end{defn}

\begin{notation*}
  Wenn $A$ ein lokaler Ring ist, $\mathfrak{m}_{A}$ das maximale Ideal,
  und $\kappa(A)=A/\mathfrak{m}_{A}$ Restklassenkörper. Ein Homomorphismus
  $\varphi:A\rightarrow B$ lokaler Ringe heißt \textbf{lokal}, falls
  $\varphi(\mathfrak{m}_{A})\subset\mathfrak{m}_{B}$. $(f,f^{\flat})=(f,f^{\sharp})=X\rightarrow Y$
  Morphismus geringter Räume induziert:
  \begin{align*}
    & \mathcal{O}_{Y,f(x)}=(f^{-1}\mathcal{O}_{Y})_{x}\xrightarrow{f_{x}^{\sharp}}\mathcal{O}_{X,x}\\
    \text{oder } & \xymatrix{\mathcal{O}_{Y}(U)\ar[r]^{f_{U}^{\flat}}\ar[d] & \mathcal{O}_{X}(f^{-1}(U))\ar[d] & f(x)\in U\subset Y\text{ offen}\\
      \mathcal{O}_{Y,f(x)}=\lim\mathcal{O}_{Y}(U)\ar@{-->}[r] & \mathcal{O}_{X,x}
    }
  \end{align*}
\end{notation*}
\begin{defn}[orig. 31]
  Ein lokal geringter Raum ist ein geringter Raum $(X,\mathcal{O}_{X})$,
  für der $\mathcal{O}_{X,x}$ für alle $x\in X$ ein \emph{lokaler
    Ring} ist. Ein Morphismus $(X,\mathcal{O}_{X})\rightarrow(Y,\mathcal{O}_{Y})$
  lokal geringter Räume ist ein Morphismus geringter Räume $(f,f^{\flat})$,
  so dass die induzierte Abbildung 
  \[
  f_{x}^{\sharp}:\mathcal{O}_{Y,f(x)}\rightarrow\mathcal{O}_{X,x}
  \]

  ein lokaler Ringhomomorphismus ist für alle $x\in X$. Dies führt
  zu einer Unterkategorie der Kategorie geringter Räume, die im Allgemeinen
  \emph{nicht }voll ist, d.h. es gibt Morphismen $f$ geringter Räume
  zwischen lokal geringten Räume, die nicht lokal sind!
\end{defn}

Bezeichne: 
\begin{itemize}
\item $\mathcal{O}_{X,x}$ der ``lokale Ring von $X$ in $x$'';
\item $\mathfrak{m}_{x}$ maximales Ideal;
\item $\kappa(x):=\mathcal{O}_{X,x}/\mathfrak{m}_{x}$ Restklassenkörper
  (bei $x$). \texttt{
  \[
  \xymatrix@R=0pt{\mathcal{O}_{X}(U)\ar[r] & \mathcal{O}_{X,x}\ar[r] & \kappa(x)\\
    f\ar[rr] &  & f(x)
  }
  \]
}
\end{itemize}
Warum \textbf{lokal} geringte Räume? Heuristik:
\begin{align*}
  \mathcal{O}_{X}(U) & \leftrightarrow\text{Funktionen auf }U\\
  \mathcal{O}_{X,x} & \leftrightarrow\text{Funktionen auf Umgebung }U\text{ von }x
\end{align*}

\emph{Wunsch}: $f(x)\neq0\overset{!}{\Rightarrow}f$ ist invertierbar
auf einer kleinen Umgebung $V$ von $x$, d.h. 
\[
\mathcal{O}_{X,x}\backslash\underbrace{\{f\mid f(x)=0\}}_{=\mathfrak{m}_{x}}\subset\mathcal{O}_{X,x}^{\times},
\]

also $\mathcal{O}_{X,x}$ lokal. \emph{Ferner}: $g\mathcal{O}_{Y,f(x)}$
mit $g(f(x))=0$ sollte implizieren: $(g\circ f)(x)=0$. Übersetzt:
\[
f_{x}^{\sharp}(\mathfrak{m}_{f(x)})\subset\mathfrak{m}_{x},\quad f_{x}^{\sharp}(g)="g\circ f"
\]

\begin{example}[orig. 32]
  $\varphi_{X}$ Garbe der $\mathbb{R}$-wertiger stetiger Funktionen
  auf einem topologischen Raum $X$. $\varphi_{X,x}$ Ring der Keime
  $[s]$ stetiger Funktionen in einer Umgebung von $X$:
  \[
  \mathfrak{m}_{x}=\{[s]\in\varphi_{X,x}\mid0=s(x)\}
  \]

  ist einziges maximale Ideal, d.h. $(X,\varphi_{X})$ ist lokal geringter
  Raum. 

  \emph{Denn}: Sei $[s]\in\varphi_{X,x}\backslash\mathfrak{m}_{x}$
  gegeben.

  $\Rightarrow s(x)\neq0$ für alle $s\in[s]$. 

  $\Rightarrow(s$ stetig) $\exists x\in U\subset X$ offen mit $s(u)\neq0$
  für alle $u\in U$.

  $\Rightarrow\frac{1}{s|_{U}}\in\varphi_{X}(U)$ existiert.

  $\Rightarrow\varphi_{X,x}\backslash\mathfrak{m}_{x}=\varphi_{X,x}^{\times}$
  Einheitengruppe. Es ist: 
  \[
  \varphi_{X,x}\rightarrow\mathbb{R},\ [s]\mapsto s(x)
  \]

  ein surjektiver Ringhomomorphismus mit $\ker=\mathfrak{m}_{x}$.

  $\Rightarrow\kappa(x)\cong\mathbb{R}$. Sei $f:X\rightarrow Y$ stetig,
  $V\subset Y$ offen.
  \begin{align*}
    f_{x}^{\flat}:\varphi_{Y}(V) & \longrightarrow\varphi_{X}(f^{-1}(V))=f_{\ast}\varphi_{X}(V)\\
    t & \longmapsto t\circ f
  \end{align*}

  Es folgt:
  \begin{align*}
    \varphi_{Y,f(x)} & \longrightarrow\varphi_{X,x}\\{}
           [t] & \longmapsto[t\circ f]
  \end{align*}

  ist ein Morphismus lokal geringter Räume. Ebenso lassen sich Prävarietäten
  über lokal geringte Räume interpretieren!
\end{example}

\chapter*{Das Ringsprektrum als lokal geringter Raum}
\label{chap:ringspektrum-lokal-geringter-raum}
Ziel: volltreuer Funktor
\begin{align*}
  \text{Ringe} & \longrightarrow\text{Kategorie lokal geringter Räume}\\
  A & \longmapsto(\Spec A,\mathcal{O}_{\Spec A})
\end{align*}

\section{Die Strukturgarbe auf Spec A}
\label{sec:strukturgarbe-auf-spec-A}

Sei $X:=\Spec(A)$, $\mathcal{B}=\{D(f)\mid f\in A\}$ Basis der Topologie.

\textbf{Vorgegeben:} Definiere Prägarbe $\mathcal{O}_{X}$ auf $\mathcal{B}$,
die Garbenaxiome bzgl. $\mathcal{B}$ erfüllt.

\textbf{Wähle:} $\mathcal{O}_{X}(X)=A$ (vgl. Prävarietäten) bzw.
$\mathcal{O}_{X}(D(f))=A_{f}$, da
\begin{align*}
  \imath_{j}:A & \longrightarrow A_{f}\\
  a & \longmapsto\frac{a}{1}
\end{align*}

einen Homöomorphismus $D(f)\xrightarrow{\sim}\Spec(A_{f})$ induziert.
(``Funktionen mit möglichen Polen in $V(f)$).

\subsection{Wohldefiniertheit}
\label{subsec:strukturgarbe-wohldefiniertheit}

$D(f)=D(g)\Rightarrow A_{f}=A_{g}$ kanonisch. Dazu: 
\begin{align*}
  D(f)\subset D(g) & \Leftrightarrow\exists n\geq1\text{ d.d. }f^{n}\in A_{g}
\end{align*}


\subsection{Induzierte Abbildung}
\label{subsec:strukturgarbe-induzierte-abbildung}

\[
\mathcal{O}_{X}(D(g))\rightarrow\mathcal{O}_{X}(D(f)),\ \rho_{f,g}=:\text{res}_{D(f)}^{D(g)}
\]

Dies definiert eine Prägarbe auf $\mathcal{B}$.
\begin{thm}[orig. 33]
  Die Prägarbe $\mathcal{O}_{X}$ ist eine Garbe auf $\mathcal{B}$.
  Die induzierte Garbe auf $X$ (Proposition 20) werde auch mit $\mathcal{O}_{X}$
  bezeichnet. Da
  \[
  \mathcal{O}_{X,x}:=\underset{\underset{D(f)\ni x}{\longrightarrow}}{\lim}\mathcal{O}_{X}(D(f))=\underset{\underset{f\in\mathfrak{p}_{x}}{\longrightarrow}}{\lim}A_{f}=A_{\mathfrak{p}_{x}}
  \]
  mit $(X,\mathcal{O}_{X})=(\Spec A,\mathcal{O}_{\Spec A})$ (kurz $\Spec A$)
  ein lokal geringter Raum.
\end{thm}

\begin{proof}
  Sei $D(f)=\bigcup_{i\in I}D(f_{i})$ Überdeckung in $\mathcal{B}$.
  Zu zeigen:
  \begin{enumerate}
  \item $s\in\mathcal{O}_{X}(D(f))$ mit $s|_{D(f_{i})}=0$, $i\in I$.

    $\overset{!}{\Rightarrow}s=0$.
  \item $s_{i}\in\mathcal{O}_{X}(D(f_{i}))$, $i\in I$, mit $s_{i}|_{D(f_{i})\cap D(f_{j})}=s_{j}|_{D(f_{i})\cap D(f_{j})}$
    $\forall i,j\in I$.

    $\overset{!}{\Rightarrow}\exists s\in\mathcal{O}_{X}(D(f))$ mit $s|_{D(f_{i})}=s_{i}$
    $\forall i\in I$.
  \end{enumerate}
  Ohne Einschränkung:
  \begin{itemize}
  \item $I$ endlich, da $D(f)$ quasi-kompakt.
  \item $f=1$, $D(f)=X$ (mit $(A_{f},\mathcal{O}_{X}|_{D(f)})$ statt $(A,\mathcal{O}_{X})$
    betrachtet) 
    \[
    X=\bigcup_{i\in I}D(f_{i})\Leftrightarrow(f_{i}\mid i\in I)=A
    \]
    Es folgt: $b_{i}=b_{i}(n)\in A$ d.d. $\sum_{i\in I}b_{i}f_{i}^{n}=1$
    \textbf{Zerlegung der 1}. (z)
  \item[Zu 1.] Sei $s=a\in A$ d.d. $0=\frac{a}{1}\in A_{f}$, $\forall i\in I$.
    $I$ endlich, also $\exists n\geq1$ d.d. $f_{i}^{n}a=0$ $\forall i\in I$.
    Mit $(z)$ folgt
    \[
    a=\left(\sum_{i\in I}b_{i}f_{i}^{n}\right)a=0
    \]
  \item[Zu 2.] $s_{i}=\frac{a_{i}}{f_{i}^{n}}$ für $n$ geeignet, unabhängig von
    $i\in I$ (endlich). Nach Voraussetzung:
    \[
    \frac{a_{i}}{f_{i}^{n}}=\frac{a_{j}}{f_{j}^{n}}\in A_{f_{i}f_{j}},\quad D(f_{i})\cap D(f_{j})=D(f_{i}f_{j})
    \]
    Es folgt: $\exists m\geq1$ d.d. $(f_{i}f_{j})^{m}(f_{j}^{n}a_{i}-f_{i}^{n}a_{j})=0$
    $\forall i,j$.
    \begin{align*}
      \frac{a_{i}}{f_{i}^{n}} & =\frac{f_{i}^{m}a_{i}}{f_{i}^{n+m}}=:\frac{a_{i}'}{f_{i}^{n'}},\quad n'=n+m
    \end{align*}
    Ohne Einschränkung: $f_{j}^{n}a_{i}=f_{i}^{n}a_{j}$ $\forall i,j\in I$,
    ({*}) denn:
    \begin{align*}
      f_{j}^{m+n}f_{i}a_{i} & =f_{i}^{m+n}f_{j}^{m}a_{j}\\
      f_{j}^{n'}a_{i}' & =f_{i}^{n'}a_{j}'
    \end{align*}
    Setze $s:=\sum_{j\in I}b_{j}a_{j}\in A$ ($(z)$). Es folgt:
    \[
    f_{i}^{n}s=f_{i}^{n}\sum b_{i}a_{j}=\sum b_{j}(f_{i}^{n}a_{j})\overset{(*)}{=}\left(\sum b_{i}f_{i}^{n}\right)a_{i}\overset{(z)}{=}a_{i}
    \]
    also $\frac{s}{1}=\frac{a_{i}}{f^{n}}=s_{i}$.
  \end{itemize}
\end{proof}

\end{document}

\section{Der Funktor $A\protect\mapsto(\Spec A,\mathcal{O}_{\Spec A})$}
\begin{defn}[34]
  Ein lokal geringter Raum $(X,\mathcal{O}_{X})$ heißt \textbf{affines
    Schema}, falls ein Ring $A$ existiert d.d
  \[
    (X,\mathcal{O}_{X})\cong(\Spec A,\mathcal{O}_{\Spec A})
  \]

  Ein \textbf{Morphismus affiner Schemata} ist ein Morphismus lokal
  geringter Räume. Bezeichne $\aff$ die Kategorie der affinen Schemata.
  \begin{align*}
    \varphi:A & \longrightarrow B & \text{Ringhom.}\\
    f:X:=\Spec A & \longrightarrow Y:=\Spec A & \text{stetige Abb.}
  \end{align*}
\end{defn}

\textbf{Ziel: }Definiere $(f,f^{\flat}):X\rightarrow Y$ mit $f:=^{a}\varphi$
Morphismus von lokal geringter Räume und
\[
  f_{\Spec A}^{\flat}=\varphi:A=\mathcal{O}_{\Spec A}(\Spec A)\rightarrow f_{\ast}\mathcal{O}_{\Spec B}(\Spec B)=B
\]

Dazu: Für $s\in A$ gilt $f^{-1}(D(s))=D(\varphi(s))$ nach Proposition
2.10. Definiere 
\[
  f_{D(s)}^{\flat}:\mathcal{O}_{Y}(D(s))=A_{s}\rightarrow B_{\varphi(s)}=f_{\ast}\mathcal{O}_{X}(D(s))
\]

als die von $\varphi$ induzierte Abbildung. $f^{\flat}$ ist kompatibel
mit $\res_{D(t)}^{D(s)}$ für prinzipal offene Mengen $D(t)\subseteq D(s)$.
$B$ Basis $\Longrightarrow f^{\flat}:\mathcal{O}_{Y}\rightarrow f_{\ast}\mathcal{O}_{X}$
Homomorphismus von Ringgarben. Für $s=1$ erhalten wir $f_{\Spec A}^{\flat}=\varphi$!

Für $x\in X$ gilt:
\[
  \xymatrix{f^{\sharp}:\mathcal{O}_{Y,f(x)}=A_{\varphi^{-1}(\mathfrak{p}_{x})=\mathfrak{p}_{f(x)}}\ar[r] & B_{\mathfrak{p}_{x}}=\mathcal{O}_{X,x}\\
    A\ar[u]\ar[r]^{\varphi} & B\ar[u]
  }
  \qquad(*)
\]

ist der von $\varphi$ induzierte Homomophismus. $f_{x}^{\sharp}$
is lokal:
\[
  \varphi(\varphi^{-1}(\mathfrak{p}_{x}))\subseteq\mathfrak{p}_{x}
\]

\textbf{Bezeichne}: $^{a}\varphi$ für $\Spec(\varphi)=(f,f^{\flat})$,
$^{a}(\psi\circ\varphi)=^{a}\varphi\circ^{a}\psi$. Wir erhalten einen
kontravarianten Funktor
\[
  \Spec:\underline{Ring}\longrightarrow\aff.
\]

Für $f:(X,\mathcal{O}_{X})\rightarrow(Y,\mathcal{O}_{Y})$ Morphismus
von geringten Räumen erhalten wir einen Ringhomomorphismus
\[
  \Gamma(f):=f_{Y}^{\flat}:\Gamma(Y,\mathcal{O}_{Y})=\mathcal{O}_{Y}(Y)\rightarrow\Gamma(X,\mathcal{O}_{X})=(f_{\ast}\mathcal{O}_{X})(Y)=\mathcal{O}_{X}(X).
\]

So erhalten wir einen kontravarianten Funktor
\[
  \Gamma:\aff\longrightarrow\underline{Ring}.
\]

\begin{thm}[35]
  Die Funktoren $\Spec$ und $\Gamma$ definieren eine Anti-Äquivalenz
  zwischen der Kategorie der Ringe und der Kategorie der affinen Schemata.
\end{thm}

\begin{proof}
  $\Spec$ ist essentiell surjektiv per Definition. $\Gamma\circ\Spec$
  ist isomorph zu $\id_{\underline{Ring}}$ nach Konstruktion.Zu zeigen:
  \[
    \Hom_{\ring}(A,B)\stackrel[\Gamma]{\Spec}{\rightleftharpoons}\Hom_{\aff}(\Spec B,\Spec A)
  \]

  sind zueinander invers. Es fehlt die Verkettung $\Spec\circ\Gamma=\id_{\aff}$.
  Sei $f\in\Hom_{\aff}(\Spec B,\Spec A)$, $\varphi:=\Gamma(f)$, $^{a}\varphi=f$.
  Für $\mathfrak{p}_{x}\in\Spec B=X$ ist $f_{x}^{\sharp}$ der eindeutig
  bestimmte Ringhomomorphismus, welcher das Diagramm
  \[
    \xymatrix{A\ar[r]^{f_{\Spec A}^{\flat}=\Gamma(f)=\varphi}\ar[d]_{\imath_{A}} & B\ar[d]^{\imath_{B}}\supset\imath_{B}^{-1}(\mathfrak{p}_{x}B_{\mathfrak{p}_{x}})=\mathfrak{p}_{x}\\
      A_{\mathfrak{p}_{f(x)}}\ar[r]_{f_{x}^{\#}\text{ lokal}} & B_{\mathfrak{p}_{x}}\underset{\max}{\supset}\mathfrak{p}_{x}B_{\mathfrak{p}_{x}}
    }
    \qquad(**)
  \]

  kommutieren lässt. Es gilt:
  \begin{align*}
    \mathfrak{p}_{f(x)}A_{\mathfrak{p}_{f(x)}} & \supseteq(f_{x}^{\sharp})^{-1}(\mathfrak{p}_{x}B\mathfrak{p}_{x})=\mathfrak{p}_{f(x)}A_{\mathfrak{p}_{f(x)}}\\
    \mathfrak{p}_{f(x)} & =\imath_{A}^{-1}(\mathfrak{p}_{f(x)}A\mathfrak{p}_{f(x)})\subset A
  \end{align*}

  $f_{x}^{\#}$ lokal $\Longrightarrow f_{x}^{\#}(\mathfrak{p}_{f(x)}A_{\mathfrak{p}_{f(x)}})\subset\mathfrak{p}_{x}B_{\mathfrak{p}_{x}}$
  $\Longrightarrow^{a}\varphi=f$ als stetige Abbildung. Wegen $(*)$
  lässt auch $(^{a}\varphi)_{x}^{\#}$ das Diagramm $(**)$ kommutieren.
  Proposition $\Longrightarrow(^{a}\varphi)^{\#}=f^{\#}$.
\end{proof}

\section{Beispiele}
\begin{example}[36, Integritätsbereiche]
  Sei $A$ integer, $K=\Quot(A)$. Sei $X=\Spec A$, $\eta=(0)$. Dann
  ist$\overline{\{\eta\}}=\Spec X$, d.h. jede nicht-leere offene Menge
  $U\subset X$ enthält $\eta$. Es folgt: $\mathcal{O}_{X,y}=A_{(0)}=K$.
  Für alle $f\in A$ gilt nach Definition 
  \[
    \mathcal{O}_{X}(D(f))=A_{f}\subset U.
  \]

  Sei $U\subset X$ beliebig offen. Es folgt:
  \[
    \mathcal{O}_{X}(U)=\underset{\underset{D(f)\subset U}{\longleftarrow}}{\lim}\mathcal{O}_{X}(D(f)=\bigcap_{\underset{D(f)\subset U}{f\in A}}A_{f}\subseteq K.
  \]

  Wie im Beweis von Satz 1.37 ist $a_{F}=\bigcap_{\mathfrak{p}\in D(f)}A_{\mathfrak{p}},$
  also $\mathcal{O}_{X}(U)=\bigcap_{x\in U}\mathcal{O}_{X,x}$.
\end{example}

\begin{example}[37, Prinzipal offene Unterschemata affiner Schemata]
  Sei $X=\Spec A$, $f\in A$. Sei $j:\Spec A_{f}\rightarrow\Spec A$
  induziert von $A\rightarrow A_{f}$. $\Longrightarrow j:\Spec A_{j}\rightarrow D(f)$
  ist Homoömorphismus (Proposition 2.12). Für alle $x\in D(f)$ ist
  $j_{x}^{\#}$ der kanonische Isomorphismus $A_{\mathfrak{p}_{x}}\overset{\cong}{\rightarrow}(A_{f})_{\mathfrak{p}_{x}}$.
  $\Longrightarrow(j,j^{\#})$ induziert einen Isomorphismus $\Spec A_{f}\cong(D(f),\mathcal{O}_{X|D(f)})$.
\end{example}

\begin{example}[38, Abgeschlossene Unterschemata affiner Schemata]
  Sei $X=\Spec A$ und $\mathfrak{a}$ ein Ideal von $A$. Sei $\imath:\Spec A/\mathfrak{a}\rightarrow\Spec A$
  der von $A\rightarrow A/\mathfrak{a}$ induzierte Morphismus affiner
  Schemata. Nach Proposition 2.12 induziert $\imath$ einen Homöomorphismus
  $\Spec A/\mathfrak{a}\overset{\cong}{\rightarrow}V(\mathfrak{a})\subseteq\Spec A$.
  Sei $\overline{\mathfrak{p}_{x}}$ das Bild von $\mathfrak{p}_{x}$
  in $A/\mathfrak{a}$. Für alle $x\in V(\mathfrak{a})$ ist der Morphismus
  $i_{x}^{\flat}$ der kanonische Homomorphismus $A_{\mathfrak{p}_{x}}\rightarrow(A/\mathfrak{a})_{\overline{\mathfrak{p_{x}}}}$.
  ($=0$, falls $x\in V(\mathfrak{a})$, also $\mathfrak{a}\notin f_{x}$.)
  Schreibe kurz $V(\mathfrak{a})$ für den lokal geringten Raum 
  \begin{align*}
    \left(V(\mathfrak{a}),\imath_{x}(\mathcal{O}_{\Spec A/\mathfrak{a}})|_{V(\mathfrak{a})}\right) & \stackrel[\imath]{\cong}{\longleftarrow}\Spec(A/\mathfrak{a})
  \end{align*}

  Da $x\in V(\mathfrak{a})$, ist $\imath_{x}(\mathcal{O}_{\Spec A/\mathfrak{a}})|_{V(\mathfrak{a})}\overset{\cong}{\longrightarrow}i_{x}\mathcal{O}_{\Spec A/\mathfrak{a}}$.
\end{example}

\begin{example}[39]
  Sei $B$ ein Ring und $\mathfrak{b}\subset B$ Ideal. $V(\mathfrak{b}^{n})=V(\mathfrak{b})\subset\Spec B$
  als abgeschlossene Teilmenge hängt \emph{nicht }von $n\geq1$ ab,
  aber $\Spec(B/\mathfrak{b}^{n})=V(\mathfrak{b}^{n})$ als offenes
  Schemata sehr wohl!

  Etwa: $B=k[T]$, $b=(T)$ mit $k$ ein algebraisch abgeschlossener
  Körper. Für die abgeschlossenen Punkte von $\mathbb{A}_{k}^{1}=\Spec k[T]$
  gilt:
  \begin{align*}
    \Spec(k[T]) & \longleftrightarrow k\\
    \mathfrak{b} & \longleftrightarrow0
  \end{align*}

  Sei $A=k[T]/(T^{n})$, $X=\Spec A=\{x\}$. Es gilt: 
  \begin{align*}
    \mathcal{O}_{X}(X) & =\mathcal{O}_{X,x}=A\\
    \kappa(x) & =k\\
    n>1:\ 0\neq\mathfrak{m}_{x} & =T\mod T^{n}
  \end{align*}

  Betrachte $X\subset\mathbb{A}_{k}^{1}$ als abgeschlossenes ,,Unterschemata``
  welches ,,konzentriert in einem Punkt`` ist. $B=k[T]$ ist $k$-Algebra
  von Funktionen auf $\mathbb{A}^{1}(k)$. (vgl. Beispiel 2.14.) Die
  Einschränkung einer solchen Funktion $f\in k[T]$ auf $X$ ist gegeben
  durch $k[T]\rightarrow k[T]/(T^{n})$. Wir unterscheiden:
  \begin{itemize}
  \item[$n=1$.] $k[T]/(T^{n})=k$, $f\mapsto f(0)$.
  \item[$n>1$.] $k[T]/(T^{n})\neq k$, $f\mapsto$ (,,Taylor-Entwicklung`` von
    $A$ um 0 der Länge $n-1$). $\{x\}\subset\mathbb{A}_{k}^{1}$ hat
    ,,infinitesimale Ausdehnung der Länge $n-1$ in $\mathbb{A}_{k}^{1}$``
  \end{itemize}
  Sei nun $\mathbb{A}_{k}^{2}:=\Spec(k[T,U])$ betrachtet als $\{(u,t)\mid u,f\in k\}=k^{2}$.
  Sei $\mathfrak{a}_{1}=(U)$, $\mathfrak{a}_{2}=(U-T^{n})$. Diese
  definieren:
  \[
    X_{1}=\{u,t)\in\mathbb{A}^{2}(k)\mid u=0\},\quad X_{2}=\{(u,t)\in\mathbb{A}^{2}(k)\mid u=t^{n}\}.
  \]

  Es ist $X_{1}\cap X_{2}=\{(0,0)\}$ als Menge. Aber für $n>1$ treffen
  sich beide Mengen \emph{nicht} transversal! Als affine Schemata wird
  später der Schnitt als $\Spec k[T,U]/(\mathfrak{a}_{1}+\mathfrak{a}_{2})$
  definiert, also eine präzisere Beschreibung als Durchschnitt.
\end{example}

>>>>>>> d388b9febcfa9ca4bcebc412bc8c9e28b5b54b70

\chapter{Schemata}
\label{chap:schemata}

<<<<<<< HEAD
\section{Schemata}

\begin{defn}
	Ein Schemata ist ein lokal geringter Raum $(X,\cat{O}_X)$, der eine offene Überdeckung $(U_i)_{i\in I}$
	besitz derart alle lokal geringter Räume $(U_i,\cat{O}_{X|U_i})$ affine Schemata sind.\\
	Für ein Schemata $S$ besitz $\textbf{Sch}/_{S}$ ~~ \textbf{Kategorie der Schemata über} $S$ oder $S$-\textbf{Schemata}\\
	\begin{itemize}
		\item Morphismen $X \rightarrow S$ von Schemata
	\end{itemize}
\end{defn}


\section{offene Unterschemata}


\section{Morphismen in affine Schemata}


\section{Morphismen der Form $\Spec(K)\protect\longrightarrow X$}


\section{Verkleben von Schemata und disjunkte Vereinigung}


\section{Der projektive Raum als Schemata}


\include{Chapter3/AlgGeo1-Chapter3-7_nullstellenmenge-in-projektiven-raeme}

\include{Chapter3/AlgGeo1-Chapter3-8_topologische-eigenschaften}

\section{Noethische Schemata}


\section{Generische Punkte}


\section{Reduzierte und ganze Schemata}


\section{Schemata von endlichem Type über $k$}


\section{Prävaritäten als Schemata}


\chapter*{Unterschemata und Immersion (Einbettung)}
\section{offene/abgeschlossen Einbettung}


\section{Reduzierte Unterschemata}



\chapter{Faserprodukte}
\label{chap:faserprod}

\include{Chapter4/AlgGeo1-Chapter4-1-Der-Punkte-Fuktor}

\section{Yoneda Lemma}
\underline{Ziel}: $h_X$ beschreibt $X$ eindeutig.

\section{Faserprodukt in beliebiger Kategorie}

$\cat{C}$ Kategorie, $S \in \text{Oj}(\cat{C})$

=======
= Verkleben affiner Schemata

\section{Schemata}
\begin{defn}
  Ein Schemata ist ein lokal geringter Raum $(X,\mathcal{O}_{X})$,
  der eine offene Überdeckung $(U_{i})_{i\in I}$ besitzt, so dass alle
  lokal geringten Räume $(U_{i},\mathcal{O}_{X|U_{i}})$ affine Schemata
  sind. Für ein Schemata $S$ bezeichne $\schs$ die \textbf{Kategorie
    der Schemata über $S$} oder $S$-Schemata. Die Objekte dieser Kategorie
  sind Morphismen $X\rightarrow S$ von Schemata, und die Morphismen
  $\Hom(X\rightarrow S,Y\rightarrow S)$ sind Morphismen $X\rightarrow Y$
  von Schemata so dass
  \[
    \xymatrix{X\ar[rr]\ar[dr] &  & Y\ar[dl]\\
      & S
    }
  \]

  kommutiert. $X\rightarrow S$ heißt \textbf{Strukturmorphismus} des
  $S$-Schematas $X$. Ist $S=\Spec R$ affin, spricht man auch von
  $R$-Schemata oder Schemata über $R$. Die Menge der Morphismen $X\rightarrow Y$
  in $\schs$ bezeichnen wir mitt $\Hom_{S}(X,Y)$ bzw. $\Hom_{R}(X,Y)$
  falls $S=\Spec R$ affin ist.
\end{defn}

\section{Offene Unterschemata}

Erinnerung: $X=\Spec A$ affin $\Longrightarrow(D(f),\mathcal{O}_{X|D(f)})$
auch affin, und $D(f)$ Basis der Topologie.
\begin{prop}[2]
  Sei $X$ ein Schemata.
  \begin{enumerate}
  \item Ist $U\subset X$ eine offene Teilmenge, dann ist der lokal geringte
    Raum $(U,\mathcal{O}_{X\mid U})$ wieder ein Schemata. $U$ heißt
    ein \textbf{offenes Unterschemata}. Ist $U$ affin, dann heißt $U$
    \textbf{affines offenes Schemata}.
  \item Die zugrundlegenden topologische Räume der affinen offene Unterschemata
    bilden eine Basis der Topologie.
  \end{enumerate}
\end{prop}

\begin{proof}
  Es gibt eine Überdeckung $(U_{i})$ von $X$, d.d. $(U_{i},\mathcal{O}_{X|U_{i}})$
  affine Schemata, $\cong\Spec A$. Es gilt:
  \[
    U=\bigcup_{i}(U\cap U_{i})=\bigcup_{i,j\in I_{i}}D(f_{ij})
  \]

  wobei die letzte Gleichheit gilt wegen $\Spec(A_{i})\supset U\cap U_{i}=\bigcup_{j\in I_{i}}D(f_{ij})$,
  $f_{ij}\in A_{i}$.
\end{proof}
Zu $U\subset X$ offen gibt es einen kanonischen Morphismus von Schemata
\[
  (j,j^{\flat}):(U,\mathcal{O}_{X|U)}\longrightarrow(X,\mathcal{O}_{X})
\]

via der Inklusion $j:U\hookrightarrow X$ und $j^{\flat}:\mathcal{O}_{X}\rightarrow j_{\ast}(\mathcal{O}_{X\mid U})$.
Für $V\subseteq X$ offen ergibt $\res_{V\cap U}^{V}$ einen Ringhomomorphismus:
\[
  \Gamma(V,\mathcal{O}_{X})\rightarrow\Gamma(V\cap U,\mathcal{O}_{X})=\Gamma(j^{-1}(V),\mathcal{O}_{X|U})=\Gamma(V,j_{\ast}\mathcal{O}_{X|U}).
\]

Eine affine offene Überdeckung eines Schematas $X$ ist eine Überdeckung
$X=\bigcup U_{i}$, sodass alle $U_{i}$ affine offene Unterschemata
sind.
\begin{lem}[3]
  Sei $X$ ein Schemata, und seien $U,V\subset X$ affin offene Unterschemata.
  Dann existiert für jedes $x\in U\cap V$ eine Umgebung $x\in W\subset U\cap V$
  offenes Unterschema, welches gleichzeitig prinzipal offen in $U$
  \emph{und} $V$ ist.
\end{lem}

\begin{proof}
  Sei ohne Einschränkung $V\subset U$ (sonst ersetze $V$ durch eine
  prinzipiale offene Teilmenge von $V$ welche $x$ enthält). Wähle:
  \[
    \xymatrix{f\in\Gamma(U,\mathcal{O}_{X})\ar[d]^{\res} & \text{d.d. }x\in D(f)\subset V\\
      f|_{V}\in\Gamma(V,\mathcal{O}_{X}) & D_{U}(f)=D_{V}(f|_{V})
    }
  \]

  denn $\Gamma(U,\mathcal{O}_{X})_{f}=\mathcal{O}_{X}(D_{U}(f)$, $\Gamma(V,\mathcal{O}_{X})_{f|_{V}}=\mathcal{O}_{X}(D_{V}|f_{|_{V}})$.
\end{proof}


\section{Morphismen in affinen Schemata hinein}
\begin{prop}[4]
  Sei $X$ ein Schemata, $Y=\Spec B$ ein affines Schemata. Dann ist
  die Abbildung
  \begin{align*}
    \Hom(X,Y) & \overset{\cong}{\longrightarrow}\Hom_{\ring}(B,\Gamma(X,\mathcal{O}_{X})),\\
    (f,f^{\flat}) & \longmapsto f_{Y}^{\flat}
  \end{align*}

  eine Bijektion.
\end{prop}

\begin{prop}[5, Verkleben von Morphismen]
  Seien $X,Y$ lokal geringte Räume. Für $U\subset X$ offen definiert
  \[
    \mathcal{F}:U\mapsto\Hom(U,Y)=\{(U,\mathcal{O}_{X|U})\rightarrow(Y,\mathcal{O}_{Y})\ \text{Morph. lokal ger. Räume\}}
  \]

  eine Garbe von Mengen auf $X$, d.h.
  \begin{enumerate}
  \item für eine offene Überdeckung $X=\bigcup_{i}U$, eine Familie $f_{i}:U_{i}\rightarrow Y_{i}$
    verkleben zu Morphismen
    \[
      f:X\rightarrow Y\Longleftrightarrow f_{i}|_{U_{i}\cap U_{j}}=f_{j}|_{U_{i}\cap U_{j}}
    \]
  \item $f$ ist eindeutig bestimmt.
  \end{enumerate}
\end{prop}

\begin{rem*}
  $\mathcal{G}:U\mapsto\Hom_{\ring}(B,\Gamma(U,\mathcal{O}_{X}))$ ist
  Garbe von Mengen.
\end{rem*}
\begin{proof}[Beweis von Proposition 5]
  Verkleben topologischer Räume + stetige Abbildung klar. $\checkmark$

  $\mathcal{O}_{Y}\rightarrow f_{\ast}\mathcal{O}_{X}$ lässt sich ebenfalls
  verkleben.
\end{proof}
% 
\begin{proof}[Beweis von Proposition 4]
  $X=\bigcup_{i}U_{i}$ sei eine affine offene Überdeckung. Nach Proposition
  2.35 ist $\Hom(U,Y)\rightarrow\Hom(B,\Gamma(U,\mathcal{O}_{X}))$
  eine Bijektion. Für $V\subset U_{i}\cap U_{j}$ kommutiert das Diagramm
  \[
    \xymatrix{\Hom(U,Y)\ar[r]^{\cong}\ar[d] & \Hom(B,\Gamma(U,\mathcal{O}_{X})\ar[d]\\
      \Hom(V,Y)\ar[r]^{\cong} & \Hom(B,\Gamma(V,\mathcal{O}_{X}))
    }
  \]

  da $\Gamma(-)$ funktoriell ist. Es folgt, dass $\mathcal{F}\rightarrow\mathcal{G}$
  ein Morphismus von Garben ist mit $\varphi_{U}:\mathcal{F}(U)\overset{\cong}{\rightarrow}\mathcal{G}(U)$
  für alle $U\in\mathcal{B}$, und $F\overset{\cong}{\rightarrow}\mathcal{G}$
  als Garbe. Somit $\mathcal{F}(X)\cong\mathcal{G}(X)$.
\end{proof}
$V\subset X$ offen beliebig, $\varphi_{V}=\underset{\underset{U\in B_{V}}{\longleftarrow}}{\lim}\varphi_{U}$.

Da $\mathbb{Z}$ kofinales Objekt in der Kategorie der Ringe ist ($\mathbb{Z}\overset{\exists_{1}}{\rightarrow}R$
für beliebige Ringe $R$), gilt:
\begin{cor}[6]
  Sei $X$ ein Schemata. $X$ besitzt einen eindeutig bestimmten Morphismus
  $X\rightarrow\Spec(\mathbb{Z})$, d.h. $\Spec(\mathbb{Z})$ ist ein
  terminales Objekt in der Kategorie der Schemata: Jedes Schemata ist
  ein $\mathbb{Z}$-Schemata.
\end{cor}

Weiterhin:
\begin{align*}
  \Hom(X,\underbrace{\Spec\mathbb{Z}[T]}_{\mathbb{A}_{\mathbb{Z}}^{1}}) & =\Hom_{\ring}(\mathbb{Z}[T],\mathcal{O}_{X}(X))=\Gamma(X,\mathcal{O}_{X})\\
  \Hom_{R}(X,\underbrace{\Spec R[T]}_{\mathbb{A}_{R}^{1}}) & =\Gamma(X,\mathcal{O}_{X})\text{ als }R\text{-Algebra für }R\text{-Schemata }X
\end{align*}

\section{Morphismen der Form $\Spec(K)\rightarrow X$}

Sei $X$ ein Schemata und sei $x\in U\subset X$ offene affine Umgebung
von $x$, z.B. $U=\Spec A$. Sei $\mathfrak{p}=\mathfrak{p}_{x}\subset A$.
Es folgt: $\mathcal{O}_{X,x}=\mathcal{O}_{U,x}=A_{\mathfrak{p}}$,
und der Homomorphismus $A\rightarrow A_{\mathfrak{p}}$ induziert
\[
  j_{x}:\Spec\mathcal{O}_{X,x}=\Spec A_{\mathfrak{p}}\longrightarrow\Spec A=U\subset X
\]

Morphismus von Schemata, welcher nach Proposition 2 unabhängig von
$U$ ist. Nach Proposition 2.22 ist
\begin{align*}
  j_{x}:\Spec\mathcal{O}_{X,x}\overset{\cong}{\longrightarrow}Z & =\{x'\in X\mid x'\text{ Verallgemeinerung von }x\}\\
  (x\in\{x'\}\Leftrightarrow\mathfrak{p}_{x'}\subset\mathfrak{p}_{x})\  & =\bigcap_{x\in U\subseteq_{\text{off.}}X}U
\end{align*}

Sei $\kappa(x)=\mathcal{O}_{X,x}/\mathfrak{m}_{x}$. Die Abbildung
$\mathcal{O}_{X,x}\rightarrow\kappa(x)$ induziert einen Morphismus
von Schemata
\begin{align*}
  i_{x}:\Spec\kappa(x) & \longrightarrow\Spec\mathcal{O}_{X,x}\longrightarrow X\\
  \{\text{pt}\} & \longmapsto x
\end{align*}

Nun sei $K$ ein beliebiger Körper, und $f:\Spec K\rightarrow X$
ein beliebiger Morphismus mit $f(\text{\{pt\}})=x\in X$. Dieser induziert
einen lokalen Homomorphismus
\[
  \xymatrix{\mathcal{O}_{X,x}\ar[r]\ar[d] & K=\mathcal{O}_{\Spec(K),(0)}\\
    \kappa(x)\ar[ur]_{\imath}
  }
\]

d.h. $f$ faktorisiert als $f=i_{x}\circ(\Spec\imath):\Spec K\rightarrow\Spec\kappa(x)\rightarrow X$.
Damit haben wir:
\begin{prop}[7]
  Die Abbildung
  \[
    \Hom(\Spec K,X)\longrightarrow\{(x,\imath):x\in X,\ \imath:\kappa(x)\rightarrow K\}
  \]

  ist eine Bijektion.
\end{prop}

\begin{proof}
  Umgekehrt bilden wir:
  \[
    (x,\imath:\kappa(x)\rightarrow K)\longrightarrow(\Spec K\overset{\Spec\imath}{\rightarrow}\Spec\kappa(x)\overset{i_{x}}{\rightarrow}X).
  \]
\end{proof}

\section{Verkleben von Schemata und disjunkte Vereinigung}
\begin{defn}
  Ein \textbf{Verklebe-Datum} von Schemata besteht aus:
  \begin{itemize}
  \item einer Indexierung $I$;
  \item ein Schemata $U_{i}$ für $i\in I$;
  \item ein affines Unterschemata $U_{ij}\subset U$ für alle $i,j\in I$;
  \item einen Isomorphismus $U_{ij}\stackrel[\cong]{\varphi_{ji}}{\longrightarrow}U_{ji}$
    für alle $(i,j)\in I\times I$, sodass:
    \begin{enumerate}
    \item $U_{ii}=U_{i}$ für alle $i\in I$;
    \item (Kozykel-Bedingung): $\varphi_{kj}\circ\varphi_{ji}=\varphi_{ki}$
      auf $U_{ij}\cap U_{ik}$, für alle $i,j,k\in I$.
    \end{enumerate}
  \end{itemize}
\end{defn}

Für die Kozykel-Bedingung soll implizit gelten:
\begin{align*}
  \varphi_{ji}(U_{ij}\cap U_{ik}) & \subseteq U_{jk}\\
  i=j=k & \Rightarrow\varphi_{ii}=\id_{U_{i}},\\
  \varphi_{ij}^{-1} & =\varphi_{ji},\text{ und}\\
  \varphi_{ji}:U_{ij}\cap U_{ik} & \overset{\cong}{\rightarrow}U_{ji}\cap U_{jk}
\end{align*}

\begin{prop}[9]
  Zu einem Verklebe-Datum $((U_{i})_{i\in I},(U_{ij})_{i,j\in I},(\varphi_{ij})_{i,j\in I})$
  gibt es ein Schemata $X$ zusammen mit Morphismen $\psi_{i}:U_{i}\rightarrow X$,
  sodass
  \begin{itemize}
  \item für alle $i\in I$ induziert $\psi_{i}$ einen Isomorphismus von $U_{i}$
    auf offene Unterschemata von $X$;
  \item $\psi_{j}\circ\varphi_{ji}=\psi_{i}$ auf $U_{ij}$ für alle $i,j\in I$;
  \item $X=\bigcup_{i}\psi_{i}(U)$;
  \item $\psi_{i}(U_{i})\cap\psi_{j}=\psi_{i}(U_{ij})=\psi_{j}(U_{ji})$ für
    alle $i,j\in I$.
  \end{itemize}
  $(X,\psi_{i\in I})$ ist eindeutig bis auf eindeutige Isomorphie bestimmt.
\end{prop}

Zusammen mit Proposition 5 folgt die universelle Eigenschaft: Für
$(T,\xi_{i}:U_{i}\rightarrow T)$ mit $\xi_{i}$ welche Isomorphismen
\[
  U_{i}\overset{\cong}{\rightarrow}\text{\{offenes Unterschemata von }T\}
\]

induzieren, sodass $\xi_{j}\circ\varphi_{ji}=\xi_{i}$ auf $U_{ij}$
für alle $i,j\in I$, dann gibt es einen eindeutigen Morphismus $\xi:X\rightarrow T$
mit $\xi\circ\psi_{i}=\xi_{i}$ für alle $i\in I$. ($\Longrightarrow$
Eindeutigkeit von Proposition 9)

\begin{proof}
Als topologischer Raum: $\coprod_{i\in I}U_{i}/\sim$ mit $x_{i}\in U_{i}\sim x_{j}\in U_{j}:\Leftrightarrow x_{i}\in U_{ij}$,
$x_{j}\in U_{ji}$ und $x_{j}=\varphi_{ji}(x_{i})$. Nach Eigenschaft
$(b)$ ist $\sim$ eine Äquivalenzrelation. Dann sind $\psi_{i}:U_{i}\rightarrow X$
injektiv. Ferner haben wir $\forall i,j\in I$ die Eigenschaft $\psi_{i}(U_{i})=\psi_{i}(U_{i})\cap\psi_{j}(U_{j})$.

$X$ hat also als topologischer Raum die Quotiententopologie, d.h.
die feinste Topologie sodass alle Abbildungen $\psi_{i}$ stetig sind.
$U\subset X$ offen genau dann, wenn $\psi_{i}^{-1}(U)\subset U_{i}$
dort offen sind $\forall i\in I$. Insbesondere sind $\psi_{i}(U_{i})$
und $\psi_{i}(U_{j})=\psi_{i}(U_{i})\cap\psi_{j}(U_{j})$ offen in
$X$.

Als lokal geringter Raum: ``Verkleben der Strukturgarben auf $U_{i}$''.
$\mathcal{O}_{X}$ ist eindeutig auf einer Basis $B$ der Topologie
definiert. Ohne Einschränkung reicht es hier, die Schnitte nur auf
$U\subset X$ offen mit $U\subset\psi_{i}(U_{i})$ für ein $i\in I$.
In dem Fall:
\[
  \mathcal{O}_{X}(U)=\mathcal{O}_{U_{i}}(\psi_{i}^{-1}(U))
\]

Für $U\subset\psi_{i}(U_{i})\cap\psi_{j}(U_{j})$
\[
  \xymatrix{U_{ij}\ar@{^{(}->}[r]\ar[d]_{\cong} & U_{i}\\
    U_{ji}\ar@{^{(}->}[r] & U_{i} & X\supset U
  }
\]

Dann gilt:

\[
  \mathcal{O}_{U_{i}}(\psi_{i}^{-1}(U))=\mathcal{O}_{U_{ij}}(\psi_{i}^{-1}(U))\cong\mathcal{O}_{U_{ji}}(\psi_{j}^{-1}(U))=\mathcal{O}_{U_{j}}(\psi_{j}^{-1}(U))
\]

Es folgt: $\mathcal{O}_{X}(U)$ unabhängig von Wahlen von $i$! Wir
halten damit $\mathcal{O}_{X}$ Ringgarbe auf $X$, sodass $(X,\mathcal{O}_{X})$
lokal geringter Raum da $(U_{i},\mathcal{O}_{U_{i}})$ lokal geringter
Raum $\forall i\in I$, $U_{i}\xrightarrow[\psi_{i}]{\cong}(\psi_{i}(U_{i}),\mathcal{O}_{X}|_{\psi_{i}(U_{i})})$
als lokal geringter Raum. Damit ist $X$ ein Schema und $X=\cup U_{i}$.

Spezialfall: $U_{ij}=\emptyset$ für alle $i\neq j\in I$, $\coprod U_{i}$
``disjunkte Vereinigung''.
\end{proof}
\begin{example}[10]
  $X_{1},\ldots,X_{n}$ affine Schemata, $X_{i}=\Spec A_{i}$. Dann
  ist
  \[
    \coprod X_{i}\cong\Spec\left(\prod_{i=1}^{n}A_{i}\right)\text{ offen}.
  \]

  (nicht für unendlich viele affin!)
\end{example}

\begin{example}[11]
  $I=\{1,2\}$, $U_{12}\subset U_{1}\xrightarrow{\varphi}U_{21}\subset U_{2}$.
  \[
    X\underset{\text{offen}}{\subset}V=U_{1}\cup_{\varphi}U_{2},
  \]

  $\Gamma(V,\mathcal{O}_{X})=\{(s_{1},s_{2})\in\Gamma(V\cap U_{1},\mathcal{O}_{U_{1}})\times\Gamma(V\cap U_{2},\mathcal{O}_{U_{2}})\}$,
  $\varphi^{\flat}(S_{2}|_{U_{21}\cap V})=S_{1}|_{U_{12}\cap V}$.
\end{example}

\textbf{Affine Gerade mit Doppelpunkt}: $k$ Körper.
\begin{align*}
  U_{1} & =U_{2}=\mathbb{A}_{k}^{1}=\Spec(k[T])\cong x\text{ abg.}\\
  U_{12} & :=U_{1}\backslash\{x\}\\
  U_{21} & :=U_{2}\backslash\{x\},\ \varphi=\id
\end{align*}

$X=U_{1}\cup_{\varphi}U_{2}$.  
\textbf{Aufgabe.} $X$ ist \textbf{nicht} affin!

\section{Der projektive Raum als Schema }

Sei $R$ Ring. $\mathbb{P}_{R}^{n}$ Verklebung von $(n+1)$-Kopien
des 
\begin{align*}
  \mathbb{A}_{R}^{n} & =\Spec(R[T_{1},\ldots,T_{n}])\\
  \shortparallel\\
  \Spec\left(R\left[\frac{X_{0}}{X_{i}},\ldots,\frac{\hat{X}_{i}}{X_{i}},\ldots,\frac{X_{n}}{X_{i}}\right]\right)=U_{i}, & i=0,\ldots,n
\end{align*}

Verklebungs Datum:
\begin{align*}
  B:= & R[X_{0},\ldots,X_{n},X_{0}^{-1},\ldots,X_{n}^{-1}]\\
  U_{ij} & :=D\left(\frac{X_{j}}{X_{i}}\right)\subset U_{i},\ \text{OE}\ i\neq j\leq n,\\
  U_{ii} & =U_{i},\ \varphi_{ii}=\id
\end{align*}

$\varphi_{ji}:U_{ij}\rightarrow U_{ji}$ definiert durch
\[
  \xymatrix{R\left[\frac{X_{0}}{X_{i}},\ldots,\frac{\hat{X}_{i}}{X_{i}},\ldots,\frac{X_{n}}{X_{i}}\right]_{\frac{x_{j}}{x_{i}}}\ar@{=}[r]^{''\id''}\ar@{^{(}->}[dr] & R\left[\frac{x_{0}}{x_{j}},\ldots,\frac{\hat{x}_{j}}{x_{j}},\ldots,\frac{x_{n}}{x_{j}}\right]_{\frac{x_{i}}{x_{j}}}\ar@{^{(}->}[d]\\
    & B
  }
\]

Mit ``$\id$'' folgt: Kozykelbedingung automatisch, $U_{i}\rightarrow\Spec(R)$
verkleben von $\mathbb{P}_{R}^{n}\rightarrow\Spec(R)$ mit $\mathbb{P}_{R}^{n}:=\coprod U_{i}/\sim$
Schema (über $R$). ``Der projektive Raum relativer Dimension $n$
über $R$''.
\textbf{Aufgabe.}
$R\xrightarrow{\sim}\Gamma(\mathbb{P}_{R}^{n},\mathcal{O}_{\mathbb{P}_{R}^{n}})$
(Strukturgarbe) d.h. für $n>0$ ist $\mathbb{P}_{R}^{n}$ nicht affin (mit
$\mathbb{P}_{R}^{n}=\Spec(R)$).

\section{Nullstellenmenge im projektiven Raum }

Sei $I\subset R[X_{0},\ldots,X_{n}]$ homogenes Ideal, d.h. erzeugt
von homogenen Elementen.

\[
  \Leftrightarrow I=\bigoplus_{d}I\cap R[X_{1},\ldots,X_{n}]_{d})
\]

Ziel: $V_{+}(I)\rightarrow\mathbb{P}_{R}^{n}$ Morphismus von Schemata
\begin{align*}
  R[X_{0},\ldots,X_{n}] & \longrightarrow R\left[\frac{X_{0}}{X_{i}},\ldots,\frac{\hat{X_{i}}}{X_{i}},\ldots,\frac{X_{n}}{X_{i}}\right]=\Gamma(U_{i},\mathcal{O}_{U_{i}})\\
  I & \longmapsto\Phi_{i}(I)\text{ das vom Bild von }I\text{ erzeugte Ideal}
\end{align*}

Verklebe $V_{i}:=\Spec(\Gamma(U_{i},\mathcal{O}_{U_{i}})/\Phi_{i})\subseteq U_{i}$
entlang
\[
  V_{ij}=D_{V_{i}}\left(\frac{X_{j}}{X_{i}}\right)\xrightarrow{\cong}V_{ji}
\]

Beachte: $f\in I$, $\deg(f)=d$, $X_{i}^{d}\Phi_{i}(f)=X_{j}^{d}\Phi_{j}(f)$,
d.h. $\Phi_{i}(f)$ und $\Phi_{j}(f)$ unterscheiden sich in einer
Einheit auf $D\left(\frac{X_{i}}{X_{j}}\right)$.

$\Longrightarrow\Phi_{i}(I)=\Phi_{j}(I)$ in $\Gamma(U_{ij},\mathcal{O}_{U_{i}})$.
$\Longrightarrow V_{ij}=V_{ji}$ und Kozykelbedingungen überträgt
sich von $U_{ij}\subset U_{i}$.

$\Longrightarrow$ Verkleben liefert Schema $V_{+}(I)$ + 

D.h. jedes solche $I$ definiert ein Schema $V_{+}(I)\rightarrow\mathbb{P}_{R}^{n}$.

\chapter*{Grundlegende Eigenschaften von Schemata und Morphismen}

\section{Topologische Eigenschaften }
\begin{defn}[12]
  Ein Schema $X$ heißt \textbf{zusammenhängend}, \textbf{quasi-kompakt}
  bzw. \textbf{irreduzibel}, falls der unterliegende topologische Raum
  diese Eigenschaft besitzt.
\end{defn}

\begin{itemize}
\item Nach Proposition II.5 ist jedes affine Schema quasi-kompakt.
\item $\coprod_{i=0}\Spec(R)$ ist \emph{nicht }quasi-kompakt.
\end{itemize}
\begin{defn}[13]
  $f:X\rightarrow Y$ heißt \textbf{injektiv} (surjektiv, bijektiv),
  falls die zugrendlegende stetige Abbildung diese Eigenschaft hat.
  Ebenso für ``offen'', ``abgeschlossen'', ``Homömorphismus''.
\end{defn}

Warnung: Homömorphismen von Schemata sind im Allgemeinen \emph{keine
}Isomorphismen!

\section{Noethersche Schemata }
\begin{defn}[14]
  Ein Schema heißt \textbf{lokal noethersch}, falls eine affine offene
  Überdeckung $X=\bigcup U_{i}$ existiert, d.d. alle $\Gamma(U_{i},\mathcal{O}_{X})$
  (affine Koordinatenringe) \textbf{noethersch} sind. $X$ heißt \textbf{noethersch},
  falls zusätlich quasi-kompakt.
\end{defn}

Faktum: Lokalisierung noetherscher Ringe bleiben noethersch.

$\Rightarrow a)$ Jedes lokal noethersche Schema besitzt eine Basis
der Topologie aus noetherschen affin offenen Unterschemata.

b) $X$ lokal noethersch. Dann ist $\mathcal{O}_{X,x}$ noethersch
$\forall x\in X$.

Für offene Schemata gilt ferner: lokal noethersch $\Rightarrow$ noethersch.

\begin{prop}[15]
  Für $X\subset\Spec A$ affin gilt:
  \[
    X\text{ noethersch }\Leftrightarrow A\text{ noethersch}
  \]
\end{prop}

\begin{proof}
  \mbox{}
  \begin{itemize}
  \item[``$\Leftarrow$''] $X$ überdeckt sich selbst mit $\Gamma(X,\mathcal{O}_{X})=A$ noethersch.
  \item[``$\Rightarrow$''] Sei $I\subset A$ beliebiges Ideal. Zu zeigen: $I$ ist endlich erzeugt.
    Nach Voraussetzung ist
    \[
      X=\bigcup_{i=1}^{n}\Spec A_{i},\quad A_{i}\text{ noethersch}.
    \]
    Ohne Einschränking: $A_{i}=A_{f_{i}}$ und noethersch. Daraus folgt:
    $J_{i}=IA_{f_{i}}=I_{f_{i}}$ sind endlich erzeugt, Behauptung folgt
    aus Lemma 16.
  \end{itemize}
\end{proof}
\begin{lem}[16]
  $\Spec(A)=\cup_{i\in I}D(f_{i})$, $\#I<\infty$, $M$ $A$-Modul.
  Dann:
  \[
    M\text{ e.e. über }A\Leftrightarrow M_{f_{i}}\text{ e.e. }A_{f_{i}}\text{-Modul }\forall i\in I.
  \]
\end{lem}

\begin{proof}
  \mbox{}
  \begin{itemize}
  \item[``$\Rightarrow$'' ] Endlich erzeugt heißt $A^{n}\twoheadrightarrow M$, Lokalisierung
    exakt also $A_{f_{i}}^{n}\twoheadrightarrow M_{f_{i}}$ exakt.
  \item[``$\Leftarrow$''] $M_{f_{i}}$ werden von $\frac{m_{ij}}{f_{i}^{n_{ij}}}$, $j=1,\ldots,r_{i}$,
    $m_{ij}\in M$, $n_{ij}\in\mathbb{N}_{0}$ als $A_{f_{i}}$-Modul
    erzeugt.

    $\Rightarrow N:=\langle m_{ij}\rangle_{A}\subset M$ ist endlich erzeugt
    und $N_{f_{i}}=M_{f_{i}}$.

    $\Rightarrow(M/N)_{\mathfrak{p}}=(M_{f_{i}}/N_{f_{i}})_{\mathfrak{p}}=0$
    für alle Primideale $\mathfrak{p}\in\Spec A$.

    $\Rightarrow$ (Lokal-Global-Prinzip aus der kommutativen Algebra)
    $N=M$.

  \end{itemize}
\end{proof}
\begin{rem*}
  $X$ noethersches Schema. Dann ist $X$ als topologischer Raum noethersch.
\end{rem*}
\begin{proof}
  Für $X$ affin klar, sonst $X=\cup_{i=1}^{r}X_{i}$, $X_{i}=\Spec(A_{i})$
  noethersch. Sei
  \[
    X\supseteq Z_{1}\supseteq\cdots\supseteq Z_{n}\supseteq\cdots
  \]

  absteigende Kette abgeschlossener Teilmengen. $(Z_{j}\cap X_{i})_{j}$
  absteigende Kette abgeschlossener Teilmengen in $X_{i}$.

  $\Longrightarrow$ (endliche Überdeckung) $\exists N$ d.d. $Z_{j}\cap X_{i}=Z_{N}\cap X_{i}$
  für alle $j\geq N$.

  $\Longrightarrow Z_{j}=Z_{N}$.
\end{proof}
\begin{cor}[17]
  Sei $X$ (lokal) noethersches Schema, $U\subset X$ offenes Unterschema.
  Dann ist $U$ ein (lokal) noethersches Schema.
\end{cor}

\begin{proof}
  Lokal noethersch $X=\bigcup U_{i}$, $U_{i}\cap U=\bigcup D(f_{i})$.
  Sei $X$ noethersch. Dann ist der topologische Raum $X$ noethersch.
  Nach Lemma $I.20$ ist dann jede offene Teilmenge quasi-kompakt.
\end{proof}

\section{Generische Punkte }
\begin{prop}[18]
  Die Abbildung
  \begin{align*}
    X & \longrightarrow\{Z\subset X\mid\text{abg., irred.}\}\\
    x & \longmapsto\overline{\{x\}}
  \end{align*}

  ist eine Bijektion, d.h. jede irreduzible abgeschlossene Teilmenge
  enthält genau einen generischen Punkt.
\end{prop}

\begin{proof}
  Gilt für affine Schemata nach Korollar II.7. Sei $Z\subset X$ irreduzibel,
  abgeschlossen sowie $U\subset X$ affin offen mit $Z\cap U\neq\emptyset$.

  $\Longrightarrow\overline{Z\cap U}^{X}=Z$, da $Z$ irreduzibel.

  $\Longrightarrow Z\cap U$ irreduzibel mit generischen Punkt $x$,
  $\overline{\{x\}}^{Z\cap U}=Z\cap U$.

  $\Longrightarrow\overline{\{x\}}^{X}=Z$.

  Umgekehrt: Sei $z\in Z$ generischer Punkt.

  $\Longrightarrow[U\subset X$ offen mit $U\cap Z\neq\emptyset$ $\Rightarrow z\in U]$
  d.h. Eindeutigkeit im affinen Fall impliziert allgemeiner Fall.
\end{proof}
``Generische Punkte reduzieren gewisse Aussagen auf das Studium von
\emph{einem} Punkt''.
\begin{prop}[19]
  Sei $f:X\rightarrow Y$ offener Morphismus von Schemata. Sei $Y=\overline{\{\eta\}}^{Y}$
  irreduzibel. Dann:
  \[
    f^{-1}(\eta)\text{ irreduzibel}\Leftrightarrow X\text{ irreduzibel}
  \]
\end{prop}

\begin{proof}
  $f$ offen $\Rightarrow\overline{\{f^{-1}(x)\}}=f^{-1}(\overline{\{\eta\}})=f^{-1}(Y)=X$.
  Mit Lemma I.14: $Z$ irreduzibel $\Leftrightarrow\overline{Z}$ irreduzibel.
\end{proof}
Topologische Räume von Schemata sind fast nie Hausdorffsch, aber:
\begin{prop}[20]
  Sei $X$ Schema. Dann ist der unterliegende topologische Raum ein
  $T_{0}$-Raum, d.h.
  \[
    \forall x\neq y\in X\ \exists U\subset X\text{ offen, mit \textbf{entweder }}x\in U\text{ oder }y\in U.
  \]
\end{prop}

\begin{proof}
  Ohne Einschränkung: $X$ affin, $x=\mathfrak{p}_{x}$, $y=\mathfrak{p}_{y}\in\Spec(\Gamma(X,\mathcal{O}_{X}))$.
  Falls $\mathfrak{p}_{x}\subsetneq\mathfrak{p}_{y}$ wähle
  \[
    \mathfrak{p}_{x}\in U=X\backslash\underbrace{V(\mathfrak{p}_{y})}_{\ni\mathfrak{p}_{y}}
  \]

  andernfalls $\exists f\in\mathfrak{p}_{x}\backslash\mathfrak{p}_{y}$,
  d.h. $U=D(f)$ enthält $y$ aber nicht $x$.
\end{proof}
Später: Separiertheit von Schemata als ``Hausdorffsch''-Ersatz.

\section{Reduzierte und ganze Schemata }
\begin{defn}[21]
  Ein Schema $X$ heißt 
  \begin{enumerate}
  \item \textbf{reduziert}, falls alle $\mathcal{O}_{X,x}$, $x\in X$, reduzierte
    Ringe sind.
  \item \textbf{ganz}, falls $X$ reduziert und irreduzibel ist.
  \end{enumerate}
\end{defn}

\begin{prop}[22]
  \mbox{}
  \begin{enumerate}
  \item $X$ schema ist reduziert (ganz) $\Leftrightarrow\Gamma(U,\mathcal{O}_{X})$
    reduziert (integer) für alle $U\subseteq X$ offen.
  \item Sei $X$ ganz. Dann ist der Halm $\mathcal{O}_{X,x}$ integer $\forall x\in X$.
    (Die Umkehrung ist im Allgemeinen falsch!)
  \end{enumerate}
\end{prop}

\begin{proof}
  \mbox{}
  \begin{enumerate}
  \item \textbf{reduzibel,} ``$\Rightarrow$''. $f\in\Gamma(U,\mathcal{O}_{X})$
    mit $f^{n}=0$. Angenommen, $f\neq0$. Dann gibt es ein $x\in U$
    mit $f_{x}\neq0$ in $\mathcal{O}_{X,x}$, $f_{x}^{n}=0$. Widerspruch

    \textbf{reduzibel,} ``$\Leftarrow$''. Sei $\overline{f}\in\mathcal{O}_{X,x}$
    nilpotent. Dann gibt es ein $x\in U\subset X$ offen und $f\in\Gamma(U,\mathcal{O}_{X})$
    mit $f_{x}=\overline{f}$. Ohne Einschränkng: $f$ nilpotent (mit
    $U$ verkleinern sodass $f^{n}|_{U}=0$). Nach Voraussetzung ist dann
    $f=0$, also $\overline{f}=0$.

    \textbf{ganz,} ``$\Rightarrow$''. Sei $X$ ganz. Dann ist $U\subset X$
    offen ganz nach den Definitionen. Daher reicht es zu zeigen, dass
    $\Gamma(X,\mathcal{O}_{X})$ integer ist. Seien $f,g\in\mathcal{O}_{X}(X)$
    mit $fg=0$. Dann ist $X=V(f)\cup V(g)$. $X$ ist irreduzibel, also
    etwa $X=V(f)$. \emph{Behauptung}: $f=0$.

    Da Verschwinden aufgrund des Garbenaxioms eine lokale Frage ist, setze
    ohne Einschränking $X=\Spec A$ affin. Es folgt: $f\in\bigcap_{\Spec A}\mathfrak{p}=\sqrt{(0)}=0$.

    \textbf{ganz, }``$\Leftarrow$''. $\Gamma(U,\mathcal{O}_{X})$ integer,
    also reduziert. Nach (1, reduziert) ist $X$ reduziert. Angenommen
    es gibt $\emptyset\neq U_{1},U_{2}\subset X$ offen mit $\emptyset=U_{1}\cap U_{2}$.
    Nach den Garbenaxiomen enthält dann $\Gamma(U_{1}\cup U_{2},\mathcal{O}_{X})=\Gamma(U_{1},\mathcal{O}_{X})\times\Gamma(U_{2},\mathcal{O}_{X}$)
    Nullteiler $(1,0)\cdot(0,1)=0$. Widerspruch.
  \item Folgt aus 1, da $A$ integer, $0\notin S$. Es folgt: $A_{S}$ integer
    ($\subseteq\Quot(A)$).
  \end{enumerate}
\end{proof}
\begin{rem*}
  $X=\Spec A$ ganz $\Leftrightarrow A$ integer, $\eta\in X$ generischer
  Punkt $\Leftrightarrow(0)\subset A$. Es ist $\mathcal{O}_{X,\eta}=A_{(0)}=\Quot(A)$,
  d.h. für jedes ganze Schema $X$ gilt: $\mathcal{O}_{X,\eta}$ ist
  Körper (mit generischer Punkt $\eta$).
\end{rem*}
\begin{defn}[23]
  $X$ ganz, $\eta\in X$ generischer Punkt. Dann heißt $K(X):=\mathcal{O}_{X,\eta}$
  der \textbf{Funktionenkörper} von $X$.
\end{defn}

\begin{prop}[24]
  Sei $X$ noethersches irreduzibles Schema, $\eta\in X$ generischer
  Punkt. Dann sind äquivalent:
  \begin{enumerate}
  \item $\mathcal{O}_{X,\eta}$ ist reduziert.
  \item $\exists\emptyset\neq U\subset X$ reduziertes offenes Unterschema.
  \end{enumerate}
  Für $(ii)$ sagt man auch: $\mathcal{O}_{X,x}$ ist \textbf{generisch}
  reduziert.
\end{prop}

\begin{proof}
  \mbox{}
  \begin{itemize}
  \item[$``\Rightarrow"$] Sei ohne Einschränkung $X=\Spec A$ affin, $A$ nach Voraussetzung
    noethersch, $\eta\leftrightarrow\mathfrak{p}$ eindeutiges minimales
    Primideal ($A$ irreduzibel). $\Longrightarrow\mathfrak{p}=(f_{1},\ldots,f_{n})_{A}$
    endlich erzeugt. $\Longrightarrow\frac{f_{i}}{1}\in\nil(A_{\mathfrak{p}})=(0)\subset A_{\mathfrak{p}}=\mathcal{O}_{X,\eta}$,
    da $\mathcal{O}_{X,\eta}$ reduziert. $\exists g\in A\backslash\mathfrak{p}$
    $\Longrightarrow$ d.d. $\frac{f_{1}}{1}=0\sim A_{g}$. $\Longrightarrow0=\nil(A)_{g}=\nil(A_{g})$,
    d.h. $A_{g}$ ist reduziert, d.h. $U:=D(g)$.
  \item[$``\Leftarrow"$] $\emptyset\neq U\subset X$ reduziert offen. $\Longrightarrow\eta\in U$
    s.d. $\mathcal{O}_{X,x}=\mathcal{O}_{X,\eta}$ reduziert.
  \end{itemize}
\end{proof}
\begin{rem*}
  Analog zeigt man: $X$ noethersches Schema und $\mathcal{O}_{X,x}$
  reduzibel für ein $x\in X$ $\Longrightarrow\exists x\in U\subset X$
  offen, d.d. $U$ reduziert ist.
\end{rem*}

\section*{Prävarietäten als Schema}

\textbf{Ziel: }``$X$ affine Varietät $\mapsto\Spec\Gamma(X,\mathcal{O}_{X})$``
(kein generischer Punkt, $\neq$ Schema). Verklebe zu: ``$X$ Prävarietät
über $k$ $\mapsto k$-Schema``. Welches Bild hat dieser Funktor?

\section{Schemata von endlichem Typ über $k$}

Sei $X$ affine Varietät über $k$, $k$ algebraisch Abgeschlossen.
Dann ist $A=\Gamma(X,\mathcal{O}_{X})$ eine endlich erzeugte $k$-Algebra.
\begin{defn}[25]
Sei $k$ Körper, $X\rightarrow\Spec(k)$ $k$-Schema. $X$ heißt:
\begin{itemize}
\item \textbf{lokal von endlichem Typ}, (l.v.e.T./$k$), falls eine affine
offene Überdeckung $X=\bigcup_{i\in I}U_{i}$ existiert, $U_{i}=\Spec A_{i}$,
mit $A_{i}$ endlich erzeugte $k$-Algebra für alle $i$.
\item \textbf{von endlichem Typ} (v.e.T$/k$), falls $X$ lokal von endlichem
Typ und quasi-kompakt ist.
\end{itemize}
\end{defn}

\begin{rem*}
Jedes $k$-Schema welches (lokal) von endlichem Typ ist, ist (lokal)
noethersch. (Da jede endlich erzeugte $k$-Algebra noethersch ist.)
\end{rem*}
\begin{prop}[26]
Sei $X$ $l.v.e.T/k$, $U\subset X$ offen affin. Dann ist $B:=\Gamma(U,\mathcal{O}_{X})$
eine endlich erzeugte $k$-Algebra.
\end{prop}

\begin{proof}
$U=\bigcup_{i=1}^{n}D(f_{i})$, $f_{i}\in B$ geeignet nach Lemma
3.3. $B$ ist endlich erzeugte $k$-Algebra $\Longrightarrow B_{f_{i}}=B\left[\frac{1}{f_{i}}\right]$
ist endlich erzeugte $k$-Algebra. Mit dem folgenden Lemma für $A=k$
folgt die Behauptung.
\end{proof}
\begin{lem}[28]
Sei $A$ ein Ring und $B$ eine $A$-Algebra, $\mathcal{L}:A\rightarrow B$
Ringhomomrphismus, $f_{1},\ldots,f_{n}\in B$ mit $(f_{1},\ldots,f_{n})=(1)$,
und so dass $B_{f_{i}}$ eine endlich erzeugte $A$-Algebra ist $\forall i$.
Dann ist $B$ eine endlich erzeugte $A$-Algebra.
\end{lem}

\begin{proof}
(Vergleiche Lemma 16.) Nach Voraussetzung gibt es ein $g_{i}\in B$
mit $\sum_{i}g_{i}f_{i}=1$. Da $B_{f_{i}}$ endlich erzeugt $\forall i$,
gibt es $b_{ij}$, $j\in J$ endlich, welche $B_{f_{i}}$ als $A$-Algebra
erzeugen. Setze $b_{ij}=c_{ij}/f_{i}^{m}$ mit $c_{ij}\in B$ für
$m\geq0$ geeignet (unabhäng von $i,j$).

Sei $C:=A$-Unteralgebra von $B$, erzeugt von $g_{i},f_{i},c_{ij}$,
d.h. endlich erzeugt über $A$. \emph{Behauptung}: $C=B$.

Sei $b\in B$. $\Longrightarrow$ $\exists N\gg0$ mit $f_{i}^{N}b\in C$
für alle $i$. Da $\sum_{i}g_{i}f_{i}=1$, ist $(f_{1},\ldots,f_{n})_{C}=(1)$.
Lemma 2.4 $\Longrightarrow$ $(f_{i}^{N},\ldots,f_{n}^{N})_{C}=(1)$.
$\Longrightarrow$ $\exists u_{1},\ldots,u_{n}\in C$ sodass $\sum_{i}u_{i}f_{i}^{N}=1$.
$\Longrightarrow b=\sum_{i}u_{i}\underbrace{f_{i}^{N}b}_{\in C}\in C$.
\end{proof}
\begin{prop}[28]
Sei $k$ algebraisch abgeschlossen, $X$ $k$-Schema l.v.e.T.$/k$.
Dann besteht die Menge der abgeschlossenen Punkte $|X|$ genau aus
den Punkten mit $\kappa(x)=k$, d.h. nach Proposition 7 gilt
\[
|X|=X(k)=\Hom_{k}(\Spec k,X).
\]
\end{prop}

\begin{proof}
Hilbert'scher Nullstellensatz $\Longrightarrow(x\in X$ abgeschlossen
$\Rightarrow\kappa(x)=k$). Daher reicht es zu zeigen: ($x\in X$
\emph{nicht} abgeschlossen, d.h. $\mathfrak{p}_{x}$ maximal $\Rightarrow\kappa(x)\neq k$).
Dazu: $\exists x\in U=\Spec(A)\subset X$ offen, mit $x$ \emph{nicht
abgeschlossen} in $U$. $\Longleftrightarrow\mathfrak{p}=\mathfrak{p}_{x}\in\Spec(A)$
\emph{nicht} maximal, d.h. $A/\mathfrak{p}_{x}$ ist \emph{kein} Körper.
$\Longrightarrow k\rightarrow(A/\mathfrak{p}_{x})\hookrightarrow\Quot(A/\mathfrak{p}_{x})=\kappa(x)$
ist \emph{echte} Inklusion.

Behauptung: $\kappa(x)$ ist nicht algebraisch abgeschlossen, d.h.
nicht abstrakt isomorph zu $k$. Denn: \emph{Noether-Normalisierung}:

$A/\mathfrak{p}$ ist endlich über $k[X_{1},\ldots,X_{n}]$. Nach
Lemma I.9 folgt $n>0$ (mit $A/\mathfrak{p}$ über $k$ ganz $\Longrightarrow A/\mathfrak{p}$
Körper). $\Longrightarrow\kappa(x)$ ist endliche Erweiterung von
$k(X_{1},\ldots,X_{n})$, $n>0$. $\Longrightarrow k$ nicht algebraisch
abgeschlossen ($[k(X_{1},\ldots,X_{n})(\sqrt[n]{X_{1}}):k(X)]\rightarrow\infty$,
$n\rightarrow\infty$)
\end{proof}
\begin{rem*}
Im Allgemeinen $\exists x\in U\subset X$ offen mit $\{x\}\subset U$
abgeschlossen, aber $\{x\}\subset X$ \emph{nicht} abgeschlossen ($X=\Spec\mathcal{O}$,
$\mathcal{O}$ DVR, $U=\{x\}$, $x=\eta$ generischer Punkt.) Für
$X$ lokal von endlichem Typ über $k$ (nicht notwendig algebraisch
abgeschlossen) kann nach der Proposition \emph{nicht} geschehen.
\end{rem*}

\section{Sehr dichte Teilmengen}
\begin{defn}[29]
  Sei $X$ topologischer Raum. Eine Teilmenge $Y\subset X$ heißt \textbf{sehr
    dicht}, falls die folgenden äquivalenten Bedingungen gelten:
  \begin{enumerate}
  \item $U\mapsto U\cap Y$ definiert eine Bijektion:
    \[
      \{\text{offenen Teilmengen in }X\}\leftrightarrow\{\text{offene Teilmengen in }Y\}.
    \]
  \item $F\mapsto F\cap Y$ definiert eine Bijektion:
    \[
      \{\text{abgeschlossene Teilmengen in }X\}\leftrightarrow\{\text{abgeschlossene Teilmengen in }Y\}.
    \]
  \item Für alle $F\subseteq X$ abgeschlossen gilt: $F=\overline{F\cap Y}$.
  \item Jede lokal abgeschlossene Teilmenge $Z\neq\emptyset$ von $X$ enthält
    einen Punkt aus $Y$.
  \end{enumerate}
\end{defn}

\begin{proof}
  Die Äquivalenz von $(i)$, $(ii)$ und $(iii)$ ist klar.
  \begin{itemize}
  \item $(iii)\Rightarrow(iv)$
  
  Für abgeschlossene Teilmengen $F'\subsetneq F$ von $X$ setze $Z:=F\backslash F'$.
    Angenommen $(F\cap Y)\backslash(F'\cap Y)=Z\cap Y=\emptyset$. $\Longrightarrow F\cap Y=F'\cap Y$.
    $(iii)\Longrightarrow F=F'$, Widerspruch.
  \item $(iv)\Rightarrow(ii)$
  
  Sei $F,F'\subset X$ abgeschlossen mit $F\cap Y=F'\cap Y$. $\Longleftrightarrow((F\cup F')\backslash(F\cap F'))\cap Y=\emptyset$.
    $\Longrightarrow(F\cup F')\backslash(F\cap F')=\emptyset$. $\Longrightarrow F=F'$.
  \end{itemize}
\end{proof}
\begin{prop}[30]
  Sei $X$ l.v.e.T über $k$ algebraisch abgeschlossen. Dann ist $|X|$
  sehr dicht in $X$.
\end{prop}

\begin{proof}
  Zeige: Bedingung $(iv)$. Sei $\emptyset\neq A\subset X$ lokal abgeschlossen.
  Ohne Einschränkung: 
  \[
    A\subset_{\text{abg. }}U=\Spec A\subset_{\text{off.}}X.
  \]

  Nach Voraussetzung ist $A$ endlich-erzeugte $k$-Algebra. $\emptyset\neq A=V(\mathfrak{a})$
  mit $\mathfrak{a}\subset\mathfrak{m}\subset A$ für ein maximales
  Ideal $\mathfrak{m}$. $\Longrightarrow V(\mathfrak{a})$ enthält
  abgeschlossenen Punkt $x\in\mathfrak{m}$. Proposition 28 $\Longrightarrow x$
  ist abgeschlossen in $X$, da $\kappa(x)=k$%
  \begin{comment}
    unlesbar
  \end{comment}
  . $\Longrightarrow A\cap|X|\neq\emptyset$.
\end{proof}

\section{Prävarietäten als Schemata}

Wir wollen einen Funktor von der Kategorie der Prävarietäten in die
Kategorie der Schemata sodass, wenn wir eine geweissen Unterkategorie
von $\sh$ betrachten, eine Äquivalenz von Kategorien entsteht.\medskip{}

\textbf{Erinnerung: }$k=\overline{k}$: $\mathbb{A}_{k}^{2}=\Spec(k[X,Y])$
besteht aus
\begin{itemize}
\item Punte des $\mathbb{A}^{2}(k)$ $\leadsto$ maximale Ideale, 0-dimensionale
  Teilmengen.
\item Irreduzible Kurve $f(x,y)=0$ $\leadsto$ Primideale, 1-dimensionale
  Teilmengen.
\item Generischer Punkt 0 $\leadsto$ 2-dimensionale Teilmengen.
\end{itemize}
Wie können wir die zusätzlichen Punkte für den Funktor
\[
  \pres\longrightarrow\schs
\]

präzisieren? Sei $X$ ein topologischer Raum, in dem alle Punkte abgeschlossen
sind. Betrachte 
\[
  t(X)=\{Z\subset X\mid Z\text{ irreduzibel abgeschlossen}\},
\]

versehen mit der Topologie: I$t(X)\supset t(Z)$, $Z\subseteq_{\text{abg.}}X$
bilden die abgeschlossenen Mengen. Überprüfe: $Z_{1},Z_{2},Z_{i}\subset X$
abgeschlossen $\Longrightarrow t(\cap_{i}Z_{i})=\cap_{i}t(Z_{i})$,
$t(Z_{1}\cup Z_{2})=t(Z_{i})\cup t(Z_{2})$. Ist $f:X\rightarrow Y$
stetig, so auch
\begin{align*}
  t(f):t(X) & \longrightarrow t(Y)\\
  Z & \longmapsto\overline{f(Z)}
\end{align*}

denn:
\begin{enumerate}
\item $\overline{f(Z)}$ irreduzibel: Sei $\overline{f(Z)}=A_{1}\cup A_{2}$,
  $A_{i}\neq\emptyset$. Dann existiert $z_{1},z_{2}\in Z$ mit $f(z_{i})\in A_{i}$,
  denn sonst gilt $f(z)\subseteq A_{1}$. $Z\subseteq f^{-1}(f(Z))$
  abgeschlossen. $\Longrightarrow Z=(f^{-1}(A_{1})\cap Z)\cup(f^{-1}(A_{2}\cap Z))$,
  Widerspruch.
\item Sei $t(Y')\subseteq t(Y)$ abgeschlossen. $t(f)^{-1}(t(Y))=\{Z\in t(X)$,
  $\overline{f(Z)}\in t(Y')\}$ denn:
  \begin{itemize}
  \item[,,$\subseteq$``] $\overline{f(Z)}\subset Y'$ $\Longrightarrow Z\in f^{-1}(\overline{f(Z))}\subset f^{-1}(Y)=t(f^{-1}(Y))$
  \item[,,$\supseteq$``] $z\in f^{-1}(Y)$ abgeschlossen $\Longrightarrow f(Z)\in\overline{f(Z)}\subset\overline{Y'}\subset Y'$.
  \end{itemize}
\end{enumerate}
Wir erhalten einen Funktor
\[
  t:\topcp\longrightarrow\Top
\]

Die irreduziblen Mengen von $f(X)$ sind gerade die $t(X)$, $Z\subseteq X$
irreduzibel. $Z\in t(Z)$ ist der eindeutige generische Punkt. Sei
\begin{align*}
  \alpha_{X}:X & \longrightarrow t(X)\\
  x & \longmapsto\{x\}\text{ irred. abg.}
\end{align*}

So ist die Abbildung
\begin{align*}
  \text{\{abg. Tm. von }f(X)\} & \longrightarrow\text{\{abg. Tm. von }X\}\\
  A=t(Z) & \longmapsto\alpha_{X}^{-1}(A)=\{x\in X:\{x\}\in t(Z)\}=Z
\end{align*}

eine Bijektion. $\Longrightarrow\alpha_{X}$ ist Homömorphismus von
$X$ auf die abgeschlossenen Punkte $|t(X)|$ von $t(X)$ {[}irred.
abg. Teilmengen $Z$ von $X$, die in $t(X)$ abgeschlossen sind.{]}

Es ist $\{Z\}=t(Z')$ für ein $Z'\subset X$ abgeschlossen. $\Longrightarrow$
Nur ein Punkt $x\in X$ in $Z$, sonst $\{x\}\subsetneq Z\subset Z'$
beide in $t(Z')$.

Es ist $|t(X)|\subset t(Y)$ eine sehr dichte Menge (Bijektion oben).
\begin{thm}[31]
  Der Funktor $X\mapsto(t(X),(\alpha_{X})_{\ast}\mathcal{O}_{X})$
  induziert eine Äquivalenz von Kategorien:
  \begin{align*}
    t:\{\prek\} & \overset{1:1}{\longleftrightarrow}\text{\{integere }k\text{-Schemata v. endl. Typ\}}\\
    \{\affk\} & \overset{1:1}{\longleftrightarrow}\text{\{affine }k\text{-Schemata v. endl. Typ}\}
  \end{align*}
\end{thm}

\begin{proof}
  Ist $X$ eine affine Varietät über $k$ mit $\Gamma(X)=A$, so ist
  $X=\maxspec(A)$. $\Longrightarrow t(X)=\Spec A$ (vgl Kapitel I),
  $\mathcal{O}_{X}(D(f))=A_{f}$, $f\in A$. $\Longrightarrow$ Behauptung
  im affinen Fall.

  Ist $f:X\rightarrow Y$ Morphismus von Prävarietäten, so erhalten
  wir 
  \begin{align*}
    t(f):t(X) & \longrightarrow t(Y),\\
    (\alpha_{Y})_{\ast}\mathcal{O}_{Y} & \longrightarrow t(f)_{\ast}((\alpha_{X})_{\ast}\mathcal{O}_{X})
  \end{align*}

  Morphismus lokal geringter Räume, da ein Morphismus von Garben auf
  $X$ und $Y$ durch Komposition von Abbildungen gegeben ist!

  Quasi-inverser Funktor $(X,\mathcal{O}_{X})\mapsto(X(k),\mathcal{O}_{X(k)}=\alpha^{-1}\mathcal{O}_{X})$
  geringter Raum. \textbf{(1)} $\alpha^{-1}(U)=U\cap(X)\overset{1:1}{\longleftrightarrow}U$
  offene Teilmenge. \textbf{Behauptung}: Bild $(X(k),\mathcal{O}_{X(k)})$
  ist Raum mit Funktionen: Sei $V\subseteq U\subseteq X$ offen. \textbf{(2)}
  Das Diagramm
  \[
    \xymatrix{\mathcal{O}_{X(k)}(U\cap X(k))\ar[d]_{\res}\ar[r] & \Abb(U\cap X(\xi),\xi)\ar[d]^{\res}\\
      \mathcal{O}_{X(k)}(V\cap X(k))\ar@{^{(}->}[r] & \Abb(V\cap X(k),k)
    }
  \]

  kommutiert. Dazu $f\in\mathcal{O}_{X(k)}(U\cap X(k))\overset{(1)}{=}\mathcal{O}_{X}(U)$,
  wir assoziieren es der Abbildung 
  \[
    U\cap X(k)\longrightarrow k,\quad x\mapsto f(x):=\pi_{x}(f),
  \]

  mit
  \[
    \xymatrix{\pi_{x}:\mathcal{O}_{X}(U)\ar[r]\ar[d]^{\res} & \mathcal{O}_{X,x}\ar[r] & \kappa(x)=k\\
      \mathcal{O}_{X}(V)\ar[ur]
    }
  \]

  $\Longrightarrow(2)$. \textbf{(3)} $f,g$ mit derselben Funktion
  \[
    f\equiv g:U'\rightarrow k\overset{!}{\Longrightarrow}f=g
  \]

  Garbenaxiom $\Longrightarrow$ kann lokal überprüft werden: $U=\Spec A$
  und $\pi_{x}(f)=\pi_{x}(g)$ für alle $x\in\maxspec$ 
  \[
    \Longrightarrow f-g\in\bigcap_{\mathfrak{m}\in\maxspec(A)}\mathfrak{m}=\nil(A)=0,
  \]

  da $A$ lokal reduzierte $k$-Algebra. Da sich $X$ durch endlich
  viele affine Schemata der Form $\Spec A$, $A$ integer endlich erzeugte
  $k$-Algebra, überdecken lässt, ist der Raum mit Funktion $X(k)$
  eine Prävarietät. Die Konstruktion ist funktoriell, da jede Menge
  von Schemata von endlichem Typ über $K$ abgeschlossene Punkte auf
  abgeschlossene Punkt schickt nach Proposition 28.

  Um zu zeigen, dass beide Funktoren Quasi-Inverse zueinander sind,
  benutze den affinen (Varietät/Schemata) Fall, so wie die Garbenaxiome.
\end{proof}
\begin{rem*}[32]
  \mbox{}Es gilt:

  \begin{align*}
    \kappa(x) & =\kappa(X(k))\\
    \mathbb{A}_{k}^{n} & \longleftrightarrow\mathbb{A}(k)\\
    \mathbb{P}_{k}^{n} & \longleftrightarrow\mathbb{P}^{n}(k)
  \end{align*}
\end{rem*}

\section*{Unterschemata und Immersion (Einbettungen)}

\section{Offene und abgeschlossene Einbettung}
\begin{defn}[33]
  Ein Morphismus $j:Y\rightarrow X$ von Schemata heißt \textbf{offene
    Einbettung}, falls die unterliegende stetige Abbildung ein Homöomorphismus
  von $Y$ auf eine \emph{offene} Menge $U\subset X$ ist, sowie der
  Garbenhomomorphismus $\mathcal{O}_{X}\rightarrow j_{\ast}\mathcal{O}_{Y}$
  einen Isomorphismus $\mathcal{O}_{X|U}\cong j_{\ast}\mathcal{O}_{Y}$
  von Garben über $U$ induziert.
\end{defn}

,,$j$ induziert Isomorphismus zu $Y$ und offenen Unterschemata
$U$``
\begin{defn}[34]
  Sei $(X,\mathcal{O}_{X})$ ein geringter Raum. Eine Untergarbe $\mathcal{I}\subset\mathcal{O}_{X}$
  heißt \textbf{Idealgarbe}, falls $\Gamma(U,\mathcal{I})\unlhd\Gamma(U,\mathcal{O}_{X})$
  Ideal ist für alle $U\subseteq X$ offen. Es bezeichne $\mathcal{O}_{X}/\mathcal{I}$
  die Quotientengarbe assoziiert von der Prägarbe $U\mapsto\mathcal{O}_{X}(U)/\mathcal{I}(U)$.
  Dies ist eine Prägarbe mit $\mathcal{O}_{X}\rightarrow\mathcal{O}_{X}/\mathcal{I}$
  surjektiv, denn auf Halme:
  \[
    \underset{\underset{x\in U}{\longrightarrow}}{\lim}(\mathcal{O}_{X}(U)\twoheadrightarrow\mathcal{O}_{X}(U)/\mathcal{I}(U))=\mathcal{O}_{X,x}\twoheadrightarrow(\mathcal{O}_{X}/\mathcal{I})_{x}.
  \]
\end{defn}

\begin{defn}[35]
  Sei $X$ ein Schemata.
  \begin{enumerate}
  \item Ein \textbf{abgeschlossenes Unterschemata von $X$ }ist gegeben durch
    eine abgeschlossene Menge $Z\subseteq X$ ($i:Z\rightarrow X$ Inklusion),
    sowie eine Garbe $\mathcal{O}_{Z}$ auf $Z$, sodass $(Z,\mathcal{O}_{Z})$
    ein Schemata und $i_{\ast}\mathcal{O}_{Z}\cong\mathcal{O}_{X}/I$
    für eine Idealgarbe $I\subset\mathcal{O}_{X}$.
  \item Ein Morphismus $i:Z\rightarrow X$ von Schemata heißt \textbf{abgeschlossene
      Einbettung}, falls die unterliegende stetige Abbildung einen Homöomorphismus
    zwischen $Z$ und eine abgeschlossene Teilmenge von $X$ ist, und
    der Garbenhomomorphismus $i^{\flat}:\mathcal{O}_{X}\rightarrow i_{\ast}\mathcal{O}_{X}$
    surjektiv ist.
  \end{enumerate}
  Ist $Z\subseteq X$ ein abgeschlossenes Unterschemata wie in (1),
  so ist $(i,i^{\flat})$ eine abgeschlossene Einbettung. Umgekehrt
  bestimmt jede abgeschlossene Einbettung einen Isomorphismus von seiner
  Quelle auf ein eindeutiges abgeschlossenes Unterschemata seines Ziels.

  \textbf{Warnung:} Nicht für jede Idealgarbe $\mathcal{I}$ ist
  \[
    (Z=\supp\mathcal{O}_{X}/\mathcal{I},\mathcal{O}_{X}/\mathcal{I})
  \]

  ein Schema. Später: gilt gdw. $\mathcal{I}$ quasi-kompakt ist.
\end{defn}

\begin{thm}[36]
  Sei $X=\Spec A$. Dann ist die Abbildung
  \begin{align*}
    \text{\{Ideale }A\} & \overset{1:1}{\longleftrightarrow}\{\text{abg. Unterschemata von }X\}\\
    \mathfrak{a} & \longmapsto V(\mathfrak{a})\cong\Spec(A/\mathfrak{a})
  \end{align*}

  eine Bijektion. Insbesondere ist jedes abgeschlossene Unterschemata
  eines affinen Schematas affin.
\end{thm}

\begin{proof}
  Sei $Z$ ein abgeschlossenes Unterschemata, $i:Z\hookrightarrow X$
  Inklusion. Definition $\Longrightarrow\mathcal{O}_{X}\twoheadrightarrow i_{\ast}\mathcal{O}_{Z}$
  surjektiv. Sei:
  \[
    \mathcal{I}_{Z}:=\ker(\mathcal{O}_{X}(X)\rightarrow\Gamma(X,i_{\ast}\mathcal{O}_{Z})=\Gamma(Z,\mathcal{O}_{Z}))\unlhd A
  \]

  Ideal. Falls $Z$ von der Form $V(\mathfrak{a})$ ist (was zu zeigen
  ist!) gilt $\mathcal{I}_{Z}=\mathfrak{a}$. Daher reicht z.z. $Z=V(\mathcal{I}_{Z})$.
  \textbf{Dazu:
    \[
      \xymatrix{A\ar[r]^{\varphi}\ar@{->>}[dr] & \Gamma(Z,\mathcal{O}_{Z})\\
        & A/\mathcal{I}_{Z}\ar@{^{(}->}[u]
      }
    \]
  }

  faktorisiert per Definition. $\Longrightarrow$ Das Diagramm
  \[
    \xymatrix{Z\ar@{^{(}->}[r]^{i}\ar[rd] & X\\
      & \Spec(A/\mathcal{I}_{Z})\ar@{^{(}->}[u]
    }
  \]

  kommutiert. Es ist $\Mor(Z,\Spec A)=\Hom(A,\Gamma(Z,\mathcal{O}_{Z}))$,
  ohne Einschränkung: $\mathcal{I}_{Z}=0$ (sonst ersetze $A$ durch
  $A/\mathcal{I}_{Z}$). Zu zeigen: $Z\hookrightarrow X=V(\mathfrak{a})$
  ist ein Isomorphismus.

  Wir wissen: die unterliegende stetige Abbildung topologischer Räume
  ist injektiv und abgeschlossen. ($A\subset_{\text{abg.}}Z\subset X$
  $\Longrightarrow A\subset X$ abg.) Bleibt zu zeigen: surjektiv.

  Sei $U\subseteq Z$ offen mit $(U,\mathcal{O}_{X|U})$ affin. So gilt:
  \begin{align*}
    U\subset U\backslash D(\varphi(s)|_{U}) & =V_{U}(\varphi(s)|_{U})\\
                                            & =\varphi(s)|_{U}\in\mathcal{O}_{Z}(U)\text{ nilpotent}.
  \end{align*}

  Endliche Überdeckung von $Z$ durch affine Schemata $\Longrightarrow\varphi(s^{N})=0$.
  $\varphi$ injektiv $\Longrightarrow s^{N}=0$ bzw. $V(s)=X$. $Z$
  abgeschlossen in $X$ $\Longrightarrow i(Z)=X$.

  \textbf{Behauptung: }Der Homomorphismus von Garben $\mathcal{O}_{X}\rightarrow\mathcal{O}_{Z}$
  ist bijektiv. Reicht zu zeigen: injektiv (da surjektiv nach Voraussetzung).

  Sei $x\in X$ beliebig, $\mathcal{O}_{X,x}=A_{\mathfrak{p}_{x}}$.
  Sei $\frac{g}{1}\in\ker(\mathcal{O}_{X,x}\rightarrow\mathcal{O}_{Z,x}$).
  Überdecke
  \[
    Z=U\cup\bigcup_{i\in I}U_{i},\quad\#I<\infty
  \]

  mit:
  \begin{enumerate}
  \item $(U,\mathcal{O}_{Z\mid U})$, $(U_{i},\mathcal{O}_{Z\mid U_{i})}$
    affin für alle $i\in I$;
  \item $x\in U$, $\varphi(g)|_{U}=0$.
  \end{enumerate}
  Wähle $s\in A$ mit $x\in D(s)\subseteq U$. \textbf{Behauptung: }$\varphi(s^{N}g)=0$
  für $N>0$. Mit $\varphi$ injektiv folgt dann $s^{N}g=0$, und $\frac{g}{1}=0$
  in $\mathcal{O}_{X,x}$ da $s$ eine Einheit ist in $\mathcal{O}_{X,x}$.
  \begin{itemize}
  \item Nach (2) ist $\varphi(g)=0$, d.h. $\varphi(s\cdot g)|_{U}=\varphi(s)|_{U}\cdot\underbrace{\varphi(g)|_{U}}_{=0}=0$.
  \item $D_{U_{i}}(\varphi(s)|_{U_{i}})=D(s)\cap U_{i}\subseteq U\cap U_{i}$,
    also $\varphi(g)|_{D_{U_{i}}(\varphi(s)|_{U_{i}})}=0$, d.h. $\frac{\varphi(g)}{1}=0$
    in $\mathcal{O}_{Z}(U_{i})_{\varphi(s)|_{U_{i}}}$. $\Longleftrightarrow\varphi(s)|_{U_{i}}^{N_{i}}\varphi(g)=\varphi(s^{N_{i}}g)=0$
    (Die Indexmenge $I$ ist endlich). Setze $N:=\max_{i\in I}\{1,N_{i}\}$.
  \end{itemize}
\end{proof}

\section{Unterschemata und Einbettung}

Offene und abgeschlossene Unterschemata sind Spezialfälle von \emph{lokal
  abgeschlossene} Unterschemata.
\begin{defn}[37]
  \mbox{}
  \begin{enumerate}
  \item Sei $X$ ein Schemata. Ein \textbf{Unterschemata} von $X$ ist ein
    Schemata $(Y,\mathcal{O}_{Y})$, so dass $Y\subset X$ eine lokal
    abgeschlossene Teilmenge von $X$ ist, und $Y$ ein abgeschlossenes
    Unterschemata von dem offenen Unterschemata $U=X\backslash(\overline{Y}\backslash Y)\subseteq X$
    ist. Wir haben dann einen natürlichen Morphismus $Y\rightarrow X$
    von Schemata.
  \item Eine \textbf{Einbettung} $i:Y\rightarrow X$ ist ein Morphismus von
    Schemata, dessen unterlegende stetige Abbildung ein Homöomorphismus
    von $Y$ auf eine lokale abgeschlossene Teilmenge von $X$ ist, und
    sodass für alle $y\in Y$ : 
    \[
      i_{y}^{\#}:\mathcal{O}_{X,i(y)}\rightarrow\mathcal{O}_{Y,y}
    \]
    surjektiv ist.
  \end{enumerate}
\end{defn}

\begin{rem}[38]
  \mbox{}
  \begin{enumerate}
  \item Ist $Y$ ein Unterschemata von $X$, dann ist $Y\hookrightarrow X$
    eine Einbettung. Umgekehrt bestimmt jede Einbettung einen Isomorphismus
    seiner Quelle mit einem eindeutigen Unterschemata seines Ziels.
  \item Ist $Y$ ein Unterschemata von $X$, wessen unterliegende Teilmenge
    abgeschlossen in $X$ ist, dann ist $Y$ ein abgeschlossenes Unterschemata
    von $X$.
  \item Das Analogon von $(ii)$ für offene Unterschemata ist i.A. falsch.
  \item Jede Einbettung $i:Y\hookrightarrow X$ faktorisiert als:
    \[
      \xymatrix{Y\ar@{^{(}->}[r]^{i}\ar@{^{(}->}[rd] & X\\
        & U=X\backslash(\overline{i(Y)}\backslash i(Y))\ar@{^{(}->}[u]
      }
    \]
  \end{enumerate}
\end{rem}

\begin{defn}[39]
  Sei $X$ ein Schemata und $Z,Z'$ Unterschemata. Wir sagen $Z'$
  \textbf{majorisiert} $Z$, wenn die Inklusion $Z\hookrightarrow X$
  faktorisiert als:
  \[
    \xymatrix{Z\ar@{^{(}->}[r]\ar[rd] & X\\
      & Z'\ar@{^{(}->}[u]
    }
    .
  \]
\end{defn}

\begin{rem}[40]
  Sei \textbf{P} die Eigenschaft eines Schemata-Morphismus, eine affine
  Einbettung, bzw. abgeschlossene Einbettung, bzw. Einbettung zu sein.
  Dann:
  \begin{enumerate}
  \item Die Eigenschaft \textbf{P} ist lokal auf der Basis, d.h. für $f:Z\rightarrow X$
    Morphismus, $X=\bigcup_{i\in I}U_{i}$ offene Überdeckung hat $f$
    die Eigenschaft \textbf{P} $\Longleftrightarrow\forall i$ hat $f^{-1}(U_{i})\rightarrow U_{i}$
    die Eigenschaft \textbf{P}.
  \item Die Komposition zweier Morphismen mit Eigenschaft \textbf{P} hat Eigenschaft
    \textbf{P}.
  \end{enumerate}
\end{rem}

\begin{example}[41]
  \mbox{}
  \begin{enumerate}
  \item Sei $I\subseteq R[T_{0},\ldots,T_{n}]$ homogenes Ideal. Dann ist
    $V_{+}(I)\subseteq\mathbb{P}_{R}^{n}$ ein abgeschlossenes Unterschemata
    von $\mathbb{P}_{R}^{n}$. (Nach Bemerkung 40.1, denn $V_{+}(I)\cap U_{i}\subseteq U_{i}$
    abgeschlossen.)
  \item Alle Unterschemata eines $k$-Schematas $X$ von endlichem Typ sind
    selbst von endlichem Typ. 
  \end{enumerate}
\end{example}

\section{Projektive und quasi-projektive Schemata über einen Körper}
\begin{defn}[42]
Sei $k$ ein Körper.
\begin{enumerate}
\item Ein $k$-Schemata $X$ heißt \textbf{projektiv} wenn es ein $n\geq0$
und eine abgeschlossene Einbettung $X\hookrightarrow\mathbb{P}_{k}^{n}$
gibt.
\item Ein $k$-Schemata $X$ heißt \textbf{quasi-projektiv }wenn es ein
$n\geq0$ und eine Einbettung $X\hookrightarrow\mathbb{P}_{k}^{n}$
gibt.
\end{enumerate}
\end{defn}

\begin{example}[43]
\mbox{}
\begin{enumerate}
\item Für ein homogenes Ideal $I$ sind $V_{+}(I)$ projektive Schemata
(Beispiel 41).
\item Sei $X=\Spec A$ affines $k$-Schemata von endlichem Typ. Dann ist
$X$ quasi-projektiv: $A\cong k[T_{1},\ldots,T_{n}]\backslash\mathfrak{a}$,
\[
\xymatrix{X\ar@{^{(}->}[r]\ar[rd] & \mathbb{A}^{n}\ar[d]^{j}\\
 & \mathbb{P}^{n}
}
\]
\end{enumerate}
\end{example}

\section{Reduzierte Unterschemata}

\[
  \Spec K[X,Y]\supset\Spec(K[X,Y]/Y^{2})\supset\Spec(K[X,Y]/Y)
\]

\begin{question*}
  Gibt es ein ,,kleinstes`` Unterschemata?
\end{question*}
Setze $\mathcal{N}_{X}\subset\mathcal{O}_{X}$, Garbifizierung der
Prägarben:
\[
  U\mapsto\nil(\Gamma(U,\mathcal{O}_{X})),\quad U\subseteq X\text{ offen}
\]

Definiere $X_{\red}:=(X,\mathcal{O}_{X}/\mathcal{N}_{X})$.

\begin{prop}[44]
  \mbox{}
  \begin{enumerate}
  \item Der geringte Raum $X_{\red}=(X,\mathcal{O}_{X}/\mathcal{N})$ ist
    ein Schema, also ein abgeschlossenes Unterschema von $X$ mit demselben
    topologischen Raum wie $X$.
  \item Falls $X'\subset X$ ein weiteres solches Unterschema ist, dann gibt
    es eine abgeschlossene Einbettung $f:X_{\red}\rightarrow X'$, sodass
    das Diagramm
    \[
      \xymatrix{X_{\red}\ar@{^{(}->}[r]\ar[d] & X\\
        X'\ar@{^{(}->}[ur]
      }
    \]
    kommutiert.
  \item $X_{\red}$ ist reduziert und heißt das \textbf{unterliegend reduzierte
      Unterschema von $X$.}
  \item Falls $X=\Spec A$ affin, gilt $X_{\red}=\Spec(A/\nil(A))$.
  \end{enumerate}
\end{prop}

\begin{proof}
  Ohne Einschränkung sei $X=\Spec A$ $\Longrightarrow U\mapsto\nil(\Gamma(U,\mathcal{O}_{X}))$
  ist bereits eine Garbe, da 
  \[
    \nil(\mathcal{O}_{X}(D(f))=\nil(A_{f})=\nil(A)A_{f}
  \]

  für alle $f\in A$. $\Longrightarrow X_{\red}=\Spec(A/\nil A)$ offensichtlich
  reduziert.\textbf{ Universelle Eigenschaft:} zu zeigen $\mathcal{O}_{X}\rightarrow\mathcal{O}_{X}/\mathcal{N}$
  faktorisiert:
  \[
    \xymatrix{\mathcal{O}_{X}\ar[r]\ar[rd] & \mathcal{O}_{X}/\mathcal{N}\\
      & \mathcal{O}_{X'}\ar[u]
    }
    ,
  \]

  d.h. $\ker(\mathcal{O}_{X}\rightarrow\mathcal{O}_{X'})\subset\mathcal{N}$.
  Es reicht zu zeigen:
  \[
    \ker(\mathcal{O}_{X}(U)\rightarrow\mathcal{O}_{X'}(U))\subset\Gamma(U,\mathcal{N})
  \]

  für alle $U$ offen affin. Ohne Einschränkung $X=\Spec A$, $X'$
  abgeschlossenes Unterschema $\Longrightarrow$ affin: $X'=\Spec B$.

  Zu zeigen: $\ker(A\rightarrow B)\subset\nil A$. Da nach Voraussetzung
  $\Spec B\rightarrow\Spec A$ bijektiv ist, folgt:
  \[
    \ker(A\rightarrow B)\subset\bigcap_{\mathfrak{g}\in\Spec A}\mathfrak{g}=\nil A
  \]
\end{proof}
\begin{cor}[45]
  $(X_{\red},i_{X}:X_{red}\rightarrow X)$ ist durch die universelle
  Eigenschaft eindeutig bis auf eindeutige Isomorphie bestimmt.
\end{cor}

\begin{lem}[46]
  Jede Einbettung $i:Z\rightarrow X$ ist ein Monomorphismus in $\sch$.
\end{lem}

\begin{proof}
  \mbox{}
  \begin{itemize}
  \item Stetige Abbildung $Z\hookrightarrow X$ klar.
  \item Die Garbenabbildung $i_{Y}^{\#}$ ist surjektiv.
  \end{itemize}
\end{proof}
% 
\begin{proof}[Beweis von Korollar 45]
  Sei $X'_{\red}$ ein weiteres Schema mit universeller Eigenschaft
  \[
    \exists f:X_{\red}\rightarrow X'_{\red},\quad g:X'_{\red}\rightarrow X_{\red}
  \]

  so dass
  \[
    \xymatrix{X_{\red}\ar[r]^{f}\ar[rd]_{i_{X}} & X'_{\red}\ar[r]^{g}\ar[d]^{i_{X'}} & X_{\red}\ar[ld]^{i_{X}}\\
      & X
    }
    ,\quad i_{X}\circ(g\circ f)=i_{X}\circ\id_{X_{\red}}
  \]

  $i_{X}$ Monomorphismus $\Longrightarrow g\circ f=\id_{X_{\red}}=f\circ g$.
  Auf $(X_{\red},i_{X})=\{\id\}$ $\Longrightarrow$ Eindeutig.
\end{proof}
$(\cdot)_{\red}$ ist ein Funktor, wie die folgende Proposition zeigt:
\begin{prop}[47]
  Sei $f:X\rightarrow Y$ ein Schemata-Morphismus. Dann gibt es:
  \[
    \xymatrix{X_{\red}\ar[r]^{f_{\red}} & Y_{\red}\\
      X_{\red}\ar@{^{(}->}[r]^{i_{X}}\ar[d]_{f_{\red}} & X\ar[d]^{f}\\
      Y_{\red}\ar@{^{(}->}[r]_{i_{Y}} & Y
    }
    ,\quad\text{d.d.}
  \]

  Für weitere Morphismen $g:Y\rightarrow Z$ gilt
  \[
    (g\circ f)_{\red}=g_{\red}\circ f_{\red}.
  \]
\end{prop}

\begin{proof}
  $i_{Y}$ Monomorphismus $\Longrightarrow f_{\red}$ eindeutig. \textbf{Existenz:
  }Nach Verklebungs-Lemma $\Longrightarrow$ ohne Einschränkung $X=\Spec A$,
  $Y=\Spec B$, $f\hat{=}\ \varphi:B\rightarrow A$.

  $\Longrightarrow\varphi(\nil(B))\subset\nil(A)$

  $\Longrightarrow\varphi_{\red}:B/\nil(B)\rightarrow A/\nil(A)$

  $\Longrightarrow f_{\red}:\Spec(A/\nil(A))\rightarrow\Spec(B/\nil(B))$.
\end{proof}
\begin{prop}[48]
  Sei $X$ Schemata, $Z\subset X$ lokal abgeschlossene Teilmenge.
  Dann existiert ein eindeutig bestimmtes reduziertes Unterschema mit
  topologischem Raum $Z$.
\end{prop}

\begin{proof}
  Eindeutigkeit: Korollar 45. Existenz: Verklebungslemma $\Longrightarrow$
  ohne Einschränkung $X=\Spec A$ affin und $Z\subset X$ abgeschlossen
  (sonst Überdeckung von $X$ zu $Z\subset_{\text{abg.}}U\subset_{\text{abg.}}X$)
  $\Longrightarrow\exists\mathfrak{a}\subset A$ so dass $Z=V(\mathfrak{A})$
  $\Longrightarrow Z'=\Spec(A/\mathfrak{a})$ ist abgeschlossenes Unterschema
  von $X$ mit topologischem Raum $Z$. Satz 44 $\Longrightarrow\exists Z'_{\red}\subset Z'\subset X$.
\end{proof}
Damit besitzt für ein lokal abgeschlossene Teilmenge die geordnete
Menge (bzgl. Inklusion) von Unterschema, denen topologischen Raum
$Z$ umfassen, ein eindeutiges minimales Element $Z_{\red}$, das
\textbf{reduzierte Unterschema} mit unterliegendem Raum $Z$.


\chapter{Faserprodukte}
\label{chap:faserprodukte}

\section{Der Punkte-Funktor}

Kontravarianter Funktor, $\forall X\in\sch$,
\begin{align*}
  h_{X}:(\sch)^{\op} & \longrightarrow\set\\
  S & \longmapsto h_{X}(S):=\Hom_{\sch}(S,X)\\
  (f:T\rightarrow S) & \longmapsto(\Hom(S,X)\overset{f^{\ast}}{\rightarrow}\Hom(T,X),\ g\mapsto g\circ f)
\end{align*}

$f^{\ast}=h_{X}(S)$ heißen $S$-wertige Punkte von $X$. \textbf{Notation:
}$X(S)$, $X(R)$, falls $S=\Spec R$.

\textbf{Relative Version:} $X\in\schs0$, $S_{0}$ fixes Schemata.
\begin{align*}
  \schs0 & \longrightarrow\set\\
  S & \longmapsto\Hom_{S_{0}}(S,X)
\end{align*}

\textbf{Notation: $X_{S_{0}}(S)$, $X_{R_{0}}(S)$, $X_{S_{0}}(R)$,
  $X_{R_{0}}(R)$}
\begin{example}[1]
  Sei $k$ algebraisch abgeschlossen, $X/k$ von endlichem Typ, $x\in X_{k}(k)$.
  Dann ist
  \[
    \im(\Spec k\overset{x}{\longrightarrow}X)\in X
  \]
  abgeschlossener Punkt. $x\mapsto\im(x)$ liefert Bijektion, $X_{k}(k)\rightarrow|X|$
  Menge der abgeschlossenen Punkte.
\end{example}

\begin{example}[2]
  Sei $X=\mathbb{A}^{n}=\Spec(\mathbb{Z}[T_{1},\ldots T_{n}])$. Dann:
  \begin{align*}
    \mathbb{A}^{n}(S) & =\Hom_{\sch}(S,\mathbb{A}^{n})=\Hom_{\ring}(\mathbb{Z}[T_{1},\ldots,T_{n}],\mathcal{O}_{S}(S))\\
                      & =\Gamma(S,\mathcal{O}_{S})^{n}
  \end{align*}
\end{example}

\begin{example}[3]
  Sei $X=\Spec(R[T_{1},\ldots,T_{n})/(f_{1},\ldots,f_{m}))$, $S$ ein
  $R$-Schema. Dann:
  \begin{align*}
    X_{R}(S) & =\Hom_{R\text{-Alg}}(R[I]/(f),\mathcal{O}_{S}(S))\\
             & =\{s\in\mathcal{O}_{S}(S)^{n}\mid f_{1}(s)=\cdots=f_{m}(s)=0\}
  \end{align*}
\end{example}

\begin{example}[4]
  Sei $X=\Spec\mathbb{Z}[T,T^{-1}]$. Dann:
  \[
    X(S)=\Hom(\mathbb{Z}[T,T^{-1}],\mathcal{O}_{S}(S))=\Gamma(S,\mathcal{O}_{S})^{\times}.
  \]
  Hier sogar $h_{X}:\sch\rightarrow\grp$. $X$ ist eine abelsche Gruppe.
\end{example}

\section{Yoneda-Lemma}

\textbf{Ziel:} $h_{X}$ beschreibt $X$ eindeutig.

Erinnerung: $\mathcal{F},\mathcal{G}:\mathcal{A}\rightarrow\mathcal{B}$
Funktoren, natürliche Transformation $f\in\Hom(\mathcal{F},\mathcal{G})$:
\[
  f=\{f(X):\mathcal{F}(X):\rightarrow\mathcal{G}(X))_{X\in\mathcal{A}}.
\]

Wir erhalten Kategorien: $\func(\mathcal{A},\mathcal{B})$, \textbf{hier:
}$\mathcal{C}=\schs0$, $\hat{C}=\func(\mathcal{C}^{\op},\set)$.
Wir erhalten einen Funktor
\begin{align*}
  \mathcal{C} & \longrightarrow\hat{\mathcal{C}},\\
  X & \longmapsto h_{X},\\
  f & \longmapsto\text{Pullback }f^{\ast}.
\end{align*}

\begin{prop}[5]
  Sei $X\in\mathcal{C}$, $\mathcal{F}\in\hat{\mathcal{C}}$. Dann
  ist die Abbildung
  \begin{align*}
    \Hom_{\hat{C}}(h_{X},\mathcal{F}) & \longrightarrow\mathcal{F}(X)\\
    \alpha & \longmapsto\alpha(X)(\id_{X})\in\Hom(h_{X}(X),\mathcal{F}(X))
  \end{align*}

  bijektiv und funktoriell.
\end{prop}

Insbesondere ist der obige Funktor $\mathcal{C}\rightarrow\hat{\mathcal{C}}$
volltreu (wähle $\mathcal{F}=h_{Y}$!)
\begin{proof}
  Umkehrabbildung:
  \begin{align*}
    \mathcal{F}(X) & \longrightarrow\Hom_{\hat{\mathcal{C}}}(h_{X},\mathcal{F})\\
    \xi & \longmapsto\alpha_{\xi}=(\alpha_{\xi}(Y))_{Y\in\schsn}
  \end{align*}
  \begin{align*}
    \text{mit}\quad\alpha_{\xi}(Y):\Hom(Y,X)=h_{X}(Y) & \longrightarrow\mathcal{F}(Y)\\
    f & \longmapsto\mathcal{F}(f)(\xi)\in\Hom(\mathcal{F}(X),\mathcal{F}(Y))
  \end{align*}
\end{proof}

\section{Faserprodukte in beliebigen Kategorien}

Sei $\mathcal{C}$ eine Kategorie, $S\in\obj(\mathcal{C})$, $f:X\rightarrow S$,
$g:Y\rightarrow S$ Morphismen.
\begin{defn}[6]
  Ein Tupel $(Z,p,q)$ mit $Z\in\obj(\mathcal{C})$ und Morphismen
  $p:Z\rightarrow X$, $q:Z\rightarrow Y$, $f\circ p=g\circ q$, hei{\small{}ßt
  }\textbf{\small{}Faserprodukt}{\small{} von $X$ und $Y$ über $S$
    (bzw. von $f,g$), falls für jedes $T\in\obj(\mathcal{C})$ und Paare
    $(u:T\rightarrow X,v:T\rightarrow Y)$ von Morphismen mit $f\circ u=g\circ v$
    genau ein Morphismus $w:T\rightarrow Z$ existiert mit $p\circ w=u$,
    $g\circ w=v$.}{\small\par}
\end{defn}

\textbf{Notation:} $X\times_{S}Y$ oder $X\times_{f,S,g}Y:=Z$, $(u,v)_{S}:=w$.
\[
  \xymatrix{T\ar@/^{1pc}/[rrd]^{u}\ar@/_{1pc}/[ddr]_{v}\ar[dr]|-{\exists_{1}}\\
    & X\times_{S}Y\ar[r]^{p}\ar[d]^{q} & X\ar[d]^{f}\\
    & Y\ar[r]_{g} & S
  }
\]

Ist $S$ ein finales Objekt in $\mathcal{C}$, so ist $X\times_{S}Y=X\times Y$
das kategorielle Produkt.
\begin{example}[7]
  \mbox{}
  \begin{enumerate}
  \item $\mathcal{C}=\set$. $X\times_{S}Y=\{(x,y)\in X\times Y\mid f(x)=g(x)(\}$.
  \item $\mathcal{C}=\Top$, $f:X\rightarrow S$, $g:Y\rightarrow S$ stetige
    Abbildungen. Versehe $\{(x,y)\mid f(x)=g(y)\}\subset X\times Y$ mit
    der Topologie induziert von der Produkttopologie auf $X\times Y$.
    Dies ist ein Faserprodukt in $\Top$.
  \end{enumerate}
\end{example}

\textbf{Ab jetzt:} Alle Faserprodukte mögen in $\mathcal{C}$ existieren.

Sei Morphismus $h:T\rightarrow S$ in $\mathcal{C}$, ein \textbf{$S$-Objekt}.
(kurz $T$, \textbf{$h$ Strukturmorphismus} von $T$.)

Für $S$-Objekte $h:T\rightarrow S$ und $f:X\rightarrow S$ schreibe
$\Hom_{S}(T,X):=X_{S}(T)$ für Morphismen $w:T\rightarrow X$ mit
$f\circ w=h$, die \textbf{$S$-Morphismen}. Nenne $X_{S}(T)$ die
\textbf{Menge der $T$-wertigen Punkte von $X$ (über $S$)}. Dies
definiert eine Kategorie $\mathcal{C}/S$ mit finalem Objekt $\id_{S}$.

\textbf{Faserprodukt }$X\times_{S}Y$ ist Produkt von $S$-Objekten
$f$ und $g$ in $\mathcal{C}/S$.
\begin{align*}
  (X\times_{S}Y)(T) & =X_{S}(T)\times Y_{S}(T)\\
  \text{(UAE)\ }\Hom_{S}(T,X\times_{S}Y) & \overset{\sim}{\longrightarrow}\Hom_{S}(T,X)\times\Hom_{S}(T,Y)\\
  w & \longmapsto(p\circ w,q\circ w)
\end{align*}

für alle $h:T\rightarrow S$.

\subsection*{Funktorialität}

Seien $X,Y,X',Y'\in\mathcal{C}/S$, $u:X\rightarrow X'$, $v:Y\rightarrow Y'$
$S$-Morphismus. $\Longrightarrow\exists_{1}$ Morphismus $u\times_{S}v$
(oder nur $u\times v$): $X\times_{S}Y\rightarrow X'\times_{S}Y'$.
d.d.
\[
  \xymatrix{X\times_{S}Y\ar[rd]|-{u\times v}\ar[r]^{p}\ar[d]_{q} & X\ar[rd]^{u}\\
    Y\ar[rd]_{v} & X'\times Y'\ar[r]\ar[d] & X'\ar[d]\\
    & Y'\ar[r] & S
  }
  \qquad\text{kommutiert.}
\]

Setze $u\times v:=(u\circ p,v\circ q)_{S}$ (Universelle Eigenschaft
von $X'\times_{S}Y'$!)\medskip{}

\textbf{Yoneda-Lemma:}
\[
  f:X\rightarrow Y\in\Mor(\mathcal{C}/S)\leftrightarrow(f_{S}(T):X_{S}(T)\rightarrow Y_{S}(T))_{T\in\mathcal{C}/S}\text{ funktoriell in }T
\]

\begin{prop}[8, Eigenschaften des Faserprodukts]
  Sei $X,Y,Z\in\mathcal{C}/S$. Dann gibt es \textbf{kanonische Isomorphismen}
  (funktoriell in $X,Y,Z$),
  \begin{enumerate}
  \item $X\times_{S}S\xrightarrow{\sim}X$
  \item $X\times_{S}Y\xrightarrow{\sim}Y\times_{S}X$
  \item $(X\times_{S}Y)\times_{S}Z\xrightarrow{\sim}X\times_{S}(Y\times_{S}Z)$
  \end{enumerate}
  auf $T$-wertigen Punkten, für alle $h:T\rightarrow S$ $S$-Objekt,
  gegeben durch:
  \begin{align*}
    X_{S}(T)\times S_{S}(T) & \xrightarrow{\sim}X_{S}(T), & (x,h)\mapsto x,\\
    X_{S}(T)\times Y_{S}(T) & \xrightarrow{\sim}Y_{S}(T)\times X_{S}(T), & (x,y)\mapsto(y,x),\\
    (X_{S}(T)\times Y_{S}(T))\times Z_{S}(T) & \xrightarrow{\sim}X_{S}(T)\times(Y_{S}(T)\times Z_{S}(T)), & ((x,y),z)\mapsto(x,(y,z)).
  \end{align*}
\end{prop}

Sei $Z\in\mathcal{C}/S$. Ein kommutatives Diagramm in $\mathcal{C}$:
\[
  \xymatrix{Z\ar[r]^{u}\ar[d]_{v}\ar@{}[dr]|{\square} & X\ar[d]^{f}\\
    Y\ar[r]_{g} & S
  }
\]

heißt \textbf{kartesisch} falls $(u,v)_{S}:Z\rightarrow X\times_{S}Y$
ein Isomorphismus ist (und dabei automatisch in $\mathcal{C}/S$).
\[
  \xymatrix{Z\ar[d]_{u}\ar@{-->}[rd]|-{(u,v)_{S}}\ar[r]^{v} & Y\\
    X & X\times_{S}Y\ar[l]^{p}\ar[u]_{q}
  }
\]

\begin{rem}
Nach dem Yoneda-Lemma ist dies äquivalent dazu, dass das Diagramm:
\[
  \xymatrix{\Hom_{\mathcal{C}}(T,Z)\ar[r]^{u(T)}\ar[d]_{v(T)} & \Hom_{\mathcal{C}}(T,X)\ar[d]^{f(T)}\\
    \Hom_{\mathcal{C}}(T,Y)\ar[r]_{g(T)} & \Hom_{\mathcal{C}}(T,S)
  }
  \qquad(*)
\]

kartesisch ist für alle $T\in\mathcal{C}$ (Beispiel 4.7). Dies ist
äquivalent zu: %
\begin{comment}
  Pfeile klappen hier nicht besonders gut, evt. Isomorphismus $\Hom_{\mathcal{C}}(T,Z)\rightarrow\cdots$
  benennen.
\end{comment}
\[
  \xymatrix{\Hom_{\mathcal{C}}(T,Z)\overset{!}{\cong}\ar[r]\ar[d]_{s}^{(u,v)_{S}(T)} & \Hom_{\mathcal{C}}(T,X)\times_{\Hom_{\mathcal{C}}(T,S)}\Hom_{\mathcal{C}}(T,Y)\\
    \Hom_{\mathcal{C}}(T,X\times_{S}Y)\ar[ur]_{f\mapsto(p\circ h,q\circ h)}
  }
\]

mit $\Hom_{\mathcal{C}}(T,X)\times_{\Hom_{\mathcal{C}}(T,S)}\Hom_{\mathcal{C}}(T,Y):=\{(h_{1},h_{2})\mid f\circ f_{1}=g\circ h_{2}\}$.
\begin{comment}
  Hier ist noch ein weiteres Diagramm, mit sich kreuzenden Pfeilen.
\end{comment}
\end{rem}

\begin{prop}[10]
  Sei das Diagramm
  \[
    \xymatrix{X''\ar[r]^{g'}\ar[d] & X'\ar[r]^{g}\ar[d]\ar@{}[dr]|{\square} & X\ar[d]\\
      S''\ar[r]_{f'} & S'\ar[r]_{f} & S
    }
  \]

  kommutativ, mit rechts ein kartesisches Diagramm. Dann gilt:
  \[
    \xymatrix{X''\ar[r]\ar[d]\ar@{}[dr]|{\square} & X'\ar[d]\\
      S''\ar[r] & S'
    }
    \quad\Longleftrightarrow\quad\xymatrix{X''\ar[r]\ar[d]\ar@{}[dr]|{\square} & X\ar[d]\\
      S''\ar[r] & S
    }
  \]
\end{prop}

\begin{proof}
  Zeige in der Kategorie $\mathcal{C}=\set$, und wende das Yoneda-Lemma,
  $(*)$ an.
\end{proof}

\section{Faserprodukte von Schemata}
\begin{prop}[11]
\end{prop}

\begin{thm}[12]
\end{thm}

\begin{cor}[13]
\end{cor}

Sei $X,X'\in\schs$, $f:X'\rightarrow X$ Morphismus in $\schs$,
$g:=f\times_{S}\id_{Y}$.
\[
  \xymatrix{Z'=X'\times_{S}Y\ar[r]^{g}\ar[d]_{p'} & Z=X\times_{S}Y\ar[r]^{q}\ar[d]_{p'} & Y\ar[d]\\
    X'\ar[r]^{f} & X\ar[r] & S
  }
\]

kommutiert. Da $q\circ g=q'$ Projektion auf $Y$ ist, ist das große
und damit auch beide Diagramme kartesisch. (Proposition 10)
\begin{prop}[14]
  $f$ induziert einen Homömorphismus von $X'$ auf $f(X')$ und:
  \begin{enumerate}
  \item $f_{x'}^{\#}:\mathcal{O}_{X,f(x')}\rightarrow\mathcal{O}_{X',x'}$
    sei surjektiv $\forall x'\in X'$ und es existiert eine offene affine
    Umgebung $U'$ von $f(x')$, sodass $f^{-1}(U')$ quasi-kompakt ist,
    oder
  \item $f_{x'}^{\#}$ ist bijektiv $\forall x'\in X'$.
  \end{enumerate}
  Dann gilt:
  \begin{enumerate}
  \item $g$ ist ein Homöomorphismus von $Z'$ auf $g(Z')=p^{-1}(f(X'))$.
  \item $\forall z'\in Z'$ haben wir das induzierte Diagramm für lokale Ringe:
    \[
      \xymatrix{\mathcal{O}_{Z',z'}\ar[d] & \mathcal{O}_{Z,g(z')}\ar[l]^{g_{z'}^{\#}}\ar[d]^{p_{g(z)}^{\#}}\\
        \mathcal{O}_{X',p'(z')} & \mathcal{O}_{X,p(g(z'))}\ar[l]^{f_{p'(z')}}
      }
    \]
  \end{enumerate}
  \begin{itemize}
  \item $g_{z'}^{\#}$ ist surjektiv;
  \item $\ker(g_{z'}^{\#})$ ist von $p_{g(z')}^{\#}(\ker f_{p'(z')}^{\#})$
    erzeugt.
  \end{itemize}
\end{prop}

\begin{proof}
  (1), (2) lassen sich lokal bzgl. $S,Y,X$ verifizieren. Ohne Einschränkung
  sei $S=\Spec R$, $X=\Spec A$, $Y=\Spec B$ affin, $X'$ quasi-kompakt.
  \[
    f\leftrightarrow\xymatrix{A\ar[r]^{\varphi}\ar@{->>}[rd]_{\varphi_{1}} & \Gamma(X,\mathcal{O}_{X})\\
      & A/\ker\varphi\ar@{^{(}->}[u]^{\varphi_{2}}
    }
    \quad R\text{-Algebren}
  \]

  $f$ faktorisiert sich als
  \[
    \xymatrix{X'\ar[r]^{f_{1}} & \Spec(A/\ker\varphi)\ar[r]^{f_{2}} & \Spec(A)=X}
    .
  \]

  Dann ist $f_{2}$ eine abgeschlossene Immersion, also surjektiv auf
  Halmen $(f_{p})_{2}$, und ein Homeomorphismus auf einer abgeschlossenen
  Teilmenge in $X$. In Situation $I$ erfüllt daher mit $f$ auch $f_{1}$
  Voraussetzung (1). Daher reicht es die folgenden 2 Fälle zu beweisen.
  \begin{enumerate}
  \item $f$ ist eine abgeschlossene Immersion ($\hat{=}$ $f_{2}$ Voraussetzung
    (1))
  \item $f_{x'}^{\#}$ ist bijektiv für alle $x'\in X'$ ($\hat{=}$ $f_{1}$
    Voraussetzung (1) + Voraussetzung (2)).
  \end{enumerate}
  \begin{lem*}
    Sei:
    \[
      \xymatrix{A\ar@{^{(}->}[r] & \Gamma(X,\mathcal{O}_{X})\\
        X'\ar[r]^{f} & \Spec(A)
      }
      ,\quad X'\text{ quasi-kompakt}
    \]

    Dann ist $f_{x'}^{\#}$ injektiv für alle $x\in X'$.
  \end{lem*}
  \textbf{Vorüberlegung. }Sei $Z$ ein Schemata und $t\in\Gamma(Z,\mathcal{O}_{Z})$,
  $Z_{T}:=\{z\mid t(z)\neq0\}\subset Z$ offen. Die Einschränkung
  \[
    \xymatrix{\Gamma(Z,\mathcal{O}_{Z})\ar[r]\ar[d] & \Gamma(Z_{t},\mathcal{O}_{Z})\\
      \Gamma(Z,\mathcal{O}_{Z})_{t}\ar@{^{(}-->}[ur]_{\rho_{t}}
    }
  \]

  definiert einen Homomorphismus $\rho_{t}$. Dieser ist injektiv, falls
  $Z$ quasi-kompakt, \textbf{denn} $Z=\bigcup U_{i}$ ist endliche
  offene affine Überdeckung. Sei $C_{i}=\mathcal{O}_{Z}(U_{i})$, $t_{i}=t|_{U_{i}}$.
  $\Longrightarrow(\prod_{i}C_{i})_{t}=\prod_{i}(C_{i})_{t_{i}}$ da
  $i$ endlich. Wir erhalten das kommutative Diagramm:
  \[
    \xymatrix{\mathcal{O}_{Z}(Z)\ar[r]\ar@{^{(}-}[d]_{\text{Garbe}} & \Gamma(Z,\mathcal{cO}_{Z})_{t}\ar[r]^{\rho_{t}}\ar@{^{(}-}[d] & \Gamma(Z_{t},\mathcal{O}_{Z})\ar@{^{(}-}[d]^{\text{Garbe}}\\
      \prod_{i}C_{i}\ar[r] & \prod_{i}(C_{i})_{t_{i}}\ar[r]_{\cong} & \prod_{i}\Gamma(D(t_{i}),\mathcal{O}_{U_{i}})
    }
  \]
  \begin{proof}[Beweis (Lemma)]
    Sei $\mathfrak{p}\subset A\cong f(x')$. Für alle $s\in A\backslash\mathfrak{p}$
    sei
    \[
      \varphi_{s}:A_{s}\longrightarrow\Gamma(X',\mathcal{O}_{X'})_{\varphi(s)}
    \]

    der injektive Homomorphismus aus $\varphi$ durch Lokalisierung in
    $s$, und sei $\psi_{s}$ die injektive Komposition
    \[
      \xymatrix{\psi_{s}:A_{s}\ar[r]^{\varphi_{s}} & \Gamma(X',\mathcal{O}_{X'})_{\varphi(s)}\ar[r]^{\rho_{\varphi(s)}} & \Gamma(X'_{\varphi(s)},\mathcal{O}_{X'}).}
    \]

    Dann ist $X'_{\varphi(s)}=f^{-1}(D(s))$. Da für $s\in A\backslash\mathfrak{p}$
    die $D(s)$ eine offene Umgebungsbasis von $f(x)'$, und da $f$ ein
    Homeomorphismus auf sein Bild ist, bilden die $X'_{\varphi(s)}$ eine
    offene Umgebungsbasis von $x'$. $\Longrightarrow\underset{\underset{s}{\longrightarrow}}{\lim}\ \Gamma(X'_{\varphi(s)},\mathcal{O}_{X'})=\mathcal{O}_{X',x'}$
    und $\underset{\underset{s}{\longrightarrow}}{\lim}\psi_{s}=f_{x'}^{\#}$
    $\Longrightarrow f_{x'}^{\#}$ injektiv.
  \end{proof}
  Zu 1.) Sei $\xymatrix{X'=\Spec(A/\mathfrak{a})\ar[r]^{f} & \Spec(A)=X}
  ,$ $\mathfrak{a}\subset A$ Ideal. Proposition 11 $\Longrightarrow Z,Z'$
  affin, und $g$ entspricht $R$-Algebren
  \[
    \xymatrix{A\otimes_{R}B\ar@{->>}[r]^{``g``} & A/\mathfrak{a}\otimes_{R}B\\
      A\ar[u]^{``p``}\ar[r]_{``f``} & A/\mathfrak{a}\ar[u]_{p'}
    }
  \]

  und $``p``(\ker``f``)\subset A\otimes_{R}B=\ker``g``$ $\Longrightarrow g$
  ist Homöomorphismus auf $g(Z')=p^{-1}(f(x'))$, $x'\in X'$. $\checkmark$\medskip{}

  Zu 2.) Es ist $f^{\#}:f^{-1}\mathcal{O}_{X}\rightarrow\mathcal{O}_{X'}$
  ein Isomorphismus bzgl. $(X',\mathcal{O}_{X'})\cong(f(x'),\mathcal{O}_{X}|_{f(x')})$
  Isomorphismus lokal geringter Räume. Leicht zu verifizieren: $(p^{-1}(f(x')),\mathcal{O}_{Z}|_{p^{-1}(f(x'))})$
  ist ein Faserprodukt von $X'$ mit $Z$ über $X$ in der Kategorie
  lokal geringter Räume, also erst recht in $\sch$. (vgl. Zusatz in
  Theorem (Existenz $X\times_{S}Y$)). $\Longrightarrow g$ Isomorphismus
  \[
    \xymatrix{(Z',\mathcal{O}_{Z})\ar[r]^{\cong} & (p^{-1}(f(x')),\mathcal{O}_{Z}|_{p^{-1}(f(x'))})}
    .
  \]
\end{proof}
\begin{example*}
  Proposition 14 gilt in folgenden Situationen.
  \begin{enumerate}
  \item $f$ ist eine Immersion von Schemata.
  \item $f$ ist der kanonische Morphismus $\Spec\mathcal{O}_{X,x}\rightarrow X$
    für ein $x\in X$, vgl. (5.4), (2.11).
  \item $f$ ist der kanonische Morphismus $\Spec\kappa(x)\rightarrow X$
    für ein $x\in X$.
  \item Komposition von Morphismen, die Proposition 14 erfüllen (und Proposition
    10).
  \end{enumerate}
\end{example*}
\section{Beispiele}

\paragraph{Produkte affiner Räume}

Sei $R$ ein Ring, und $\mathbb{A}_{R}^{n}=\Spec(R[T_{1},\ldots,T_{n}])$
der affine Raum über $R$. Für $n,m\geq0$ haben wir 
\[
  R[T_{1},\ldots,T_{n}]\otimes_{R}R[T_{n+1},\ldots,T_{n+m}]\cong R[T_{1},\ldots,T_{n+m}]
\]

und deshalb nach Proposition 11
\[
  \mathbb{A}_{R}^{n}\times_{R}\mathbb{A}_{R}^{m}\cong\mathbb{A}_{R}^{n+m}.
\]


\paragraph{Produkte von Prävarietäten}

Sei $k$ ein algebraisch abgeschlossener Körper, und $X$ ein $k$-Schema
endlichen Typs. Nach 3.14 ist $X_{k}(k)=X_{0}$ (abgeschlossene Punkte
von $X$).
\begin{align*}
  x:\Spec k & \longrightarrow X\longrightarrow\text{Bild}\\
  \text{\{Integ. Sch. v.e.T./}k\} & \longleftrightarrow\text{\{Präv./}k\}\\
  X & \longmapsto\{X_{0},\mathcal{O}_{X}|_{X_{0}}\}
\end{align*}

\begin{lem}[15]
  Sei $k$ ein Körper und seien $X,Y$ integre $k$-Schemata. Dann
  ist $X\times_{k}Y$ ein integres $k$-Schemata.
\end{lem}

Beweis: später. Falls $X,Y$ integral von endlichem Typ über $k$
sind, dann ist auch $X\times_{k}Y$ integral von endlichem Typ über
$k$. Denn: $X=\bigcup_{\text{endl.}}X_{i}$, $Y=\bigcup_{\text{endl.}}Y_{j}$
$\Longrightarrow X\times_{k}Y=\bigcup_{i,j}X_{i}\times_{k}Y_{i}$.
$\Longrightarrow X=\Spec A$, $Y=\Spec B$ mit $A,B$ endlich erzeugte
$k$-Algebras. $\Longrightarrow X\times_{k}Y=\Spec A\otimes_{k}B$
endlich erzeugte $k$-Algebra.\medskip{}

Seien $X_{0}$ und $Y_{0}$ die Prävarietäten zu $X$ bzw. $Y$, und
$Z_{0}$ die Prävarietät zu $X\times_{k}Y$. Dann gilt nach der universellen
Eigenschaft des Faserprodukts:
\[
  Z_{0}=(X\times_{k}Y)_{k}(k)=X_{k}(k)\times Y_{k}(k)=X_{0}\times Y_{0},
\]

d.h. das Faserprodukt von 2 Prävarietäten $X_{0},Y_{0}$ ist wieder
eine Prävarietät $Z_{0}$ (als volle Unterkategorie von $\schk$)
mit $Z_{0}=X_{0}\times Y_{0}$ (als Mengen). Die Projektionen $Z_{0}\rightarrow X_{0}$
und $Z_{0}\rightarrow Y_{0}$ sind stetig, aber im Allgemeinen ist
die Topologie auf $Z_{0}$ \textbf{feiner }als die Produkttopologie
von $X_{0}$ und $Y_{0}$.

\section{Basiswechsel}

Sei $\mathcal{C}$ eine beliebige Kategorie mit Faserprodukten (z.B.
$\sch$), $u:S'\rightarrow S$ ein Morphismus in $\mathcal{C}$, $X\rightarrow S$
ein $S$-Objekt. $\Longrightarrow q:X\times_{S}S'\rightarrow S'$
ist ein $S'$-Objekt. Bezeichne $u^{\ast}(X)=:q$ oder $X_{(s')}$
\textbf{Urbild} oder \textbf{Basiswechsel} von $X$ bzgl. $u$.\medskip{}

Sei $f:X\rightarrow Y$ Morphismus von $S$-Objekten. $\Longrightarrow f\times_{S}\id_{S'}:X\times_{S}S'\rightarrow Y\times_{S}S'$
ist ein Morphismus von $S'$-Objekten. Bezeichne $f\times_{S}\id_{S'}=:u^{*}(f)=:f_{(s')}$
der \textbf{Basiswechsel von $f$ bzgl. $u$}. Wir erhalten einen
kontravarianten Funktor
\[
  u^{\ast}:\mathcal{C}/S\longrightarrow\mathcal{C}/S'
\]

der Kategorie von $S$-Objekten in $\mathcal{C}$ zu der Kategorie
der $S'$-Objekten in $\mathcal{C}$. Nenne $u^{\ast}$ den \textbf{Basiswechsel
  bzgl. $u$}.

\paragraph{Transitivität des Basiswechsels}

Sei $u':S''\rightarrow S'$ ein weiterer Morphismus in $\mathcal{C}$.
Nach Proposition 10 ist $(u\circ u')^{\ast}\cong u'^{\ast}\circ u^{\ast}$
ein Isomorphismus von Funktoren. Sei
\[
  \xymatrix{T\ar[r]^{h}\ar[rd] & S'\ar[d]^{u}\\
    & S
  }
  \in\mathcal{C}/S'.
\]

Wir können $T$ als $S$-Objekt auffassen durch $u\circ h$. Sei $p:X_{(S')}\rightarrow X$
die erste Projektion. Dann erhalten wir zueinander inverse Bijektionen,
funktoriell in $T$ und $X$:
\begin{align*}
  t & \longmapsto p\circ\\
  \hom_{S'}(T,X_{(S')}) & \longleftrightarrow\hom_{S}(T,X)\\
  (t,h)_{S'} & \longmapsfrom t
\end{align*}

\begin{defn}[16]
  Sei $\mathbb{P}$ eine Eigenschaft von Morphismen in $\mathcal{C}$,
  sodass $\id_{X}$ $\mathbb{P}$ erfüllt für alle $X\in\mathcal{C}$.
  \begin{enumerate}
  \item $\mathbb{P}$ heißt \textbf{stabil}
    \begin{enumerate}
    \item \textbf{unter Komposition}, wenn mit $f:X\rightarrow Y$ und $g:Y\rightarrow Z$
      auch $g\circ f$ $\mathbb{P}$ erfüllt.
    \item \textbf{unter Basiswechsel}, wenn mit $f:X\rightarrow S$ auch $f_{(S')}:X_{(S')}\rightarrow S'$
      für alle Morphismen $S'\rightarrow S$, $\mathbb{P}$ erfüllt.
    \end{enumerate}
  \item Wir sagen, dass $f:X\rightarrow S$ $\mathbb{P}$ \textbf{universell}
    erfüllt, falls $f_{(S')}$ $\mathbb{P}$ erfüllt für alle $S'\rightarrow S$.
  \end{enumerate}
\end{defn}

\begin{rem}[17]
  Sei $\mathbb{P}$ stabil unter Komposition. Dann sind äquivalent:
  \begin{enumerate}
  \item $\forall S\in\mathcal{C}$, $\forall S$-Morphismen $f:X'\rightarrow X$,
    $g:Y'\rightarrow Y$, die $\mathbb{P}$ erfüllen, erfüllt auch $f\times_{S}g$
    $\mathbb{P}$.
  \item $\mathbb{P}$ ist stabil unter Basiswechsel.
  \end{enumerate}
\end{rem}

\begin{proof}
  \mbox{}
  \begin{itemize}
  \item $(i)\Rightarrow(ii)$
  
  $f_{(S')}=f\times_{S}\id_{S'}$.

  \item $(ii)\Rightarrow(i)$
  
  Seien $f,g$ Morphismen (wie in 1) die $\mathbb{P}$ erfüllen. Da
    $f\times_{S}g=(f\times_{S}\id_{Y})\circ(\id_{X}\times_{S}g)$ sei
    ohne Einschränkung $g=\id_{Y}$.
    \begin{align*}
      f_{(X\times_{S}Y)}=f\times_{S}\id_{Y}: & X'\times_{S}Y=X'\underbrace{\times_{X}}_{\text{bzgl. }f}(X\times_{S}Y)\rightarrow X\times_{S}Y
    \end{align*}
    erfüllt $\mathbb{P}$.
  \end{itemize}
  In $\sch$ sind fast alle betrachteten Eigenschaften von Morphismen
  stabil unter Komposition, aber nicht unbedingt unter Basiswechsel,
  z.B. injektiv oder abgeschlossen.
\end{proof}
\begin{example*}
  Es ist:
  \begin{align*}
    f:X=\Spec\mathbb{Q}(\xi_{p}) & \longrightarrow\Spec\mathbb{Q}=S\\
    u:S' & \longrightarrow\Spec\mathbb{Q}
  \end{align*}

  Homöomorphismus, d.h. injektiv, aber
  \begin{align*}
    f_{(S')}:X\times_{S}S' & \longrightarrow\underbrace{S'=\Spec\mathbb{Q}(\xi_{p})}_{1\text{ Punkt}}
  \end{align*}

  ist nicht injektiv:
  \[
    \Spec(\mathbb{Q}(\xi_{p})\otimes\mathbb{Q}_{p})\cong\underbrace{\prod^{p-1}\mathbb{Q}(\xi_{p})}_{p-1\text{ Punkte}}.
  \]
\end{example*}
\textbf{Warnung.} Absolute Eigenschaften von Schemata sind oft nicht
kompatibel mit Basiswechsel. Sei $k=\mathbb{F}_{p}(t)$ (nicht perfekt!), $K=\bigcup_{n\geq1}\mathbb{F}_{p}(t^{\frac{1}{p^{n}}})$
perfekter Abschluss von $k$, $A:=K\otimes_{k}K$. Man kann zeigen:
$\nil(A)$ ist \emph{nicht} endlich erzeugt, d.h. $\Spec(A)$, $A$
ist nicht reduzibel und nicht noethersch.
\section{Fasern von Morphismen}

\textbf{Ziel:} $\xymatrix{X\ar[d]^{f}\supset f^{-1}(s)\\
  S\ni s
}
$ als Schema.
\begin{defn}[18]
  Für den kanonischen Morphismus $\Spec(\kappa(s))$ nennen wir
  \[
    X_{s}:=X\otimes_{S}\kappa(s)
  \]

  die \textbf{Faser von $f$ in $s$}, ein $\kappa(s)$-Schema. Proposition
  14 $\Longrightarrow$
  \[
    \xymatrix{X_{s}\ar[r]\ar[d] & S\times_{S}X\ar[r]^{q=\id_{X}}\ \ \ar[d]_{f} & X\ar[d]\\
      \Spec\kappa(s)\ar[r]_{\text{canon.}} & S\ar[r]_{\id_{S}} & S
    }
    \qquad\text{kommutativ,}
  \]

  besagt $X_{s}\cong f^{-1}(s)$ (=homoömorph zu Bild(canon)), d.h.
  wir können $f^{-1}(s)$ als Schema auffassen.

  \textbf{Denkweise}: $\xymatrix{X\ar[d]_{f}\\
    S
  }
  \cong$ Familie von $\kappa(s)$-Schemata $X_{s}$, parametrisiert durch
  Punkte von $S$.
\end{defn}

\begin{example}[19]
  Sei $k$ algebraisch abgeschlossen.
  \[
    X(k):=\{(u,t,s)\in\mathbb{A}^{3}(k)\mid ut=s\}
  \]

  Da $UT-s\in k[U,T,S]$ irreduzibel ist, ist $X(k)$ eine affine Varietät.
  $\leftrightarrow$ $X=\Spec(k[U,T,S]/(UT-S)$ ist ganzes $k$-Schema.
  Sei $S=\mathbb{A}^{1}$,
  \begin{align*}
    X & \longrightarrow S\\
    (u,t,s) & \longmapsto s
  \end{align*}

  Projektion. $s\in\mathbb{A}^{1}(k)$, $X_{s}=\Spec A_{s}$, 
  \begin{align*}
    A_{s} & =k[U,T,S]/(UT-S)\otimes_{k[S]}k[S]/(S-s)\\
          & \cong k[U,T]/(UT-S).
  \end{align*}

  $UT-s\in k[U,T]$ irreduzibel für $s\neq0$, reduzibel für $s=0$.
  $\Longrightarrow X\rightarrow S$ definiert Familie $X_{s}$ von $k$-Schemata,
  sodass $X_{0}$ reduzibel, $X_{s}$ irreduzibel für $s\neq0$.
\end{example}

\begin{lem}[20]
  Sei das Diagramm
  \[
    \xymatrix{ & X\times_{S}Y\ar[ld]_{p}\ar[rd]^{q}\\
      x\in X\ar[rd]_{f} &  & Y\ni y\ar[ld]^{g}\\
      \Spec\kappa(x)\ar[u]^{\xi} & S & \Spec\kappa(y)\ar[u]_{\psi}
    }
  \]

  Dann gilt:
  \begin{enumerate}
  \item Es gibt ein $z\in X\times_{S}Y$ mit $p(z)=x$, $q(z)=y$, genau dann
    wenn $f(x)=g(y).$
  \item Es gelte $(1)$, setze $s:=f(x)=g(y)$. Dann ist
    \begin{align*}
      \zeta:=\xi\times_{S}\psi:Z:=\Spec(\kappa(x)\otimes_{\kappa(s)}\kappa(y)) & \longrightarrow X\times_{S}Y
    \end{align*}
    ein Homöomorphismus von $Z$ auf Teilraum $\zeta(Z)=p^{-1}(x)\cap q^{-1}(y)$.
  \end{enumerate}
\end{lem}

\begin{proof}
  Setze $Z:=p^{-1}(x)\times_{(X\times_{S}Y)}q^{-1}(y)$, und betrachte:
  \[
    \xymatrix{ &  & Z\ar[rd]^{g'}\ar[ld]_{f'}\\
      & p^{-1}(x)\ar[rd]\ar[ld] &  & q^{-1}(y)\ar[ld]\ar[rd]\\
      \Spec\kappa(s)\ar[rd] &  & X\times_{S}Y\ar[ld]\ar[rd] &  & \Spec\kappa(y)\ar[ld]\\
      & X\ar[rd] &  & Y\ar[ld]\\
      &  & S
    }
  \]

  Wende Proposition 14 zweifach an $\Longrightarrow$
  \[
    g'(Z)=i^{-1}(p^{-1}(x))=q^{-1}(y)\cap p^{-1}(x).
  \]
\end{proof}

\section{Eigenschaften von Schemata-Morphismen}
\begin{notation}[21]
  Sei $\mathbb{P}$ Eigenschaft von Morphismen von Schemata.
  \begin{enumerate}
  \item $\mathbb{P}$ heißt \textbf{lokal im Ziel}, falls für alle Morphismen
    $f$ und alle offenen Überdeckungen $S=\bigcup_{j\in J}S_{j}$ gilt:
    \[
      f:X\rightarrow S\text{ erfüllt }\mathbb{P}\Longleftrightarrow f|_{f^{-1}(S_{j})}:f^{-1}(S_{j}):S_{j}\text{ erfüllt }\mathbb{P}\ \forall j\in J
    \]
  \item $\mathbb{P}$ heißt \textbf{lokal in der Quelle}, falls für alle $f:X\rightarrow Y$
    und alle offenen Überdeckungen $X=\bigcup_{i\in I}U_{i}$ gilt:
    \[
      f\text{ erfüllt }\mathbb{P}\Longleftrightarrow f|_{U_{i}}:U_{i}\rightarrow Y\text{ erfüllt }\mathbb{P}\ \forall i\in I
    \]
  \end{enumerate}
\end{notation}

\begin{defn*}
  Ein Morphismus $f:X\rightarrow Y$ von Schemata heißt \textbf{(treu)flach},
  falls $\forall x\in X$ die Abbildung
  \[
    \mathcal{O}_{Y,f(x)}\longrightarrow\mathcal{O}_{X,x}
  \]

  (treu)flach ist.
\end{defn*}
\begin{prop}[22]
  Die folgende Eigenschaft von Schemata-Morphismen sind:
  \begin{enumerate}
  \item stabil unter Komposition: ,,injektiv``, ,,surjektiv``, ,,bijektiv``,
    ,,homöomorph``, ,,flach``, ,,treuflach``, ,,offen``, ,,abgeschlossen``,
    ,,offene Immersion``, ,,abgeschlossene Immersion``, ,,Immersion``;
  \item stabil unter Basiswechsel: ,,surjektiv``, ,,offene Immersion``,
    ,,abgeschlossene Immersion``, ,,Immersion``, ,,flach``, ,,treuflach``;
  \item lokal bzgl. Ziel: ,,surjektiv``, ,,bijektiv``, ,,homöomorph``,
    ,,offen``, ,,abgeschlossen``, ,,offene Immersion``, ,,abgeschlossene
    Immersion``, ,,Immersion``, ,,flach``, ,,treuflach``;
  \item lokal bzgl. Quelle: ,,offen``, ,,flach``.
  \end{enumerate}
\end{prop}

\begin{proof}
  Die Fälle $(1)$, $(3)$, $(4)$ sind klar. Der Fall $(2)$ für offene/abgeschlossene
  Immersionen wird in Abschnitt 9 behandelt. Lemma 20 $\Longrightarrow$
  ,,surjektiv`` stabil unter Basiswechsel.
  \[
    \xymatrix{X\times_{S}Y\ar[r]^{q} & Y & y\ar@{|->}[d]\\
      X\ar@{->>}[r]^{f} & S & s\\
      \exists x\ar@/_{1pc}/@{|->}[rru]
    }
  \]

  Sei $X\rightarrow S$ (treu)flach und $S'\rightarrow S$ beliebig.
  Sei in Fall $(3)$, $(4)$ ohne Einschränkung $X=\Spec A$, $S=\Spec R$,
  $S'=\Spec R'$, $A$ (treu)flache $R$-Algebra. $\Longrightarrow A\otimes_{R}R'$
  (treu)flache $R'$-Algebra, d.h. $f_{(S')}$ ist treuflach.
  \[
    (A\otimes_{R}R')\otimes_{R'}M\cong A\otimes_{R}M
  \]
\end{proof}
\begin{cor}[23]
  Die folgende Eigenschaften sind stabil unter Komposition, stabil
  unter Basiswechsel und lokal bzgl. Ziel:

  ,,universal injektiv``, ,,universal bijektiv``, ,,universal homöomorph``,
  ,,universell offen``, ,,universell abgeschlossen``.
\end{cor}

\section{Urbilder und Schema-theoretische Schnitte von Unterschemata}

Sei $f:X\rightarrow Y$ ein Morphismus von Schemata und $i:Z\rightarrow Y$
eine Immersion.
\[
  \xymatrix{Z\times_{Y}X\ar[r]\ar[d]_{i_{(X)}} & Z\ar[d]^{i}\\
    X\ar[r]_{f} & Y
  }
\]

Proposition 14 $\Longrightarrow i_{(X)}$ ist surjektiv auf Halmen,
Homöomorphismus von $Z\times_{Y}X$ auf lokal abgeschlossene Teilmenge
$f^{-1}(Z)$ (genau $f^{-1}(i(Z))$), d.h. $i_{(X)}$ ist Immersion.
Fasse $Z\times_{Y}X$ als Unterschema von $X$ auf, das \textbf{Urbild
  von $Z$ unter $f$}.
\begin{rem*}
  \mbox{}
  \begin{enumerate}
  \item Ist $Z\subset Y$ offenes Unterschema, so auch $f^{-1}(Z)\subset X$.
  \item Ist $Z=V(\mathfrak{p})$ abgeschlossenes Unterschema, $\mathfrak{p}\subset\mathcal{O}_{Y}$
    Idealgarbe, so auch 
    \begin{align*}
      f^{-1}(Z) & =V(f^{\ast^{-1}}(\mathfrak{p})\mathcal{O}_{X})\\
                & =\text{Bild}(f^{\ast^{-1}}(\mathfrak{p})\rightarrow f^{\ast^{-1}}\mathcal{O}_{Y}\rightarrow\mathcal{O}_{X}).
    \end{align*}
  \end{enumerate}
  \textbf{Spezialfall}: Durchschnitt von 2 Unterschemata $i:Y\rightarrow X$,
  $j:Z\rightarrow X$:
  \[
    Y\cap Z:=Y\times_{X}Z=i^{-1}(Z)=j^{-1}(Y)
  \]

  heißt \textbf{(Schema-theoretischer) Durchschnitt von $Y$ und $Z$
    in $X$}.
\end{rem*}

\subsubsection*{Universelle Eigenschaft (aus univ. Eig. Faserprodukt)}

Ein Morphismus $h:T\rightarrow X$ faktorisiert durch $Y\cap Z$ genau
dann wenn $h$ faktorisiert durch $Y$ und $Z$. Sind $Y=V(\mathfrak{p})$,
$Z=V(\mathfrak{q})$ abgeschlossene Unterschemata, so folgt:
\begin{align*}
  V(\mathfrak{p})\cap V(\mathfrak{q}) & =V(\mathfrak{p}+\mathfrak{q})\\
  A/\mathfrak{p}\otimes_{A}A/\mathfrak{q} & \cong A/\mathfrak{q+}\mathfrak{p}
\end{align*}

\begin{example*}
  $f_{1},\ldots,f_{r},g_{1},\ldots,g_{s}\in R[X_{0},\ldots,X_{n}]$
  homogene Polynome. Dann ist:
  \[
    V_{+}(f_{1},\ldots,f_{r})\cap V_{+}(g_{1},\ldots,g_{s})=V_{+}(f_{1},\ldots,f_{r},g_{1},\ldots,g_{s})\subseteq\mathbb{P}_{R}^{n}
  \]
\end{example*}

\section{Affine und projektive Räume über beliebige Basen}
\begin{itemize}
\item[] $\mathbb{A}^{n}:=\mathbb{A}_{\mathbb{Z}}^{n}$, $S$ beliebiges Schema.
\item[] $\mathbb{A}_{S}^{n}:=\mathbb{A}^{n}\times_{\mathbb{Z}}S$ \textbf{affiner
    Raum der relativen Dimension $n$ über $S$}.
\item[] $\mathbb{P}_{S}^{n}:=\mathbb{P}^{n}\times_{\mathbb{Z}}S$ \textbf{projektiver
    Raum der relativen Dimension $n$ über $S$}.
\item[] $S=\Spec R$ affin:
  \begin{align*}
    \mathbb{A}_{S}^{n} & =\Spec(\mathbb{Z}[T_{1},\ldots,T_{n}]\otimes_{\mathbb{Z}}R)=\Spec(R[T_{1},\ldots,T_{n}])\\
                       & =\mathbb{A}_{R}^{n}\text{ wie zuvor!}\\
    \mathbb{P}_{S}^{n} & =\mathbb{P}_{R}^{n}\text{ analog.}
  \end{align*}
\item[] $\mathbb{A}_{n}^{S}\times_{S}S'=\mathbb{A}^{n}\times_{\mathbb{Z}}S\times_{S}S'=\mathbb{A}_{S'}^{n}$
\item[] $\mathbb{P}_{S}^{n}\times_{S}S'=\mathbb{P}_{S'}^{n}$ für einen beliebigen
  Basiswechsel $S'\rightarrow S$.
\end{itemize}
Sei $X$ ein beliebiges Schema.
\begin{align*}
  \Gamma(X,\mathcal{O}_{X}) & =\Hom_{\ring}(\mathbb{Z}[T],\Gamma(X,\mathcal{O}_{X}))\\
                            & =\Hom_{\schz}(X,\mathbb{A}_{\mathbb{Z}}^{1})\\
  \varphi(T) & \mapsfrom\varphi
\end{align*}

Sei $X$ ein $S$-Schema.
\[
  \Gamma(X,\mathcal{O}_{X})=\Hom_{\schs}(X,\mathbb{A}_{S}^{1})
\]

\section{Diagonal, Graph und Kern in beliebigen Kategorien}

Sei $\mathcal{C}$ Kategorie mit Faserprodukten, $S\in\mathcal{C}$,
$X,T\in\mathcal{C}/S$, $X_{S}(T)$ Menge der $S$-Morphismen.
\begin{defn}[24]
  Der Morphismus
  \begin{enumerate}
  \item $\Delta_{X/S}:=\Delta_{u}:=(\id_{X},\id_{X}):X\rightarrow X\times_{S}X$,
    $u:X\rightarrow S$, heißt \textbf{Diagonale }(diagonaler Morphismus)
    \textbf{von $X$ über $S$}.
  \item Sei $f:X\rightarrow Y\in$ Morph/$S$. Der Morphismus
    \[
      \Gamma_{j}:=(\id_{X},f)_{S}:X\longrightarrow X\times_{S}Y
    \]
    heißt der \textbf{Graph(morphismus) von $f$}.
  \item Seien $f,g:X\rightarrow Y\in$ Morph/$S$. Ein $S$-Monomorphismus
    $i:K\rightarrow X$ heißt \textbf{(Differenzen)kern} von $f$ und
    $g$, falls für alle $T\in\mathcal{C}/S$ die Abbildung $i(T):K_{S}(T)\rightarrow X_{S}(T)$
    injektiv ist mit
    \[
      \text{Bild}(i(T))=\{x\in X_{S}(T)\mid f(T)(x)=g(T)(x)\}.
    \]
    Beizchne $K(f,g)_{S}$ oder $\ker(f,g)$, $i$ ,,kanonisch``. Mit
    anderen Worten, $\ker(f,g)$ separiert den Funktor
    \begin{align*}
      \mathcal{C}/S & \longrightarrow\sch\\
      T & \longmapsto\{x\in X_{S}(T)\mid f(T)(x)=g(T)(x)\}
    \end{align*}
  \end{enumerate}
\end{defn}

\begin{example}[25]
  In der Kategorie $\mathcal{C}=\set$ gilt für:
  \[
    \xymatrix{X\ar[r]^{f,g}\ar[rd]_{u} & Y\ar[d]^{v}\\
      & S
    }
  \]
  \begin{itemize}
  \item[]
    $\begin{array}{rl}
       \Delta_{u}:X & \longrightarrow X\times X=\{(x,x')\in X\times X\mid u(x)=u(x')\}\\
       x & \longmapsto(x,x)
     \end{array}$
   \item[]
     $\begin{array}{rl}
        \Gamma_{j}:X & \longrightarrow X\times_{S}Y=\{(x,y)\in X\times Y\mid u(x)=v(y)\}\\
        x & \longmapsto(x,f(x))
      \end{array}$
    \item[]
      $\ker(f,g)=\{x\in X\mid f(x)=g(x)\}$
    \end{itemize}
    Da $p\circ\Gamma_{j}=\id_{X}$, sind $\Gamma_{j}$ und $\Delta_{X/S}=\Gamma_{\id_{X}}$
    Monomorphismen.
\end{example}

\begin{prop}[26]
  Seien $f,g:X\rightarrow Y$ $S$-Morphismen.
  \begin{enumerate}
  \item $\Delta_{X/S}=\Gamma_{\id_{X}}$,
    
    $\Gamma_{f}=\ker(\xymatrix{X\times_{S}Y\ar@<1ex>[r]^{q}\ar@<-1ex>[r]_{f\circ p} & Y}
    )\longrightarrow X\times_{S}Y$ kanonisch.
  \item Alle Rechtecke des folgenden kommutativen Diagramms sind kartesisch:
    \[
      \xymatrix{\ker(f,g)\ar[r]^{\text{canon.}}\ar[d] & X\ar[r]^{f}\ar[d]_{\Gamma_{f}} & Y\ar[d]^{\Delta_{Y/S}}\\
        X\ar[r]_{\Gamma_{g}} & X\times_{S}Y\ar[r]_{f\times\id_{S}} & Y\times_{S}Y
      }
    \]
  \item Sei $s:S\rightarrow X$ ein Schnitt von $f$ ($f\circ s=\id_{S}$).
    Dann ist das folgende Diagramm kartesisch:
    \[
      \xymatrix{S\ar[r]^{s}\ar[d]_{s} & X\ar[d]^{\Gamma_{s\circ f}}\\
        X\ar[r]_{\Delta_{X/S}} & \quad X\times_{S}X
      }
    \]
  \end{enumerate}
\end{prop}

\begin{proof}
  Nach dem Yoneda-Lemma reicht es, denn Fall $\mathcal{C}=\set$ zu
  verifizieren. Dies ist elementar aufgrund der Beschreibung in Beispiel
  25.
\end{proof}
Insbesondere existiert $\ker(f,g)$ stets!

\section{Diagonal für Schemata}
\begin{prop}[27]
  Für affine $S$-Schemata
  \begin{align*}
    X & =\Spec(B)\overset{u}{\longrightarrow}S=\Spec(R)\\
    Y & =\Spec(A)\overset{v}{\longrightarrow}S
  \end{align*}

  und $S$-Morphismus $f=\Spec\varphi:X\rightarrow Y$ zu einem $R$-Algebra
  Morphismus $\varphi:A\rightarrow B$, entsprechen $\Delta_{X/S}$
  und $\Gamma_{f}$ den folgenden surjektiven Ringhomomorphismen:
  \[
    \begin{array}{rl}
      \Delta_{B/R}:B\otimes_{R}B & \longrightarrow B\\
      b\otimes b' & \longmapsto bb'
    \end{array},\qquad
    \begin{array}{rl}
      \Gamma_{\varphi}:A\otimes_{R}B & \longrightarrow B\\
      a\otimes b & \longmapsto\varphi(a)b
    \end{array}.
  \]
  
  Insbesondere sind $\Delta_{X/S}$, $\Gamma_{j}$ abgeschlossene Immersionen.
\end{prop}

Im allgemeinen sind $\Delta_{X/S}$, $\Gamma_{f}$ Immersionen (nicht
notwendig abgeschlossen!): Seien $Z,Z'\subset X$ Unterschemata. $\Longrightarrow Z\times_{S}Z'\subset X\times_{S}X$
Unterschemata (Immersionen und stabil unter Basiswechsel und Komposition),
und
\[
  Z\cap Z'=\Delta_{X/S}^{-1}(Z\times_{S}Z')\qquad(*)
\]

\begin{prop}[28]
  Seien $X,Y\in\schs$, $f,g:X\rightarrow Y$ $S$-Morphismen. Dann
  sind $\Delta_{X/S}$, $\Gamma_{f}$, $\ker(f,g)\rightarrow X$ Immersionen.
\end{prop}

\begin{proof}
  Es reicht zu zeigen: $\Delta_{X/S}$ ist eine Immersion (und $(2)$
  in Proposition 26, da ,,Immersion`` stabil ist unter Basiswechsel)
  lokal bzgl. Ziel. Sei also ohne Einschränkung $S$ affin. Falls $X=\bigcup_{i\in I}U_{i}$
  offene Überdeckung, dann ist $\Delta_{X/S}(X)=\bigcup_{i\in I}U_{i}\times_{S}U_{i}$
  offene Überdeckung. 
  \[
    \xymatrix@C=4pc{X\ar[r]_(0.3){\text{abg. Imm.}} & \bigcup_{i\in I}(U_{i}\times_{S}U_{i})\ar@{^{(}->}_(0.6){\text{off. Imm.}}[r] & X\times_{S}X}
  \]
  
  $(*)\Longrightarrow$ ohne Einschränkung, $X$ affin. Wende nun Proposition
  27 an.
\end{proof}
Das Unterschema
\begin{itemize}
\item $X\cong\Delta_{X/S}(X)\subset X\times_{S}X$ heißt die \textbf{Diagonale
    von $X\times_{S}X$}.
\item $\Gamma_{f}(X)\subset X\times_{S}Y$ heißt der \textbf{Graph von $f$}.
\end{itemize}

\begin{rem}[29]
  \mbox{}
  \begin{enumerate}
  \item Ein Unterschema $T\subset X\times_{S}Y$ ist der Graph eines $S$-Morphismus
    $f:X\rightarrow Y$ genau dann, wenn $p|_{T}:T\rightarrow X\times_{S}Y\underset{p}{\rightarrow}X$
    ein Isomorphismus ist, \emph{denn} $f=q\circ(p|_{T})^{-1}$.
  \item Im Allgemeinen ist die mengentheoretische Inklusion
    \[
      \Delta_{X/S}(X)\subset\{z\in X\times_{S}X\mid f(z)=g(z)\}
    \]
    \emph{keine} Gleichheit!
  \end{enumerate}
\end{rem}

\section{Separierte Morphismen}

Erinnerung: Für einen topologischen Raum $X$ sind äquivalent:
\begin{enumerate}
\item $X$ ist Hausdorff.
\item $\Delta\subset X\times X$ ist abgeschlossen bzgl. der Produkttopolgie.
\item Für jedes Paar von stetigen Abbildungen $f,g:Y\rightarrow X$ ist
  $\ker(f,g)\subset X$ abgeschlossen.
\item Für jeder stetige Morphismus $f:Y\rightarrow X$ ist $\Gamma_{f}\subset X\times Y$
  abgeschlossen.
\end{enumerate}
Für ein Schema $X$ ist der unterliegende topologische Raum selten
Hausdorff, aber $(2)-(4)$ geben sinnvolle Konzepte für Schemata (im
Allgemeinen ist die Produkttopologie ungleich der Faserprodukttopologie).
\begin{defn}[30]
  Ein Morphismus $v:Y\rightarrow S$ von Schemata heißt separiert,
  falls folgende äquivalente Bedingungen erfüllt sind.
  \begin{enumerate}
  \item $\Delta_{Y/S}$ ist eine \emph{abgeschlossene} Immersion.
  \item Für jedees Paar $f,g:X\rightarrow Y$ ist $\ker(f,g)\subset X$ ein
    abgeschlossenes Unterschema.
  \item Für jeden $S$-Morphismus $f:X\rightarrow Y$ ist $\Gamma_{f}$ eine
    abgeschlossene Immersion.
  \end{enumerate}
  Dann heißt auch $Y$ ist \textbf{separiert über $S$}. Ein Schema
  $Y$ heißt \textbf{separiert}, falls es separiert über $\mathbb{Z}$
  ist.
\end{defn}

\begin{proof}
  Die Äquivalenz folgt nach Proposition 26, und dass ,,abgeschlossene
  Immersion`` stabil unter Basiswechsel ist.
\end{proof}
Nach Proposition 27 ist jeder Morphismus zwischen affinen Schemata
separiert. Insbesondere ist jedes affine Schemata separiert.
\begin{prop}[31]
  Seien $X,Y\in\schs$, $Y$ separiert über $S$, $U\subset X$ offenes
  dichtes Unterschema, $f,g:X\rightarrow Y$ $S$-Morphismus mit $f|_{U}=g|_{U}$.
  Dann ist $f|_{X_{\red}}=g|_{X_{\red}}$.
\end{prop}

\begin{proof}
  Nach Voraussetzung ist $U\subseteq\ker(f,g)$. Da $Y$ separiert ist
  über $S$, ist $\ker(f,g)\subset X$ abgeschlossenes Unterschema.
  Da $U$ dicht ist in $X$, ist der unterliegende topologische Raum
  von $\ker(f,g)$ gleich $X$. $\Longrightarrow X_{\red}\subseteq\ker(f,g)$
  als Schema.
\end{proof}
\begin{example}[32]
  Sei
  \[
    \,\xymatrix{\ar@{-}[r] & :\ar@{-}[r] & \,}
  \]

  affine Gerade mit Doppelpunkt (siehe Beispiel 11, III.5) ist \emph{nicht}
  separiert: $V\subset U$ offen, nicht abgeschlossen.
  \[
    j,j':U\longrightarrow U\cup_{V}U\quad\Rightarrow\quad\ker(j,j')=V\subset U\subset X\text{\,nicht abg.!}
  \]
\end{example}

\begin{rem}[33]
  $\mathbb{P}$ sei eine Eigenschaft von Morphismen, sodass gilt:
  \begin{itemize}
  \item stabil unter Komposition und Basiswechsel;
  \item jede (abgeschlossene) Immersion erfült $\mathbb{P}$.
  \end{itemize}
  Für jedes kommutative Diagramm:
  \[
    \xymatrix{X\ar[r]^{f}\ar[rd]_{u} & Y\ar[d]^{v}\\
      & S
    }
  \]

  mit $u$ erfüllt $\mathbb{P}$ (und $v$ seperariert) $\Longrightarrow f$
  erfüllt $\mathbb{P}$, da:
  \[
    f:X\xrightarrow[\text{(abg.) Imm. erfüllt }\mathbb{P}]{\Gamma_{f}}X\times_{S}Y\xrightarrow[\text{Basisw. erfüllt }\mathbb{P}]{q}Y
  \]

  erfüllt $\mathbb{P}$, wegen stabil unter Komposition.
\end{rem}

\begin{prop}[34]
  \mbox{}
  \begin{enumerate}
  \item Jeder Monomorphismsu von Schemata (insbesondere jede Immersion) ist
    separiert.
  \item Die Eigenschaft ,,separiert`` ist stabil unter Komposition, stabil
    unter Basiswechsel, und lokal bzgl. Ziel.
  \item Ist die Komposition $X\rightarrow Y\rightarrow Z$ zweier Morphismen
    separiert, so auch $X\rightarrow Y$.
  \item $f:X\rightarrow Y$ ist seperariert genau dann, wenn $f_{\red}:X_{\red}\rightarrow Y_{\red}$
    separiert ist.
  \end{enumerate}
\end{prop}

\begin{proof}
  \mbox{}
  \begin{enumerate}
  \item Wenn $f$ Monomorphismus ist (d.h. injektiv auf $T$-wertigen Punkten
    für alle Schemata $T$), dann ist $\Delta_{f}$ Isomorphismus (d.h.
    bijektiv auf allen $T$-wertigen Punkte für alle $T$). Insbesondere
    ist $\Delta_{f}$ eine abgeschlossene Immersion.
  \item Seien $f:X\rightarrow Y$, $g:Y\rightarrow Z$ separierte Schemata-Morphismen,
    $p,q:X\times_{Y}X\rightarrow X$ die zwei Projektionen. Das folgende
    Diagramm ist kommutativ, und das rechte Viereck ist kartesisch (überprüfe
    in $\set$):
    \[
      \xymatrix{X\ar[r]^{\Delta_{f}}\ar[rd]_{\Delta_{g\circ f}} & X\times_{Y}X\ar[r]^(0.6){f\circ p=f\circ q}\ar[d]|-{(p,q)_{Z}} & Y\ar[d]^{\Delta_{g}}\\
        & X\times_{Z}X\ar[r]_{f\times f} & Y\times_{Z}Y.
      }
    \]
    Da $\Delta_{g}$ abgeschlossene Immersion, ist $(p,q)_{Z}$ abgeschlossene
    Immersion $\Longrightarrow$ die Komposition $\Delta_{g\circ f}$
    ist abgeschlossene Immersion $\Longrightarrow g\circ f$ ist separiert.

    $\Delta_{f}$ abgeschlossene Immersion $\Longrightarrow\Delta_{f_{(S')}}$
    ist abgeschlossene Immersion. Dies zeigt das ,,separiert`` abgeschlossen
    ist unter Basiswechsel. Weiter ist ,,separiert`` lokal bzgl. Ziel,
    da dies gilt für ,,abgeschlossene Immersion``.
  \item Folgt aus (1), (2) nach Bemerkung 33. ($u=``\circ"$, $v=Y\rightarrow Z$,
    $f:X\rightarrow Y$)
  \item Sei $f:X\rightarrow S$ Morphismus, $i:X_{\red}\rightarrow X$ kanonische
    Immersion. Dann ist $i$ surjektive Immersion, also ein universeller
    Homöomorphismus. Identifizieren von $X_{\red}\times_{S_{\red}}X_{\red}$
    mit $X_{\red}\times_{S}X_{\red}$ liefert $\Delta_{f}\circ i=(i\times_{S}i)\circ\Delta_{f_{\red}}$.
    $\Longrightarrow\Delta_{f}$ ist abgeschlossene Immersion genau dann
    wenn $\Delta_{f_{\red}}$ abgeschlossene Immersion.
  \end{enumerate}
\end{proof}

\begin{example}[35]
  Sei $S$ beliebiges Schema, $n\in\mathbb{N}$. Dann ist $\mathbb{A}_{S}^{n}$
  separiert über $S$, ebenso jedes Unterschema, denn $\mathbb{A}_{S}^{n}=\mathbb{A}_{\mathbb{Z}}^{n}\times_{\Spec\mathbb{Z}}S$
  und ,,separiert`` ist stabil unter Basiswechsel nach Proposition
  34.
\end{example}

\begin{prop}[36]
  Sei $S=\Spec R$ affin und $X$ ein $S$-Schema. Dann sind äquivalent:
  \begin{enumerate}
  \item $X$ ist separiert.
  \item Für je zwei offene affine $U,V\subseteq X$ ist $U\cap V$ affin,
    und
    \begin{align*}
      \rho_{U,V}:\mathcal{O}_{X}(U)\otimes_{R}\mathcal{O}_{X}(V) & \longrightarrow\mathcal{O}_{X}(U\cap V),\\
      s\otimes t & \longmapsto s|_{U\cap V}\cdot t|_{V\cap U}.
    \end{align*}
  \item Es gibt eine offene affine Überdeckung $X=\bigcup_{i\in I}U_{i}$,
    sodass $\forall i,j\in I$: $\rho_{U,V}$ ist surjektiv.
  \end{enumerate}
\end{prop}

\begin{proof}
  Für alle offene affine $U,V\subseteq X$ gilt:
  \[
    U\cap V=\Delta_{X/S}^{-1}(U\times_{S}V).
  \]

  ,,abgeschlossene Immersion`` ist lokal auf dem Ziel, daher: $\Delta_{X/S}$
  ist abgeschlossene Immersion

  $\Longleftrightarrow$ für alle $U,V\subseteq X$ offen affin ist
  \[
    U\cap V\xrightarrow{\Delta_{X/S}|_{U\cap V}}U\times_{S}V
  \]

  abgeschlossene Immersion.

  $\Longleftrightarrow$ für jede offene affine Überdeckung $X=\bigcup_{i\in I}U_{i}$
  und alle $i,j\in I$ ist
  \[
    U_{i}\cap U_{j}\longrightarrow U_{i}\times_{S}U_{j}
  \]

  abgeschlossene Immersion. Sind $U=\Spec A$, $V=\Spec B$ affin, so
  ist auch
  \[
    U\times_{S}V=\Spec(A\otimes_{R}B)
  \]

  affin. Daher:
  \[
    U\cap V\longrightarrow U\times_{S}V
  \]

  abgeschlossene Immersion. $\Longleftrightarrow\rho_{U,V}$ surjektiv.
\end{proof}
\begin{example}[37]
  Für jedes Schema $S$ und $n\in\mathbb{N}$ ist $\mathbb{P}_{S}^{n}$
  separiert über $S$, denn ,,separiert`` ist lokal auf dem Ziel (Proposition
  34), daher ohne Einschränkung $S=\Spec R$ affin. Sei $\mathbb{P}_{R}^{n}=\bigcup_{i=0}^{n}U_{i}$
  mit $U_{i}=\Spec R\left[\frac{X_{0}}{X_{i}},\ldots,\frac{\widehat{X_{i}}}{X_{i}},\ldots,\frac{X_{n}}{X_{i}}\right]$
  und
  \begin{align*}
    \rho_{U_{i},U_{j}}:R & \left[\frac{X_{0}}{X_{i}},\ldots,\frac{\widehat{X_{i}}}{X_{i}},\ldots,\frac{X_{n}}{X_{i}}\right]\otimes_{R}R\left[\frac{X_{0}}{X_{j}},\ldots,\frac{\widehat{X_{j}}}{X_{j}},\ldots,\frac{X_{n}}{X_{j}}\right]\\
                         & \longrightarrow R\left[\frac{X_{0}}{X_{i}},\ldots,\frac{\widehat{X_{i}}}{X_{i}},\ldots,\frac{X_{n}}{X_{i}}\right]\left[\frac{X_{i}}{X_{j}}\right]
  \end{align*}

  ist surjektiv.
\end{example}

\begin{example}[38]
  Sei $k$ algebraisch abgeschlossener Körper, $X$ Prävarietät über
  $k$, d.h. ganzes Schema von endlichem Typ über $k$. $X$ heißt \textbf{Varietät},
  wenn $X$ separabel ist. Affine Prävarietäten sind also automatisch
  Varietäten. $\mathbb{P}^{n}(k)$ ist Varietät (Beispiel 37) $\Longrightarrow$
  Jede quasi-projektive Prävarietät ist eine Varietät!
\end{example}

\section{Eigentliche Morphismen}

(eng. ,,proper``)\medskip{}

Ist $f:X\rightarrow Y$ stetige Abbildung zwischen topologischen Räume,
dann heißt $f$ \textbf{eigentlich}, wenn die Urbilder kompakter Mengen
kompat sind.

Sei $X$ Hausdorff, $Y$ lokal kompakt. Dann ist $f$ eigentlich $\Longleftrightarrow f$
universell abgeschlossen. (Bourbaki, Topologie générale, $I$.10 nr.
3, Prop. 7)

$f:X\rightarrow Y$ heißt von \textbf{endlichem Typ}, wenn $\forall U\subset Y$
offen eine Überdeckung $f^{-1}(U)=\bigcup_{i=1}^{n}V_{i}$ existiert,
sodass $\forall i:\mathcal{O}_{X}(V_{i})$ ist endlich erzeugte $\mathcal{O}_{Y}(U)$-Algebra.%

\begin{defn}[39]
  Ein Morphismus $f:X\rightarrow Y$ von Schemata heißt \textbf{eigentlich},
  wenn:
  \begin{enumerate}
  \item $f$ von endlichem Typ.
  \item $f$ separiert.
  \item $f$ universell abgeschlossen.
  \end{enumerate}
  Ein $Y$-Schema heißt \textbf{eigentlich}, wenn der Strukturmorphismus
  eigentlich ist. ,,eigentlich`` ist lokal auf dem Ziel.\medskip{}
\end{defn}

\begin{defn}[40]
  Ein Morphismus $f:X\rightarrow Y$ heißt \textbf{projektiv}, wenn
  er sich faktorisieren lässt als
  \[
    \xymatrix{X\ar[rr]^{f}\ar[rd]_{\text{abg. Imm.}}^{g} &  & Y\\
      & \mathbb{P}_{Y}^{n}\ar[ur]_{\text{kan. Morph.}}
    }
  \]

  für ein $n\in\mathbb{N}$.
\end{defn}

\begin{prop}[41]
  Sei $\mathbb{P}$ eine der folgenden Eigenschaften:
  \begin{itemize}
  \item[I.] von endlichem Typ;
  \item[II.] eigentlich;
  \item[III.] projektiv.
  \item[IV.] (separiert)
  \end{itemize}
  Dann gilt:
  \begin{enumerate}
  \item Abgeschlossene Immersionen erfüllen $\mathbb{P}$.
  \item $\mathbb{P}$ ist stabil unter Komposition.
  \item $\mathbb{P}$ ist stabil unter Basiswechsel.
  \item Falls $f:X\rightarrow Z$, $g:Y\rightarrow Z$ $\mathbb{P}$ erfüllen,
    dann auch $X\times_{Z}Y\rightarrow Z$.
  \item Erfüllt $X\xrightarrow{f}Y\xrightarrow{g}Z$ $\mathbb{P}$, so auch
    $f$, falls:
    \begin{itemize}
    \item $f$ quasi-kompakt (d.h. $Y$ hat offene affine Überdeckung, deren
      Urbilder quasi-kompakt sind).
    \item $g$ separiert, im Falle II, III.
    \item (stetig im Fall IV)
    \end{itemize}
  \end{enumerate}
\end{prop}

% Die Beweis-Skizze aus der Vorlesung fehlt.
\begin{proof}
	Lin, ,,Algebraic Geometry and Arithmetic curves``, Prop.
	24, 3.16, 3.32, (3.9).
\end{proof}
\begin{prop}[42]
  Sei $f:X\rightarrow Y$ surjektiver Morphismus von $S$-Schemata,
  und $Y$ separiert von endlichem Typ über $S$, sowie $X$ eigentlich
  über $S$. Dann ist $Y$ eigentlich über $S$.
\end{prop}

\begin{proof}
  $f$ surjektiv $\Longrightarrow\forall T\rightarrow S$ ist $f_{(T)}:X_{T}\rightarrow Y_{T}$
  surjektiv.

  $\Longrightarrow\xymatrix{A\underset{\text{abg.}}{\subset}Y_{T}\ar[r] & T\\
    & \varphi^{-1}(A)\underset{\text{abg.}}{\subset}X_{T}\ar@{->>}[ul]_{\varphi}\ar[u]_{X\text{ eig.}\Rightarrow\text{abg.}}
  }
  $

  $\Longrightarrow Y_{T}\rightarrow T$ abgeschlossen.

  $\Longrightarrow Y\rightarrow S$ universal abgeschlossen.
\end{proof}
\begin{prop}[43]
  Sei $X$ eigentliches Schema über $S=\Spec A$. Dann ist $\mathcal{O}_{X}(X)$
  ganz über $A$. Ist $X=\Spec B$ affin, so ist $B$ endlich über $A$.
\end{prop}

\begin{proof}
  Lin, ,,Algebraic Geometry and Arithmetic curves``, 3.17/3.18.
\end{proof}
\begin{cor}
  Sei $X$ reduzierte eigentliche Varietät über $k$. Dann ist $\mathcal{O}_{X}(X)$
  endlich-dimensionaler $k$-Vektorraum.
\end{cor}

\begin{thm}[45]
  Sei $\mathcal{O}_{K}$ Bewertungsring, $K=\Quot(\mathcal{O}_{X})$,
  $X/\mathcal{O}_{K}$ eigentlich. Dann ist $X_{\mathcal{O}_{K}}(\mathcal{O}_{K})\rightarrow X_{K}(K)$
  bijektiv.
\end{thm}

\textbf{Bewertungskriterium }(vgl. Hartshorne, Lin 3.26)
\[
  \xymatrix{\Spec K\ar[r]\ar[d] & X\ar[d]^{f}\\
    \Spec\mathcal{O}_{K}\ar@{-->}[ur]|-{\exists!}\ar[r] & Y
  }
\]

$f$ eigentlich $\Longleftrightarrow$ universelle Eigenschaft oben
erfüllt. (Theorem auf $X\times_{Y}\Spec\mathcal{O}_{K}\rightarrow\Spec\mathcal{O}_{K}$).
\begin{thm}[46, Lin III 3.30]
  Jeder projektive Morphismus ist eigentlich.
\end{thm}

Zum Beispiel: abelsche Varietäten, etwa elliptische Kurven. Theorem
46 $\Longrightarrow E(\mathbb{Q}_{p})=E(\mathbb{Z}_{p})$.


\chapter{Dimensionen}
\label{chap:dimensionen}

<<<<<<< HEAD
\section{Allgemine Schemata}

Hallo
=======
\section{Allgemeine Schemata}
\begin{defn}[1]
  Für einen topologischen Raum $X$ ist die (Krull-)Dimension das Supremum
  der Länge aller Ketten
  \[
    Z_{0}\subsetneq Z_{1}\subsetneq\cdots\subsetneq Z_{n}\subseteq X
  \]

  irreduzibler abgeschlossener Teilmengen $Z_{i}$. $X$ sei \textbf{von
    Dimension $n$}, falls alle irreduzible Komponenten von $X$ die Dimension
  $n$ haben ($\dim\emptyset=-\infty$, sonst $\dim X\in\mathbb{N}\cup\{+\infty\}$).

  Die Dimension eines Schemas ist per Definition die Dimension des unterliegenden
  topologischen Raums, also $\dim X=\dim X_{\red}$.
\end{defn}

\begin{example}[2]
  \mbox{}
  \begin{enumerate}
  \item 
  \end{enumerate}
\end{example}
>>>>>>> d388b9febcfa9ca4bcebc412bc8c9e28b5b54b70

>>>>>>> d388b9febcfa9ca4bcebc412bc8c9e28b5b54b70

\section{Faserprodukte von Schemata}

\underline{Ziel:} $X,Y$ $S$-Schemata


\section{Beispiele}



\section{Basiswechsel}

$\mathcal{C}$ belibige Kategorie


\section{Fasern von Morphismen}


\section{Eigenschaften von Schematamorphismen}


\section{Urbilder und Schema-theoretische-Durchschnitte}


\section{Affine/Projektive Räume über belibige Basen}


\section{Diagonale, Graph, und Kern in belibige Kategorien}


\section{Diagonal für Schemata}



\section{Seperite Morphismen}


\section{Eigentliche Morphismen}


\chapter{Dimensionen}
\label{chap:dim}

<<<<<<< HEAD
\section{Allgemine Schemata}

Hallo
=======
\section{Allgemeine Schemata}
\begin{defn}[1]
  Für einen topologischen Raum $X$ ist die (Krull-)Dimension das Supremum
  der Länge aller Ketten
  \[
    Z_{0}\subsetneq Z_{1}\subsetneq\cdots\subsetneq Z_{n}\subseteq X
  \]

  irreduzibler abgeschlossener Teilmengen $Z_{i}$. $X$ sei \textbf{von
    Dimension $n$}, falls alle irreduzible Komponenten von $X$ die Dimension
  $n$ haben ($\dim\emptyset=-\infty$, sonst $\dim X\in\mathbb{N}\cup\{+\infty\}$).

  Die Dimension eines Schemas ist per Definition die Dimension des unterliegenden
  topologischen Raums, also $\dim X=\dim X_{\red}$.
\end{defn}

\begin{example}[2]
  \mbox{}
  \begin{enumerate}
  \item 
  \end{enumerate}
\end{example}
>>>>>>> d388b9febcfa9ca4bcebc412bc8c9e28b5b54b70


\section{Ganze Morphismen}


\newpage{}
\printindex{}
\end{document}
