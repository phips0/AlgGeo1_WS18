\documentclass[12pt,a4paper]{book}
\usepackage[T1]{fontenc}
\usepackage[utf8]{inputenc}
\usepackage{geometry}
\geometry{verbose,tmargin=2cm,bmargin=2cm,lmargin=2cm,rmargin=2cm}
\pagestyle{headings}
\usepackage[ngerman]{babel}
\usepackage{verbatim}
\usepackage{amsmath}
\usepackage{amsthm}
\usepackage{amssymb}
\usepackage{stmaryrd}
\usepackage{makeidx}
\makeindex
\usepackage{setspace}
\usepackage[all]{xy}
\onehalfspacing
\usepackage[bookmarks=true]{hyperref}

%% Theorems (numbered by Part)
\newtheorem{thm}{Theorem}[chapter]
\theoremstyle{definition}
\newtheorem{example}[thm]{Beispiel}
\theoremstyle{definition}
\newtheorem{defn}[thm]{Definition}
\theoremstyle{plain}
\newtheorem{prop}[thm]{Satz}
\theoremstyle{plain}
\newtheorem{cor}[thm]{Korollar}
\theoremstyle{plain}
\newtheorem{lem}[thm]{Lemma}
\theoremstyle{remark}
\newtheorem{rem}[thm]{Bemerkung}
\theoremstyle{plain}

%% Theorems (unnumbered)
\newtheorem*{question*}{Frage}
\theoremstyle{remark}
\newtheorem*{claim*}{Behauptung}
\theoremstyle{definition}
\newtheorem*{notation*}{Notation}
\theoremstyle{definition}
\newtheorem*{example*}{Beispiel}
\theoremstyle{plain}
\newtheorem*{rem*}{Bemerkung}
\theoremstyle{remark}

%% User-specified commands
\DeclareMathOperator{\rad}{rad}
\DeclareMathOperator{\Spec}{Spec}
\DeclareMathOperator{\Quot}{Quot}
\DeclareMathOperator{\im}{\mathrm{im}}
\DeclareMathOperator{\Hom}{\mathrm{Hom}}
\DeclareMathOperator{\Mor}{\mathrm{Mor}}
\DeclareMathOperator{\id}{\mathrm{id}}

%%sets
\DeclareMathOperator{\CC}{\mathbb{C}}
\DeclareMathOperator{\RR}{\mathbb{R}}
\DeclareMathOperator{\QQ}{\mathbb{Q}}
\DeclareMathOperator{\ZZ}{\mathbb{Z}}
\DeclareMathOperator{\NN}{\mathbb{N}}

%% categories
\DeclareMathOperator{\ouv}{\mathcal{O}uv}
\DeclareMathOperator{\set}{\underline{Set}}
\DeclareMathOperator{\ab}{\underline{Ab}}
\DeclareMathOperator{\cattop}{\underline{Top}}
\DeclareMathOperator{\cring}{\underline{CRing}}
\DeclareMathOperator{\psh}{\underline{\mathcal{PS}h}}
\DeclareMathOperator{\sh}{\underline{\mathcal{S}h}}

%% commands
\newcommand{\cat}[1]{\mathcal{#1}}
\newcommand{\sheaf}[1]{\mathcal{#1}}
\renewcommand{\labelenumi}{(\roman{enumi})}
\renewcommand{\labelenumii}{\arabic{enumii}.}

\begin{document}

%% Title page
\title{Algebraische Geometrie I}
\author{Prof. Dr. Venjakob}
\maketitle

\tableofcontents{}
\newpage{}

\section*{Literatur}
\begin{itemize}
\item Görtz, Wedhorn. \emph{Algebraic Geometry I}
\item Hartshorne. \emph{Algebraic Geometry}
\item Shafarevich. \emph{Basic Algebraic Geometry 1 \& 2}
\item Grothendieck. \emph{Eléments de géometrie algébrique, EGA I-IV}
\end{itemize}

\paragraph{Kommutative Algebra}
\begin{itemize}
\item Brüske, Ischebeck, Vogel. \emph{Kommutative Algebra}
\item Kunz. \emph{Einführung in die kommutative Algebra und algebraische Geometrie}
\end{itemize}

\chapter{Prä-Varietäten}
\label{chap:prae-varietaeten}

\include{Chapter1/AlgGeo1-Chapter1-1_Einfuehrung}

\include{Chapter1/AlgGeo1-Chapter1-2_Die-Zariski-Topologie}

\include{Chapter1/AlgGeo1-Chapter1-3_Affine-algebraische-Mengen}

\include{Chapter1/AlgGeo1-Chapter1-4_Der-Hilbertsche-Nullstellensatz}

\include{Chapter1/AlgGeo1-Chapter1-5_Korrespondenz-zwischen-Radikalidealen}

\include{Chapter1/AlgGeo1-Chapter1-6_Irreduzible-topologische-Raeume}

\include{Chapter1/AlgGeo1-Chapter1-7_Irreduzible-algebraische-Mengen}

\include{Chapter1/AlgGeo1-Chapter1-8_Quasikompakte-und-noethersche-topologische-Raeume}

\include{Chapter1/AlgGeo1-Chapter1-9_Morphismen-von-affinen-algebraischen-Mengen}

\include{Chapter1/AlgGeo1-Chapter1-10_Unzulaenglichkeiten}

\include{Chapter1/AlgGeo1-Chapter1-11_Der-affine-Koordinatenring}

\include{Chapter1/AlgGeo1-Chapter1-12_Funktorielle-Eigenschaften-des-Koordinatenrings}

\include{Chapter1/AlgGeo1-Chapter1-13_Raeume-mit-Funktionen}

\include{Chapter1/AlgGeo1-Chapter1-14_Der-Raum-mit-Funktionen-zu-einer-affin-algebraischen-Menge}

\include{Chapter1/AlgGeo1-Chapter1-15_Funktorialitaet-der-Konstruktion}

\include{Chapter1/AlgGeo1-Chapter1-16_Definition-von-Praevarietaeten}

\include{Chapter1/AlgGeo1-Chapter1-17_Vergleich-mit-differenzierbaren-komplexen-Mannigfaltigkeiten}

\include{Chapter1/AlgGeo1-Chapter1-18_Topologische-Eigenschaften-von-Praevarietaeten}

\include{Chapter1/AlgGeo1-Chapter1-19_Offene-Untervarietaeten}

\include{Chapter1/AlgGeo1-Chapter1-20_Funktionenkoerper-einer-Praevarietaet}

\include{Chapter1/AlgGeo1-Chapter1-21_Abgeschlossene-Unterpraevarietaeten}

\include{Chapter1/AlgGeo1-Chapter1-22_Homogene-Polynome}

\include{Chapter1/AlgGeo1-Chapter1-23_Definition-des-projektiven-Raumes}

\include{Chapter1/AlgGeo1-Chapter1-24_Projektive-Varietaeten}

\include{Chapter1/AlgGeo1-Chapter1-25_Koordinatenwechsel-im-projektiven-Raum}

\include{Chapter1/AlgGeo1-Chapter1-26_Lineare-Unterraeume-des-projektiven-Raums}

\include{Chapter1/AlgGeo1-Chapter1-27_Kegel}

\include{Chapter1/AlgGeo1-Chapter1-28_Quadriken}

\chapter{Das Ringspektrum}
\label{chap:das-ringspektrum}

\include{Chapter2/AlgGeo1-Chapter2-1_Definition-von-Spec-A}

\include{Chapter2/AlgGeo1-Chapter2-2_Topologische-Eigenschaften-von-Spec-A}

\include{Chapter2/AlgGeo1-Chapter2-3_Der-Funktor-A-Spec-A}

\include{Chapter2/AlgGeo1-Chapter2-4_Beispiele}

\include{Chapter2/AlgGeo1-Chapter2-5_Garben}

\include{Chapter2/AlgGeo1-Chapter2-6_Halme-von-Garben}

\include{Chapter2/AlgGeo1-Chapter2-7_Die-zu-einer-Praegarbe-assoziierte-Garbe}

%% TODO:
%% - Limiten einfügen in Garben
\section{Direktes und Inverses Bild von Garben}


\section{Lokal geringte Räume}


\section{Die Strukturgrabe auf $\Spec(A)$}


\section{Der Funktor $A \protect\mapsto(\Spec(A),\cat{O}_{\Spec(A)})$}


\section{Beispiele}


\section{Direktes und inverses Bild von Garben}
\label{sec:garben-direktes-inverses-bild}

Sei $f:X\rightarrow Y$ stetige Abbildung topologischer Räume, $\mathcal{F}$
eine Prägarbe auf $X$. Ziel: $f_{\ast}\mathcal{F}$ Prägarbe auf
$Y$, das direkte Bild von $\mathcal{F}$ unter $f$. Definiere $(f_{\ast}\mathcal{F})(V):=\mathcal{F}(f^{-1}(V))$
mit Restriktionsabbildung von $\mathcal{F}$ ($V_{1}\subseteq V_{2}:$
$s\in f_{\ast}\mathcal{F}(V_{2})\rightarrow s|_{V_{1}}=\mathcal{F}res_{f^{-1}(V_{1})}^{f^{-1}(V_{2})}$).
\begin{align*}
  f_{\ast}:PSh(X) & \longrightarrow PSh(Y)\\
  \mathcal{F} & \longmapsto f_{\ast}\mathcal{F}\\
  \mathcal{F}\overset{\varphi}{\rightarrow}\mathcal{G} & \longmapsto f_{\ast}(U):f_{\ast}\mathcal{F}\rightarrow f_{\ast}\mathcal{G}
\end{align*}

ist Funktor via $(f_{\ast}\varphi)_{V}=\varphi_{f^{-1}(V)}$.
\begin{rem}[28]
  \mbox{}
  \begin{enumerate}
  \item $\mathcal{F}$ Garbe auf $X$ $\Longrightarrow f_{\ast}\mathcal{F}$
    Garbe auf $X$, d.h. $f_{\ast}:Sh(X)\rightarrow Sh(Y)$.
  \item Ist $g:Y\rightarrow Z$ eine weitere stetige Abbildung topologischer
    Räume, so existiert ein offensichtlicher Isomorphismus $g_{\ast}\circ(f_{\ast}\mathcal{F})=(g\circ f)_{\ast}\mathcal{F}$,
    funktoriell in $\mathcal{F}$.
  \end{enumerate}
  \medskip{}
\end{rem}

Nun sei $\mathcal{G}$ eine Prägarbe auf $Y$.

\textbf{Ziel:} Definiere $f^{+}\mathcal{G}$ Prägarbe auf $X$. $f^{-1}\mathcal{G}=\widetilde{f^{+}\mathcal{G}}$
Garbe auf $X$, \textbf{Inverses Bild zu $\mathcal{G}$ unter $f$}
via 
\[
(f^{+}\mathcal{G})(U):=\underset{\underset{Y\supseteq V\supseteq f(U)}{\longrightarrow}}{\lim}\mathcal{G}(V)
\]

mit induzierte Restriktionsabbildung.\medskip{}

\textbf{Warnung:} $\mathcal{G}$ Garbe auf $Y$ $\leadsto f^{+}\mathcal{G}$
im Allgemeinen keine Garbe auf $X$. Falls $f:X\hookrightarrow Y$
Inklusion, $\mathcal{G}|_{X}:=f^{-1}\mathcal{G}$. Ist $X\subseteq Y$
offen stimmt $\mathcal{G}|_{X}$ mit der Einschränkung aus Beispiel
19 überein (cofinales Objekt). $\leadsto f^{-1}:PSh(Y)\rightarrow Sh(X)$
Funktor.

$g:Y\xrightarrow{\text{stetig}}Z$, $\mathcal{H}$ Prägarbe auf $Z$,
$U\subseteq X$ offen. 
\[
Z\underset{\text{offen}}{\supseteq}W\supseteq g(f(U))\Longleftrightarrow W\supseteq g(V)
\]

für ein $f(U)\subseteq V\subseteq Y$ offen. 
\begin{align*}
  \underset{\longrightarrow}{\lim}\underset{\longrightarrow}{\lim}=\underset{\longrightarrow}{\lim}\Longrightarrow f^{+}(g^{+}\mathcal{H}) & =(g\circ f)^{+}\mathcal{H}\quad(*)\\
  \Longrightarrow f^{-1}(g^{-1}\mathcal{H}) & =(g\circ f)^{-1}\mathcal{H}
\end{align*}

\begin{example}
  $\imath:\{x\}\rightarrow X$ Inklusion, $\mathcal{F}$ Prägarbe auf
  $X$. $\Longrightarrow\imath^{-1}(\mathcal{F})=\mathcal{F}_{x}$ per
  Definition. $(\ast)\Longrightarrow$
  \[
  \begin{array}{ccc}
    (f^{-1}\mathcal{G})_{x} & = & \mathcal{G}_{f(x)}\\
    \shortparallel &  & \shortparallel\\
    \imath^{-1}\circ(f^{-1}\mathcal{G}) & = & (f\circ\imath)^{-1}\mathcal{G}
  \end{array}
  \]
\end{example}

\begin{prop}[29]
  Für $f:X\rightarrow Y$ stetig sind die Funktionen $f_{\ast}$ und
  $f^{-1}$ zueinander adjungiert, d.h. für $\mathcal{F}$ Garbe auf
  $X$, $\mathcal{G}$ Prägarbe auf $Y$ existiert eine bijektion
  \begin{align*}
    \hom_{Sh(x)}(f^{-1}\mathcal{G},\mathcal{F}) & \longleftrightarrow\hom_{Psh(Y)}(\mathcal{G},f_{\ast}\mathcal{F})\\
    \varphi & \longmapsto\varphi^{\flat}\\
    \psi^{\sharp} & \longmapsfrom\psi
  \end{align*}

  funktoriell in $\mathcal{F}$ und $\mathcal{G}$.
\end{prop}

\begin{proof}
  $\varphi:f^{-1}\mathcal{G}\rightarrow\mathcal{F}$ Morphismus von
  Garben auf $X$. $t\in\mathcal{G}(V)$, $V\subseteq Y$ offen
  \begin{align*}
    \mathcal{G}(V) & \rightarrow f^{+}\mathcal{G}(f^{-1}(V))\xrightarrow{\imath_{f^{+}\mathcal{G}}}f^{-1}\mathcal{G}(f^{-1}(V))\xrightarrow{\varphi_{f^{-1}(V)}}\mathcal{F}(f^{-1}(V))=f_{\ast}\mathcal{F}(V)\\
    & \phantom{\rightarrow\ }\shortparallel\underset{\underset{Y\supseteq W\supseteq ff^{-1}(V)\subseteq V}{\longrightarrow}}{\lim}\\
    t & \mapsto\varphi_{V}^{\flat}(t)
  \end{align*}

  Definition von $\psi^{\#}$. $\mathcal{G}\xrightarrow{\psi}f_{\ast}\mathcal{F}$
  Morphismus von Prägarben auf . Wir definieren $\psi^{\#}:f^{+}\mathcal{G}\rightarrow\mathcal{F}$,
  welches dann $\psi^{\#}:f^{-1}\mathcal{G}\rightarrow\mathcal{F}$
  induziert. $U\subseteq X$ offen, $S\subseteq f^{+}\mathcal{G}(U)$,
  $s=[(V,s_{V})]$, $V\supseteq f(U)$, $s_{V}\in\mathcal{G}(V)$. $\Longrightarrow f^{-1}(V)\supseteq U$.
  \[
  \xymatrix{\psi_{V}(s_{V})\in f_{\ast}\mathcal{F}(V)\ar@{=}[r]\ar@{|->}[rd] & \mathcal{F}(f^{-1}(V))\ar[d]\\
    & \psi_{U}^{\#}(s)\in\mathcal{F}(U)
  }
  \]

  Überprüfe $\varphi^{\flat^{\#}}=\varphi$, $\psi^{\#^{\flat}}=\mathcal{H}$
  und Funktoriell.
\end{proof}
Definition + Proposition 29 verallgemeinern sich zu (Prä)Garben von
Ringen, $R$-Moduln, $R$-Algebren.

\textbf{Beschreibung} von:
\[
\mathcal{G}_{f(x)}=(f^{-1}\mathcal{G})_{x}\overset{\varphi_{x}}{\longmapsto}\mathcal{F}_{x},\ x\in X
\]

\[
\xymatrix{f(x)\in U\underset{\text{offen}}{\subseteq}Y & \mathcal{G}(U)\ar[r]^{\varphi_{U}^{\flat}}\ar@{-->}[d] & \mathcal{F}(f^{-1}(U))\ar[r] & \mathcal{F}_{x}\\
  \underset{\underset{U}{\longrightarrow}}{\lim} & \mathcal{G}_{f(x)}\ar@{-->}[rru]
}
\]


\section{Lokal geringte Räume}
\label{sec:lokal-geringte-raeume}
\begin{defn}
  Ein geringter Raum ist ein Paar $(X,\mathcal{O}_{X})$ bestehend aus
  einem topologischen Raum $X$ und einer Garbe $\mathcal{O}_{X}$ (kommutativer)
  Ringe. Ein Morphismus $(X,\mathcal{O}_{X})\rightarrow(Y,\mathcal{O}_{Y})$
  geringter Räume ist wiederum ein Paar $(f,f^{\flat})$ bestehend aus
  einer stetigen Abbildung $f:X\rightarrow Y$ und einem Homomorphismus
  $f^{\flat}:\mathcal{O}_{Y}\rightarrow f_{\ast}\mathcal{O}_{X}$ von
  Ringgarben auf $Y$. Dieses Datum ist gleichbedeutend (Proposition
  29) mit $(f,f^{\sharp})$, wobei nun $f^{\sharp}:f^{-1}\mathcal{O}_{Y}\rightarrow\mathcal{O}_{X}$
  ein Garbenhomomorphismus auf $X$ ist.

  Bezeichne: $f$ oder $(f,f^{\flat})$ oder $(f,f^{\sharp})$. Damit
  haben wir eine \textbf{Kategorie der geringten Räume}. $\mathcal{O}_{X}$
  heißt Strukturgarbe\index{Strukturgarbe} von $X$, oft schreiben
  wir $X$ für $\mathcal{O}_{X}$.

  Idee: $\mathcal{O}_{X}$ beschreibt die zulässigen Funktionen auf
  $U\subset X$, d.h. etwa stetige, differenzierbare, holomorphe, rigid
  analytische usw. Funktionen. Solche Funktionen auf $V\subset Y$ sollen
  beim ``Zurückziehen'' unter $f$ in dieselbe Klasse überführt werden.
  Dies wird formal durch das Datum $f^{\flat}$ sichergestellt.
\end{defn}

\begin{notation*}
  Wenn $A$ ein lokaler Ring ist, $\mathfrak{m}_{A}$ das maximale Ideal,
  und $\kappa(A)=A/\mathfrak{m}_{A}$ Restklassenkörper. Ein Homomorphismus
  $\varphi:A\rightarrow B$ lokaler Ringe heißt \textbf{lokal}, falls
  $\varphi(\mathfrak{m}_{A})\subset\mathfrak{m}_{B}$. $(f,f^{\flat})=(f,f^{\sharp})=X\rightarrow Y$
  Morphismus geringter Räume induziert:
  \begin{align*}
    & \mathcal{O}_{Y,f(x)}=(f^{-1}\mathcal{O}_{Y})_{x}\xrightarrow{f_{x}^{\sharp}}\mathcal{O}_{X,x}\\
    \text{oder } & \xymatrix{\mathcal{O}_{Y}(U)\ar[r]^{f_{U}^{\flat}}\ar[d] & \mathcal{O}_{X}(f^{-1}(U))\ar[d] & f(x)\in U\subset Y\text{ offen}\\
      \mathcal{O}_{Y,f(x)}=\lim\mathcal{O}_{Y}(U)\ar@{-->}[r] & \mathcal{O}_{X,x}
    }
  \end{align*}
\end{notation*}
\begin{defn}[orig. 31]
  Ein lokal geringter Raum ist ein geringter Raum $(X,\mathcal{O}_{X})$,
  für der $\mathcal{O}_{X,x}$ für alle $x\in X$ ein \emph{lokaler
    Ring} ist. Ein Morphismus $(X,\mathcal{O}_{X})\rightarrow(Y,\mathcal{O}_{Y})$
  lokal geringter Räume ist ein Morphismus geringter Räume $(f,f^{\flat})$,
  so dass die induzierte Abbildung 
  \[
  f_{x}^{\sharp}:\mathcal{O}_{Y,f(x)}\rightarrow\mathcal{O}_{X,x}
  \]

  ein lokaler Ringhomomorphismus ist für alle $x\in X$. Dies führt
  zu einer Unterkategorie der Kategorie geringter Räume, die im Allgemeinen
  \emph{nicht }voll ist, d.h. es gibt Morphismen $f$ geringter Räume
  zwischen lokal geringten Räume, die nicht lokal sind!
\end{defn}

Bezeichne: 
\begin{itemize}
\item $\mathcal{O}_{X,x}$ der ``lokale Ring von $X$ in $x$'';
\item $\mathfrak{m}_{x}$ maximales Ideal;
\item $\kappa(x):=\mathcal{O}_{X,x}/\mathfrak{m}_{x}$ Restklassenkörper
  (bei $x$). \texttt{
  \[
  \xymatrix@R=0pt{\mathcal{O}_{X}(U)\ar[r] & \mathcal{O}_{X,x}\ar[r] & \kappa(x)\\
    f\ar[rr] &  & f(x)
  }
  \]
}
\end{itemize}
Warum \textbf{lokal} geringte Räume? Heuristik:
\begin{align*}
  \mathcal{O}_{X}(U) & \leftrightarrow\text{Funktionen auf }U\\
  \mathcal{O}_{X,x} & \leftrightarrow\text{Funktionen auf Umgebung }U\text{ von }x
\end{align*}

\emph{Wunsch}: $f(x)\neq0\overset{!}{\Rightarrow}f$ ist invertierbar
auf einer kleinen Umgebung $V$ von $x$, d.h. 
\[
\mathcal{O}_{X,x}\backslash\underbrace{\{f\mid f(x)=0\}}_{=\mathfrak{m}_{x}}\subset\mathcal{O}_{X,x}^{\times},
\]

also $\mathcal{O}_{X,x}$ lokal. \emph{Ferner}: $g\mathcal{O}_{Y,f(x)}$
mit $g(f(x))=0$ sollte implizieren: $(g\circ f)(x)=0$. Übersetzt:
\[
f_{x}^{\sharp}(\mathfrak{m}_{f(x)})\subset\mathfrak{m}_{x},\quad f_{x}^{\sharp}(g)="g\circ f"
\]

\begin{example}[orig. 32]
  $\varphi_{X}$ Garbe der $\mathbb{R}$-wertiger stetiger Funktionen
  auf einem topologischen Raum $X$. $\varphi_{X,x}$ Ring der Keime
  $[s]$ stetiger Funktionen in einer Umgebung von $X$:
  \[
  \mathfrak{m}_{x}=\{[s]\in\varphi_{X,x}\mid0=s(x)\}
  \]

  ist einziges maximale Ideal, d.h. $(X,\varphi_{X})$ ist lokal geringter
  Raum. 

  \emph{Denn}: Sei $[s]\in\varphi_{X,x}\backslash\mathfrak{m}_{x}$
  gegeben.

  $\Rightarrow s(x)\neq0$ für alle $s\in[s]$. 

  $\Rightarrow(s$ stetig) $\exists x\in U\subset X$ offen mit $s(u)\neq0$
  für alle $u\in U$.

  $\Rightarrow\frac{1}{s|_{U}}\in\varphi_{X}(U)$ existiert.

  $\Rightarrow\varphi_{X,x}\backslash\mathfrak{m}_{x}=\varphi_{X,x}^{\times}$
  Einheitengruppe. Es ist: 
  \[
  \varphi_{X,x}\rightarrow\mathbb{R},\ [s]\mapsto s(x)
  \]

  ein surjektiver Ringhomomorphismus mit $\ker=\mathfrak{m}_{x}$.

  $\Rightarrow\kappa(x)\cong\mathbb{R}$. Sei $f:X\rightarrow Y$ stetig,
  $V\subset Y$ offen.
  \begin{align*}
    f_{x}^{\flat}:\varphi_{Y}(V) & \longrightarrow\varphi_{X}(f^{-1}(V))=f_{\ast}\varphi_{X}(V)\\
    t & \longmapsto t\circ f
  \end{align*}

  Es folgt:
  \begin{align*}
    \varphi_{Y,f(x)} & \longrightarrow\varphi_{X,x}\\{}
           [t] & \longmapsto[t\circ f]
  \end{align*}

  ist ein Morphismus lokal geringter Räume. Ebenso lassen sich Prävarietäten
  über lokal geringte Räume interpretieren!
\end{example}


\chapter*{Das Ringsprektrum als lokal geringter Raum}
\label{chap:ringspektrum-lokal-geringter-raum}
Ziel: volltreuer Funktor
\begin{align*}
  \text{Ringe} & \longrightarrow\text{Kategorie lokal geringter Räume}\\
  A & \longmapsto(\Spec A,\mathcal{O}_{\Spec A})
\end{align*}

\section{Die Strukturgarbe auf Spec A}
\label{sec:strukturgarbe-auf-spec-A}

Sei $X:=\Spec(A)$, $\mathcal{B}=\{D(f)\mid f\in A\}$ Basis der Topologie.

\textbf{Vorgegeben:} Definiere Prägarbe $\mathcal{O}_{X}$ auf $\mathcal{B}$,
die Garbenaxiome bzgl. $\mathcal{B}$ erfüllt.

\textbf{Wähle:} $\mathcal{O}_{X}(X)=A$ (vgl. Prävarietäten) bzw.
$\mathcal{O}_{X}(D(f))=A_{f}$, da
\begin{align*}
  \imath_{j}:A & \longrightarrow A_{f}\\
  a & \longmapsto\frac{a}{1}
\end{align*}

einen Homöomorphismus $D(f)\xrightarrow{\sim}\Spec(A_{f})$ induziert.
(``Funktionen mit möglichen Polen in $V(f)$).

\subsection{Wohldefiniertheit}
\label{subsec:strukturgarbe-wohldefiniertheit}

$D(f)=D(g)\Rightarrow A_{f}=A_{g}$ kanonisch. Dazu: 
\begin{align*}
  D(f)\subset D(g) & \Leftrightarrow\exists n\geq1\text{ d.d. }f^{n}\in A_{g}
\end{align*}


\subsection{Induzierte Abbildung}
\label{subsec:strukturgarbe-induzierte-abbildung}

\[
\mathcal{O}_{X}(D(g))\rightarrow\mathcal{O}_{X}(D(f)),\ \rho_{f,g}=:\text{res}_{D(f)}^{D(g)}
\]

Dies definiert eine Prägarbe auf $\mathcal{B}$.
\begin{thm}[orig. 33]
  Die Prägarbe $\mathcal{O}_{X}$ ist eine Garbe auf $\mathcal{B}$.
  Die induzierte Garbe auf $X$ (Proposition 20) werde auch mit $\mathcal{O}_{X}$
  bezeichnet. Da
  \[
  \mathcal{O}_{X,x}:=\underset{\underset{D(f)\ni x}{\longrightarrow}}{\lim}\mathcal{O}_{X}(D(f))=\underset{\underset{f\in\mathfrak{p}_{x}}{\longrightarrow}}{\lim}A_{f}=A_{\mathfrak{p}_{x}}
  \]
  mit $(X,\mathcal{O}_{X})=(\Spec A,\mathcal{O}_{\Spec A})$ (kurz $\Spec A$)
  ein lokal geringter Raum.
\end{thm}

\begin{proof}
  Sei $D(f)=\bigcup_{i\in I}D(f_{i})$ Überdeckung in $\mathcal{B}$.
  Zu zeigen:
  \begin{enumerate}
  \item $s\in\mathcal{O}_{X}(D(f))$ mit $s|_{D(f_{i})}=0$, $i\in I$.

    $\overset{!}{\Rightarrow}s=0$.
  \item $s_{i}\in\mathcal{O}_{X}(D(f_{i}))$, $i\in I$, mit $s_{i}|_{D(f_{i})\cap D(f_{j})}=s_{j}|_{D(f_{i})\cap D(f_{j})}$
    $\forall i,j\in I$.

    $\overset{!}{\Rightarrow}\exists s\in\mathcal{O}_{X}(D(f))$ mit $s|_{D(f_{i})}=s_{i}$
    $\forall i\in I$.
  \end{enumerate}
  Ohne Einschränkung:
  \begin{itemize}
  \item $I$ endlich, da $D(f)$ quasi-kompakt.
  \item $f=1$, $D(f)=X$ (mit $(A_{f},\mathcal{O}_{X}|_{D(f)})$ statt $(A,\mathcal{O}_{X})$
    betrachtet) 
    \[
    X=\bigcup_{i\in I}D(f_{i})\Leftrightarrow(f_{i}\mid i\in I)=A
    \]
    Es folgt: $b_{i}=b_{i}(n)\in A$ d.d. $\sum_{i\in I}b_{i}f_{i}^{n}=1$
    \textbf{Zerlegung der 1}. (z)
  \item[Zu 1.] Sei $s=a\in A$ d.d. $0=\frac{a}{1}\in A_{f}$, $\forall i\in I$.
    $I$ endlich, also $\exists n\geq1$ d.d. $f_{i}^{n}a=0$ $\forall i\in I$.
    Mit $(z)$ folgt
    \[
    a=\left(\sum_{i\in I}b_{i}f_{i}^{n}\right)a=0
    \]
  \item[Zu 2.] $s_{i}=\frac{a_{i}}{f_{i}^{n}}$ für $n$ geeignet, unabhängig von
    $i\in I$ (endlich). Nach Voraussetzung:
    \[
    \frac{a_{i}}{f_{i}^{n}}=\frac{a_{j}}{f_{j}^{n}}\in A_{f_{i}f_{j}},\quad D(f_{i})\cap D(f_{j})=D(f_{i}f_{j})
    \]
    Es folgt: $\exists m\geq1$ d.d. $(f_{i}f_{j})^{m}(f_{j}^{n}a_{i}-f_{i}^{n}a_{j})=0$
    $\forall i,j$.
    \begin{align*}
      \frac{a_{i}}{f_{i}^{n}} & =\frac{f_{i}^{m}a_{i}}{f_{i}^{n+m}}=:\frac{a_{i}'}{f_{i}^{n'}},\quad n'=n+m
    \end{align*}
    Ohne Einschränkung: $f_{j}^{n}a_{i}=f_{i}^{n}a_{j}$ $\forall i,j\in I$,
    ({*}) denn:
    \begin{align*}
      f_{j}^{m+n}f_{i}a_{i} & =f_{i}^{m+n}f_{j}^{m}a_{j}\\
      f_{j}^{n'}a_{i}' & =f_{i}^{n'}a_{j}'
    \end{align*}
    Setze $s:=\sum_{j\in I}b_{j}a_{j}\in A$ ($(z)$). Es folgt:
    \[
    f_{i}^{n}s=f_{i}^{n}\sum b_{i}a_{j}=\sum b_{j}(f_{i}^{n}a_{j})\overset{(*)}{=}\left(\sum b_{i}f_{i}^{n}\right)a_{i}\overset{(z)}{=}a_{i}
    \]
    also $\frac{s}{1}=\frac{a_{i}}{f^{n}}=s_{i}$.
  \end{itemize}
\end{proof}

\end{document}


\chapter{Schemata}
\label{chap:schemata}

\section{Schemata}

\begin{defn}
	Ein Schemata ist ein lokal geringter Raum $(X,\cat{O}_X)$, der eine offene Überdeckung $(U_i)_{i\in I}$
	besitz derart alle lokal geringter Räume $(U_i,\cat{O}_{X|U_i})$ affine Schemata sind.\\
	Für ein Schemata $S$ besitz $\textbf{Sch}/_{S}$ ~~ \textbf{Kategorie der Schemata über} $S$ oder $S$-\textbf{Schemata}\\
	\begin{itemize}
		\item Morphismen $X \rightarrow S$ von Schemata
	\end{itemize}
\end{defn}


\section{offene Unterschemata}


\section{Morphismen in affine Schemata}


\section{Morphismen der Form $\Spec(K)\protect\longrightarrow X$}


\section{Verkleben von Schemata und disjunkte Vereinigung}


\section{Der projektive Raum als Schemata}


\include{Chapter3/AlgGeo1-Chapter3-7_nullstellenmenge-in-projektiven-raeme}

\include{Chapter3/AlgGeo1-Chapter3-8_topologische-eigenschaften}

\section{Noethische Schemata}


\section{Generische Punkte}


\section{Reduzierte und ganze Schemata}


\section{Schemata von endlichem Type über $k$}


\section{Prävaritäten als Schemata}


\chapter*{Unterschemata und Immersion (Einbettung)}
\section{offene/abgeschlossen Einbettung}


\section{Reduzierte Unterschemata}



\chapter{Faserprodukte}
\label{chap:faserprod}

\include{Chapter4/AlgGeo1-Chapter4-1-Der-Punkte-Fuktor}

\include{Chapter4/AlGGeo1-Chapter4-2-Yoneda-Lemma}

\section{Faserprodukt in beliebiger Kategorie}

$\cat{C}$ Kategorie, $S \in \text{Oj}(\cat{C})$


\newpage{}
\printindex{}
\end{document}
