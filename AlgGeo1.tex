\documentclass[12pt,a4paper]{book}
\usepackage[T1]{fontenc}
\usepackage[utf8]{inputenc}
\usepackage{geometry}
\geometry{verbose,tmargin=2cm,bmargin=2cm,lmargin=2cm,rmargin=2cm}
\pagestyle{headings}
\usepackage[ngerman]{babel}
\usepackage{verbatim}
\usepackage{amsmath}
\usepackage{amsthm}
\usepackage{amssymb}
\usepackage{stmaryrd}
\usepackage{makeidx}
\makeindex
\usepackage{setspace}
\usepackage[all]{xy}
\onehalfspacing
\usepackage[bookmarks=true]{hyperref}

%% Theorems (numbered by Part)
\newtheorem{thm}{Theorem}[chapter]
\theoremstyle{definition}
\newtheorem{example}[thm]{Beispiel}
\theoremstyle{definition}
\newtheorem{defn}[thm]{Definition}
\theoremstyle{plain}
\newtheorem{prop}[thm]{Satz}
\theoremstyle{plain}
\newtheorem{cor}[thm]{Korollar}
\theoremstyle{plain}
\newtheorem{lem}[thm]{Lemma}
\theoremstyle{remark}
\newtheorem{rem}[thm]{Bemerkung}
\theoremstyle{plain}

%% Theorems (unnumbered)
\newtheorem*{question*}{Frage}
\theoremstyle{remark}
\newtheorem*{claim*}{Behauptung}
\theoremstyle{definition}
\newtheorem*{notation*}{Notation}
\theoremstyle{definition}
\newtheorem*{example*}{Beispiel}
\theoremstyle{plain}
\newtheorem*{rem*}{Bemerkung}
\theoremstyle{remark}

%% User-specified commands
\DeclareMathOperator{\rad}{rad}
\DeclareMathOperator{\Spec}{Spec}
\DeclareMathOperator{\Quot}{Quot}
\DeclareMathOperator{\im}{\mathrm{im}}
\DeclareMathOperator{\Hom}{\mathrm{Hom}}
\DeclareMathOperator{\Mor}{\mathrm{Mor}}
\DeclareMathOperator{\id}{\mathrm{id}}

%%sets
\DeclareMathOperator{\CC}{\mathbb{C}}
\DeclareMathOperator{\RR}{\mathbb{R}}
\DeclareMathOperator{\QQ}{\mathbb{Q}}
\DeclareMathOperator{\ZZ}{\mathbb{Z}}
\DeclareMathOperator{\NN}{\mathbb{N}}

%% categories
\DeclareMathOperator{\ouv}{\mathcal{O}uv}
\DeclareMathOperator{\set}{\underline{Set}}
\DeclareMathOperator{\ab}{\underline{Ab}}
\DeclareMathOperator{\cattop}{\underline{Top}}
\DeclareMathOperator{\cring}{\underline{CRing}}
\DeclareMathOperator{\psh}{\underline{\mathcal{PS}h}}
\DeclareMathOperator{\sh}{\underline{\mathcal{S}h}}

%% commands
\newcommand{\cat}[1]{\mathcal{#1}}
\newcommand{\sheaf}[1]{\mathcal{#1}}
\renewcommand{\labelenumi}{(\roman{enumi})}
\renewcommand{\labelenumii}{\arabic{enumii}.}

\begin{document}

%% Title page
\title{Algebraische Geometrie I}
\author{Prof. Dr. Venjakob}
\maketitle

\tableofcontents{}
\newpage{}

\section*{Literatur}
\begin{itemize}
\item Görtz, Wedhorn. \emph{Algebraic Geometry I}
\item Hartshorne. \emph{Algebraic Geometry}
\item Shafarevich. \emph{Basic Algebraic Geometry 1 \& 2}
\item Grothendieck. \emph{Eléments de géometrie algébrique, EGA I-IV}
\end{itemize}

\paragraph{Kommutative Algebra}
\begin{itemize}
\item Brüske, Ischebeck, Vogel. \emph{Kommutative Algebra}
\item Kunz. \emph{Einführung in die kommutative Algebra und algebraische Geometrie}
\end{itemize}

\chapter{Prä-Varietäten}
\label{chap:prae-varietaeten}

\include{Chapter1/AlgGeo1-Chapter1-1_Einfuehrung}

\include{Chapter1/AlgGeo1-Chapter1-2_Die-Zariski-Topologie}

\include{Chapter1/AlgGeo1-Chapter1-3_Affine-algebraische-Mengen}

\include{Chapter1/AlgGeo1-Chapter1-4_Der-Hilbertsche-Nullstellensatz}

\include{Chapter1/AlgGeo1-Chapter1-5_Korrespondenz-zwischen-Radikalidealen}

\include{Chapter1/AlgGeo1-Chapter1-6_Irreduzible-topologische-Raeume}

\include{Chapter1/AlgGeo1-Chapter1-7_Irreduzible-algebraische-Mengen}

\include{Chapter1/AlgGeo1-Chapter1-8_Quasikompakte-und-noethersche-topologische-Raeume}

\include{Chapter1/AlgGeo1-Chapter1-9_Morphismen-von-affinen-algebraischen-Mengen}

\include{Chapter1/AlgGeo1-Chapter1-10_Unzulaenglichkeiten}

\include{Chapter1/AlgGeo1-Chapter1-11_Der-affine-Koordinatenring}

\include{Chapter1/AlgGeo1-Chapter1-12_Funktorielle-Eigenschaften-des-Koordinatenrings}

\include{Chapter1/AlgGeo1-Chapter1-13_Raeume-mit-Funktionen}

\include{Chapter1/AlgGeo1-Chapter1-14_Der-Raum-mit-Funktionen-zu-einer-affin-algebraischen-Menge}

\include{Chapter1/AlgGeo1-Chapter1-15_Funktorialitaet-der-Konstruktion}

\include{Chapter1/AlgGeo1-Chapter1-16_Definition-von-Praevarietaeten}

\include{Chapter1/AlgGeo1-Chapter1-17_Vergleich-mit-differenzierbaren-komplexen-Mannigfaltigkeiten}

\include{Chapter1/AlgGeo1-Chapter1-18_Topologische-Eigenschaften-von-Praevarietaeten}

\include{Chapter1/AlgGeo1-Chapter1-19_Offene-Untervarietaeten}

\include{Chapter1/AlgGeo1-Chapter1-20_Funktionenkoerper-einer-Praevarietaet}

\include{Chapter1/AlgGeo1-Chapter1-21_Abgeschlossene-Unterpraevarietaeten}

\include{Chapter1/AlgGeo1-Chapter1-22_Homogene-Polynome}

\include{Chapter1/AlgGeo1-Chapter1-23_Definition-des-projektiven-Raumes}

\include{Chapter1/AlgGeo1-Chapter1-24_Projektive-Varietaeten}

\include{Chapter1/AlgGeo1-Chapter1-25_Koordinatenwechsel-im-projektiven-Raum}

\include{Chapter1/AlgGeo1-Chapter1-26_Lineare-Unterraeume-des-projektiven-Raums}

\include{Chapter1/AlgGeo1-Chapter1-27_Kegel}

\include{Chapter1/AlgGeo1-Chapter1-28_Quadriken}

\chapter{Das Ringspektrum}
\label{chap:das-ringspektrum}

\include{Chapter2/AlgGeo1-Chapter2-1_Definition-von-Spec-A}

\include{Chapter2/AlgGeo1-Chapter2-2_Topologische-Eigenschaften-von-Spec-A}

\include{Chapter2/AlgGeo1-Chapter2-3_Der-Funktor-A-Spec-A}

\include{Chapter2/AlgGeo1-Chapter2-4_Beispiele}

\include{Chapter2/AlgGeo1-Chapter2-5_Garben}

\include{Chapter2/AlgGeo1-Chapter2-6_Halme-von-Garben}

\include{Chapter2/AlgGeo1-Chapter2-7_Die-zu-einer-Praegarbe-assoziierte-Garbe}

\section{Direktes und inverses Bild von Garben}
\label{sec:garben-direktes-inverses-bild}

Sei $f:X\rightarrow Y$ stetige Abbildung topologischer Räume, $\mathcal{F}$
eine Prägarbe auf $X$. Ziel: $f_{\ast}\mathcal{F}$ Prägarbe auf
$Y$, das direkte Bild von $\mathcal{F}$ unter $f$. Definiere $(f_{\ast}\mathcal{F})(V):=\mathcal{F}(f^{-1}(V))$
mit Restriktionsabbildung von $\mathcal{F}$ ($V_{1}\subseteq V_{2}:$
$s\in f_{\ast}\mathcal{F}(V_{2})\rightarrow s|_{V_{1}}=\mathcal{F}res_{f^{-1}(V_{1})}^{f^{-1}(V_{2})}$).
\begin{align*}
  f_{\ast}:PSh(X) & \longrightarrow PSh(Y)\\
  \mathcal{F} & \longmapsto f_{\ast}\mathcal{F}\\
  \mathcal{F}\overset{\varphi}{\rightarrow}\mathcal{G} & \longmapsto f_{\ast}(U):f_{\ast}\mathcal{F}\rightarrow f_{\ast}\mathcal{G}
\end{align*}

ist Funktor via $(f_{\ast}\varphi)_{V}=\varphi_{f^{-1}(V)}$.
\begin{rem}[28]
  \mbox{}
  \begin{enumerate}
  \item $\mathcal{F}$ Garbe auf $X$ $\Longrightarrow f_{\ast}\mathcal{F}$
    Garbe auf $X$, d.h. $f_{\ast}:Sh(X)\rightarrow Sh(Y)$.
  \item Ist $g:Y\rightarrow Z$ eine weitere stetige Abbildung topologischer
    Räume, so existiert ein offensichtlicher Isomorphismus $g_{\ast}\circ(f_{\ast}\mathcal{F})=(g\circ f)_{\ast}\mathcal{F}$,
    funktoriell in $\mathcal{F}$.
  \end{enumerate}
  \medskip{}
\end{rem}

Nun sei $\mathcal{G}$ eine Prägarbe auf $Y$.

\textbf{Ziel:} Definiere $f^{+}\mathcal{G}$ Prägarbe auf $X$. $f^{-1}\mathcal{G}=\widetilde{f^{+}\mathcal{G}}$
Garbe auf $X$, \textbf{Inverses Bild zu $\mathcal{G}$ unter $f$}
via 
\[
(f^{+}\mathcal{G})(U):=\underset{\underset{Y\supseteq V\supseteq f(U)}{\longrightarrow}}{\lim}\mathcal{G}(V)
\]

mit induzierte Restriktionsabbildung.\medskip{}

\textbf{Warnung:} $\mathcal{G}$ Garbe auf $Y$ $\leadsto f^{+}\mathcal{G}$
im Allgemeinen keine Garbe auf $X$. Falls $f:X\hookrightarrow Y$
Inklusion, $\mathcal{G}|_{X}:=f^{-1}\mathcal{G}$. Ist $X\subseteq Y$
offen stimmt $\mathcal{G}|_{X}$ mit der Einschränkung aus Beispiel
19 überein (cofinales Objekt). $\leadsto f^{-1}:PSh(Y)\rightarrow Sh(X)$
Funktor.

$g:Y\xrightarrow{\text{stetig}}Z$, $\mathcal{H}$ Prägarbe auf $Z$,
$U\subseteq X$ offen. 
\[
Z\underset{\text{offen}}{\supseteq}W\supseteq g(f(U))\Longleftrightarrow W\supseteq g(V)
\]

für ein $f(U)\subseteq V\subseteq Y$ offen. 
\begin{align*}
  \underset{\longrightarrow}{\lim}\underset{\longrightarrow}{\lim}=\underset{\longrightarrow}{\lim}\Longrightarrow f^{+}(g^{+}\mathcal{H}) & =(g\circ f)^{+}\mathcal{H}\quad(*)\\
  \Longrightarrow f^{-1}(g^{-1}\mathcal{H}) & =(g\circ f)^{-1}\mathcal{H}
\end{align*}

\begin{example}
  $\imath:\{x\}\rightarrow X$ Inklusion, $\mathcal{F}$ Prägarbe auf
  $X$. $\Longrightarrow\imath^{-1}(\mathcal{F})=\mathcal{F}_{x}$ per
  Definition. $(\ast)\Longrightarrow$
  \[
  \begin{array}{ccc}
    (f^{-1}\mathcal{G})_{x} & = & \mathcal{G}_{f(x)}\\
    \shortparallel &  & \shortparallel\\
    \imath^{-1}\circ(f^{-1}\mathcal{G}) & = & (f\circ\imath)^{-1}\mathcal{G}
  \end{array}
  \]
\end{example}

\begin{prop}[29]
  Für $f:X\rightarrow Y$ stetig sind die Funktionen $f_{\ast}$ und
  $f^{-1}$ zueinander adjungiert, d.h. für $\mathcal{F}$ Garbe auf
  $X$, $\mathcal{G}$ Prägarbe auf $Y$ existiert eine bijektion
  \begin{align*}
    \hom_{Sh(x)}(f^{-1}\mathcal{G},\mathcal{F}) & \longleftrightarrow\hom_{Psh(Y)}(\mathcal{G},f_{\ast}\mathcal{F})\\
    \varphi & \longmapsto\varphi^{\flat}\\
    \psi^{\sharp} & \longmapsfrom\psi
  \end{align*}

  funktoriell in $\mathcal{F}$ und $\mathcal{G}$.
\end{prop}

\begin{proof}
  $\varphi:f^{-1}\mathcal{G}\rightarrow\mathcal{F}$ Morphismus von
  Garben auf $X$. $t\in\mathcal{G}(V)$, $V\subseteq Y$ offen
  \begin{align*}
    \mathcal{G}(V) & \rightarrow f^{+}\mathcal{G}(f^{-1}(V))\xrightarrow{\imath_{f^{+}\mathcal{G}}}f^{-1}\mathcal{G}(f^{-1}(V))\xrightarrow{\varphi_{f^{-1}(V)}}\mathcal{F}(f^{-1}(V))=f_{\ast}\mathcal{F}(V)\\
    & \phantom{\rightarrow\ }\shortparallel\underset{\underset{Y\supseteq W\supseteq ff^{-1}(V)\subseteq V}{\longrightarrow}}{\lim}\\
    t & \mapsto\varphi_{V}^{\flat}(t)
  \end{align*}

  Definition von $\psi^{\#}$. $\mathcal{G}\xrightarrow{\psi}f_{\ast}\mathcal{F}$
  Morphismus von Prägarben auf . Wir definieren $\psi^{\#}:f^{+}\mathcal{G}\rightarrow\mathcal{F}$,
  welches dann $\psi^{\#}:f^{-1}\mathcal{G}\rightarrow\mathcal{F}$
  induziert. $U\subseteq X$ offen, $S\subseteq f^{+}\mathcal{G}(U)$,
  $s=[(V,s_{V})]$, $V\supseteq f(U)$, $s_{V}\in\mathcal{G}(V)$. $\Longrightarrow f^{-1}(V)\supseteq U$.
  \[
  \xymatrix{\psi_{V}(s_{V})\in f_{\ast}\mathcal{F}(V)\ar@{=}[r]\ar@{|->}[rd] & \mathcal{F}(f^{-1}(V))\ar[d]\\
    & \psi_{U}^{\#}(s)\in\mathcal{F}(U)
  }
  \]

  Überprüfe $\varphi^{\flat^{\#}}=\varphi$, $\psi^{\#^{\flat}}=\mathcal{H}$
  und Funktoriell.
\end{proof}
Definition + Proposition 29 verallgemeinern sich zu (Prä)Garben von
Ringen, $R$-Moduln, $R$-Algebren.

\textbf{Beschreibung} von:
\[
\mathcal{G}_{f(x)}=(f^{-1}\mathcal{G})_{x}\overset{\varphi_{x}}{\longmapsto}\mathcal{F}_{x},\ x\in X
\]

\[
\xymatrix{f(x)\in U\underset{\text{offen}}{\subseteq}Y & \mathcal{G}(U)\ar[r]^{\varphi_{U}^{\flat}}\ar@{-->}[d] & \mathcal{F}(f^{-1}(U))\ar[r] & \mathcal{F}_{x}\\
  \underset{\underset{U}{\longrightarrow}}{\lim} & \mathcal{G}_{f(x)}\ar@{-->}[rru]
}
\]


\section{Lokal geringte Räume}


\section{Die Strukturgrabe auf $\Spec(A)$}


\section{Der Funktor $A \protect\mapsto(\Spec(A),\cat{O}_{\Spec(A)})$}


\section{Beispiele}



\chapter{Schemata}
\label{chap:schemata}

\section{Schemata}

\begin{defn}
	Ein Schemata ist ein lokal geringter Raum $(X,\cat{O}_X)$, der eine offene Überdeckung $(U_i)_{i\in I}$
	besitz derart alle lokal geringter Räume $(U_i,\cat{O}_{X|U_i})$ affine Schemata sind.\\
	Für ein Schemata $S$ besitz $\textbf{Sch}/_{S}$ ~~ \textbf{Kategorie der Schemata über} $S$ oder $S$-\textbf{Schemata}\\
	\begin{itemize}
		\item Morphismen $X \rightarrow S$ von Schemata
	\end{itemize}
\end{defn}


\section{offene Unterschemata}


\section{Morphismen in affine Schemata}


\section{Morphismen der Form $\Spec(K)\protect\longrightarrow X$}


\section{Verkleben von Schemata und disjunkte Vereinigung}


\section{Der projektive Raum als Schemata}


\include{Chapter3/AlgGeo1-Chapter3-7_nullstellenmenge-in-projektiven-raeme}

\include{Chapter3/AlgGeo1-Chapter3-8_topologische-eigenschaften}

\section{Noethische Schemata}


\section{Generische Punkte}


\section{Reduzierte und ganze Schemata}


\section{Schemata von endlichem Type über $k$}


\section{Prävaritäten als Schemata}


\chapter*{Unterschemata und Immersion (Einbettung)}
\section{offene/abgeschlossen Einbettung}


\section{Reduzierte Unterschemata}



\chapter{Faserprodukte}
\label{chap:faserprod}

\include{Chapter4/AlgGeo1-Chapter4-1-Der-Punkte-Fuktor}

\section{Yoneda Lemma}
\underline{Ziel}: $h_X$ beschreibt $X$ eindeutig.

\section{Faserprodukt in beliebiger Kategorie}

$\cat{C}$ Kategorie, $S \in \text{Oj}(\cat{C})$


\section{Faserprodukte von Schemata}

\underline{Ziel:} $X,Y$ $S$-Schemata


\section{Beispiele}



\section{Basiswechsel}

$\mathcal{C}$ belibige Kategorie


\section{Fasern von Morphismen}


\section{Eigenschaften von Schematamorphismen}


\section{Urbilder und Schema-theoretische-Durchschnitte}


\section{Affine/Projektive Räume über belibige Basen}


\section{Diagonale, Graph, und Kern in belibige Kategorien}


\section{Diagonal für Schemata}



\section{Seperite Morphismen}


\section{Eigentliche Morphismen}


\chapter{Dimensionen}
\label{chap:dim}

<<<<<<< HEAD
\section{Allgemine Schemata}

Hallo
=======
\section{Allgemeine Schemata}
\begin{defn}[1]
  Für einen topologischen Raum $X$ ist die (Krull-)Dimension das Supremum
  der Länge aller Ketten
  \[
    Z_{0}\subsetneq Z_{1}\subsetneq\cdots\subsetneq Z_{n}\subseteq X
  \]

  irreduzibler abgeschlossener Teilmengen $Z_{i}$. $X$ sei \textbf{von
    Dimension $n$}, falls alle irreduzible Komponenten von $X$ die Dimension
  $n$ haben ($\dim\emptyset=-\infty$, sonst $\dim X\in\mathbb{N}\cup\{+\infty\}$).

  Die Dimension eines Schemas ist per Definition die Dimension des unterliegenden
  topologischen Raums, also $\dim X=\dim X_{\red}$.
\end{defn}

\begin{example}[2]
  \mbox{}
  \begin{enumerate}
  \item 
  \end{enumerate}
\end{example}
>>>>>>> d388b9febcfa9ca4bcebc412bc8c9e28b5b54b70


\section{Ganze Morphismen}


\newpage{}
\printindex{}
\end{document}
