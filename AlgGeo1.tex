\documentclass[12pt,a4paper]{book}
\usepackage[T1]{fontenc}
\usepackage[utf8]{inputenc}
\usepackage{geometry}
\geometry{verbose,tmargin=2cm,bmargin=2cm,lmargin=2cm,rmargin=2cm}
\pagestyle{headings}
\usepackage[ngerman]{babel}
\usepackage{verbatim}
\usepackage{amsmath}
\usepackage{amsthm}
\usepackage{amssymb}
\usepackage{stmaryrd}
\usepackage{makeidx}
\makeindex
\usepackage{setspace}
\usepackage[all]{xy}
\onehalfspacing
\usepackage[bookmarks=true]{hyperref}

%% Theorems (numbered by Part)
\newtheorem{thm}{Theorem}[chapter]
\theoremstyle{definition}
\newtheorem{example}[thm]{Beispiel}
\theoremstyle{definition}
\newtheorem{defn}[thm]{Definition}
\theoremstyle{plain}
\newtheorem{prop}[thm]{Satz}
\theoremstyle{plain}
\newtheorem{cor}[thm]{Korollar}
\theoremstyle{plain}
\newtheorem{lem}[thm]{Lemma}
\theoremstyle{remark}
\newtheorem{rem}[thm]{Bemerkung}
\theoremstyle{plain}

%% Theorems (unnumbered)
\newtheorem*{question*}{Frage}
\theoremstyle{remark}
\newtheorem*{claim*}{Behauptung}
\theoremstyle{definition}
\newtheorem*{example*}{Beispiel}
\theoremstyle{plain}
\newtheorem*{rem*}{Bemerkung}
\theoremstyle{remark}

%% User-specified commands
\DeclareMathOperator{\rad}{rad}
\DeclareMathOperator{\Spec}{Spec}
\DeclareMathOperator{\Quot}{Quot}
\DeclareMathOperator{\im}{\mathrm{im}}
\DeclareMathOperator{\Hom}{\mathrm{Hom}}
\DeclareMathOperator{\Mor}{\mathrm{Mor}}
\DeclareMathOperator{\id}{\mathrm{id}}

%%sets
\DeclareMathOperator{\CC}{\mathbb{C}}
\DeclareMathOperator{\RR}{\mathbb{R}}
\DeclareMathOperator{\QQ}{\mathbb{Q}}
\DeclareMathOperator{\ZZ}{\mathbb{Z}}
\DeclareMathOperator{\NN}{\mathbb{N}}

%% categories
\DeclareMathOperator{\ouv}{\mathcal{O}uv}
\DeclareMathOperator{\set}{\underline{Set}}
\DeclareMathOperator{\ab}{\underline{Ab}}
\DeclareMathOperator{\cattop}{\underline{Top}}
\DeclareMathOperator{\cring}{\underline{CRing}}
\DeclareMathOperator{\psh}{\underline{\mathcal{PS}h}}
\DeclareMathOperator{\sh}{\underline{\mathcal{S}h}}

%% commands
\newcommand{\cat}[1]{\mathcal{#1}}
\newcommand{\sheaf}[1]{\mathcal{#1}}
\renewcommand{\labelenumi}{(\roman{enumi})}
\renewcommand{\labelenumii}{\arabic{enumii}.}

\begin{document}

%% Title page
\title{Algebraische Geometrie I}
\author{Prof. Dr. Venjakob}
\maketitle

\tableofcontents{}
\newpage{}

\section*{Literatur}
\begin{itemize}
\item Görtz, Wedhorn. \emph{Algebraic Geometry I}
\item Hartshorne. \emph{Algebraic Geometry}
\item Shafarevich. \emph{Basic Algebraic Geometry 1 \& 2}
\item Grothendieck. \emph{Eléments de géometrie algébrique, EGA I-IV}
\end{itemize}

\paragraph{Kommutative Algebra}
\begin{itemize}
\item Brüske, Ischebeck, Vogel. \emph{Kommutative Algebra}
\item Kunz. \emph{Einführung in die kommutative Algebra und algebraische Geometrie}
\end{itemize}

\chapter{Prä-Varietäten}
\label{chap:prae-varietaeten}

\include{Chapter1/AlgGeo1-Chapter1-1_Einfuehrung}

\include{Chapter1/AlgGeo1-Chapter1-2_Die-Zariski-Topologie}

\include{Chapter1/AlgGeo1-Chapter1-3_Affine-algebraische-Mengen}

\include{Chapter1/AlgGeo1-Chapter1-4_Der-Hilbertsche-Nullstellensatz}

\include{Chapter1/AlgGeo1-Chapter1-5_Korrespondenz-zwischen-Radikalidealen}

\include{Chapter1/AlgGeo1-Chapter1-6_Irreduzible-topologische-Raeume}

\include{Chapter1/AlgGeo1-Chapter1-7_Irreduzible-algebraische-Mengen}

\include{Chapter1/AlgGeo1-Chapter1-8_Quasikompakte-und-noethersche-topologische-Raeume}

\include{Chapter1/AlgGeo1-Chapter1-9_Morphismen-von-affinen-algebraischen-Mengen}

\include{Chapter1/AlgGeo1-Chapter1-10_Unzulaenglichkeiten}

\include{Chapter1/AlgGeo1-Chapter1-11_Der-affine-Koordinatenring}

\include{Chapter1/AlgGeo1-Chapter1-12_Funktorielle-Eigenschaften-des-Koordinatenrings}

\include{Chapter1/AlgGeo1-Chapter1-13_Raeume-mit-Funktionen}

\include{Chapter1/AlgGeo1-Chapter1-14_Der-Raum-mit-Funktionen-zu-einer-affin-algebraischen-Menge}

\include{Chapter1/AlgGeo1-Chapter1-15_Funktorialitaet-der-Konstruktion}

\include{Chapter1/AlgGeo1-Chapter1-16_Definition-von-Praevarietaeten}

\include{Chapter1/AlgGeo1-Chapter1-17_Vergleich-mit-differenzierbaren-komplexen-Mannigfaltigkeiten}

\include{Chapter1/AlgGeo1-Chapter1-18_Topologische-Eigenschaften-von-Praevarietaeten}

\include{Chapter1/AlgGeo1-Chapter1-19_Offene-Untervarietaeten}

\include{Chapter1/AlgGeo1-Chapter1-20_Funktionenkoerper-einer-Praevarietaet}

\include{Chapter1/AlgGeo1-Chapter1-21_Abgeschlossene-Unterpraevarietaeten}

\include{Chapter1/AlgGeo1-Chapter1-22_Homogene-Polynome}

\include{Chapter1/AlgGeo1-Chapter1-23_Definition-des-projektiven-Raumes}

\include{Chapter1/AlgGeo1-Chapter1-24_Projektive-Varietaeten}

\include{Chapter1/AlgGeo1-Chapter1-25_Koordinatenwechsel-im-projektiven-Raum}

\include{Chapter1/AlgGeo1-Chapter1-26_Lineare-Unterraeume-des-projektiven-Raums}

\include{Chapter1/AlgGeo1-Chapter1-27_Kegel}

\include{Chapter1/AlgGeo1-Chapter1-28_Quadriken}

\chapter{Das Ringspektrum}
\label{chap:das-ringspektrum}

\include{Chapter2/AlgGeo1-Chapter2-1_Definition-von-Spec-A}

\include{Chapter2/AlgGeo1-Chapter2-2_Topologische-Eigenschaften-von-Spec-A}

\include{Chapter2/AlgGeo1-Chapter2-3_Der-Funktor-A-Spec-A}

\include{Chapter2/AlgGeo1-Chapter2-4_Beispiele}

\include{Chapter2/AlgGeo1-Chapter2-5_Garben}

\include{Chapter2/AlgGeo1-Chapter2-6_Halme-von-Garben}

\include{Chapter2/AlgGeo1-Chapter2-7_Die-zu-einer-Praegarbe-assoziierte-Garbe}

%% TODO:
%% - Limiten einfügen in Garben
\section{Direktes und Inverses Bild von Garben}


\section{Lokal geringte Räume}


\section{Die Strukturgrabe auf $\Spec(A)$}


\section{Der Funktor $A \protect\mapsto(\Spec(A),\cat{O}_{\Spec(A)})$}


\section{Beispiele}


\chapter{Schemata}
\label{chap:schemata}

\section{Schemata}

\begin{defn}
	Ein Schemata ist ein lokal geringter Raum $(X,\cat{O}_X)$, der eine offene Überdeckung $(U_i)_{i\in I}$
	besitz derart alle lokal geringter Räume $(U_i,\cat{O}_{X|U_i})$ affine Schemata sind.\\
	Für ein Schemata $S$ besitz $\textbf{Sch}/_{S}$ ~~ \textbf{Kategorie der Schemata über} $S$ oder $S$-\textbf{Schemata}\\
	\begin{itemize}
		\item Morphismen $X \rightarrow S$ von Schemata
	\end{itemize}
\end{defn}


\section{offene Unterschemata}


\section{Morphismen in affine Schemata}


\section{Morphismen der Form $\Spec(K)\protect\longrightarrow X$}


\section{Verkleben von Schemata und disjunkte Vereinigung}


\section{Der projektive Raum als Schemata}


\include{Chapter3/AlgGeo1-Chapter3-7_nullstellenmenge-in-projektiven-raeme}

\include{Chapter3/AlgGeo1-Chapter3-8_topologische-eigenschaften}

\section{Noethische Schemata}


\section{Generische Punkte}


\section{Reduzierte und ganze Schemata}


\section{Schemata von endlichem Type über $k$}


\section{Prävaritäten als Schemata}


\chapter*{Unterschemata und Immersion (Einbettung)}
\section{offene/abgeschlossen Einbettung}


\section{Reduzierte Unterschemata}



\chapter{Faserprodukte}
\label{chap:faserprod}

\include{Chapter4/AlgGeo1-Chapter4-1-Der-Punkte-Fuktor}

\include{Chapter4/AlGGeo1-Chapter4-2-Yoneda-Lemma}

\section{Faserprodukt in beliebiger Kategorie}

$\cat{C}$ Kategorie, $S \in \text{Oj}(\cat{C})$


\newpage{}
\printindex{}
\end{document}
