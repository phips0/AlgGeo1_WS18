\section{Reduzierte Unterschemata}

\[
  \Spec K[X,Y]\supset\Spec(K[X,Y]/Y^{2})\supset\Spec(K[X,Y]/Y)
\]

\begin{question*}
  Gibt es ein ,,kleinstes`` Unterschemata?
\end{question*}
Setze $\mathcal{N}_{X}\subset\mathcal{O}_{X}$, Garbifizierung der
Prägarben:
\[
  U\mapsto\nil(\Gamma(U,\mathcal{O}_{X})),\quad U\subseteq X\text{ offen}
\]

Definiere $X_{\red}:=(X,\mathcal{O}_{X}/\mathcal{N}_{X})$.

\begin{prop}[44]
  \mbox{}
  \begin{enumerate}
  \item Der geringte Raum $X_{\red}=(X,\mathcal{O}_{X}/\mathcal{N})$ ist
    ein Schema, also ein abgeschlossenes Unterschema von $X$ mit demselben
    topologischen Raum wie $X$.
  \item Falls $X'\subset X$ ein weiteres solches Unterschema ist, dann gibt
    es eine abgeschlossene Einbettung $f:X_{\red}\rightarrow X'$, sodass
    das Diagramm
    \[
      \xymatrix{X_{\red}\ar@{^{(}->}[r]\ar[d] & X\\
        X'\ar@{^{(}->}[ur]
      }
    \]
    kommutiert.
  \item $X_{\red}$ ist reduziert und heißt das \textbf{unterliegend reduzierte
      Unterschema von $X$.}
  \item Falls $X=\Spec A$ affin, gilt $X_{\red}=\Spec(A/\nil(A))$.
  \end{enumerate}
\end{prop}

\begin{proof}
  Ohne Einschränkung sei $X=\Spec A$ $\Longrightarrow U\mapsto\nil(\Gamma(U,\mathcal{O}_{X}))$
  ist bereits eine Garbe, da 
  \[
    \nil(\mathcal{O}_{X}(D(f))=\nil(A_{f})=\nil(A)A_{f}
  \]

  für alle $f\in A$. $\Longrightarrow X_{\red}=\Spec(A/\nil A)$ offensichtlich
  reduziert.\textbf{ Universelle Eigenschaft:} zu zeigen $\mathcal{O}_{X}\rightarrow\mathcal{O}_{X}/\mathcal{N}$
  faktorisiert:
  \[
    \xymatrix{\mathcal{O}_{X}\ar[r]\ar[rd] & \mathcal{O}_{X}/\mathcal{N}\\
      & \mathcal{O}_{X'}\ar[u]
    }
    ,
  \]

  d.h. $\ker(\mathcal{O}_{X}\rightarrow\mathcal{O}_{X'})\subset\mathcal{N}$.
  Es reicht zu zeigen:
  \[
    \ker(\mathcal{O}_{X}(U)\rightarrow\mathcal{O}_{X'}(U))\subset\Gamma(U,\mathcal{N})
  \]

  für alle $U$ offen affin. Ohne Einschränkung $X=\Spec A$, $X'$
  abgeschlossenes Unterschema $\Longrightarrow$ affin: $X'=\Spec B$.

  Zu zeigen: $\ker(A\rightarrow B)\subset\nil A$. Da nach Voraussetzung
  $\Spec B\rightarrow\Spec A$ bijektiv ist, folgt:
  \[
    \ker(A\rightarrow B)\subset\bigcap_{\mathfrak{g}\in\Spec A}\mathfrak{g}=\nil A
  \]
\end{proof}
\begin{cor}[45]
  $(X_{\red},i_{X}:X_{red}\rightarrow X)$ ist durch die universelle
  Eigenschaft eindeutig bis auf eindeutige Isomorphie bestimmt.
\end{cor}

\begin{lem}[46]
  Jede Einbettung $i:Z\rightarrow X$ ist ein Monomorphismus in $\sch$.
\end{lem}

\begin{proof}
  \mbox{}
  \begin{itemize}
  \item Stetige Abbildung $Z\hookrightarrow X$ klar.
  \item Die Garbenabbildung $i_{Y}^{\#}$ ist surjektiv.
  \end{itemize}
\end{proof}
% 
\begin{proof}[Beweis von Korollar 45]
  Sei $X'_{\red}$ ein weiteres Schema mit universeller Eigenschaft
  \[
    \exists f:X_{\red}\rightarrow X'_{\red},\quad g:X'_{\red}\rightarrow X_{\red}
  \]

  so dass
  \[
    \xymatrix{X_{\red}\ar[r]^{f}\ar[rd]_{i_{X}} & X'_{\red}\ar[r]^{g}\ar[d]^{i_{X'}} & X_{\red}\ar[ld]^{i_{X}}\\
      & X
    }
    ,\quad i_{X}\circ(g\circ f)=i_{X}\circ\id_{X_{\red}}
  \]

  $i_{X}$ Monomorphismus $\Longrightarrow g\circ f=\id_{X_{\red}}=f\circ g$.
  Auf $(X_{\red},i_{X})=\{\id\}$ $\Longrightarrow$ Eindeutig.
\end{proof}
$(\cdot)_{\red}$ ist ein Funktor, wie die folgende Proposition zeigt:
\begin{prop}[47]
  Sei $f:X\rightarrow Y$ ein Schemata-Morphismus. Dann gibt es:
  \[
    \xymatrix{X_{\red}\ar[r]^{f_{\red}} & Y_{\red}\\
      X_{\red}\ar@{^{(}->}[r]^{i_{X}}\ar[d]_{f_{\red}} & X\ar[d]^{f}\\
      Y_{\red}\ar@{^{(}->}[r]_{i_{Y}} & Y
    }
    ,\quad\text{d.d.}
  \]

  Für weitere Morphismen $g:Y\rightarrow Z$ gilt
  \[
    (g\circ f)_{\red}=g_{\red}\circ f_{\red}.
  \]
\end{prop}

\begin{proof}
  $i_{Y}$ Monomorphismus $\Longrightarrow f_{\red}$ eindeutig. \textbf{Existenz:
  }Nach Verklebungs-Lemma $\Longrightarrow$ ohne Einschränkung $X=\Spec A$,
  $Y=\Spec B$, $f\hat{=}\ \varphi:B\rightarrow A$.

  $\Longrightarrow\varphi(\nil(B))\subset\nil(A)$

  $\Longrightarrow\varphi_{\red}:B/\nil(B)\rightarrow A/\nil(A)$

  $\Longrightarrow f_{\red}:\Spec(A/\nil(A))\rightarrow\Spec(B/\nil(B))$.
\end{proof}
\begin{prop}[48]
  Sei $X$ Schemata, $Z\subset X$ lokal abgeschlossene Teilmenge.
  Dann existiert ein eindeutig bestimmtes reduziertes Unterschema mit
  topologischem Raum $Z$.
\end{prop}

\begin{proof}
  Eindeutigkeit: Korollar 45. Existenz: Verklebungslemma $\Longrightarrow$
  ohne Einschränkung $X=\Spec A$ affin und $Z\subset X$ abgeschlossen
  (sonst Überdeckung von $X$ zu $Z\subset_{\text{abg.}}U\subset_{\text{abg.}}X$)
  $\Longrightarrow\exists\mathfrak{a}\subset A$ so dass $Z=V(\mathfrak{A})$
  $\Longrightarrow Z'=\Spec(A/\mathfrak{a})$ ist abgeschlossenes Unterschema
  von $X$ mit topologischem Raum $Z$. Satz 44 $\Longrightarrow\exists Z'_{\red}\subset Z'\subset X$.
\end{proof}
Damit besitzt für ein lokal abgeschlossene Teilmenge die geordnete
Menge (bzgl. Inklusion) von Unterschema, denen topologischen Raum
$Z$ umfassen, ein eindeutiges minimales Element $Z_{\red}$, das
\textbf{reduzierte Unterschema} mit unterliegendem Raum $Z$.
