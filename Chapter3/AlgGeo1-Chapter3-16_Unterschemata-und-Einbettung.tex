\section{Unterschemata und Einbettung}

Offene und abgeschlossene Unterschemata sind Spezialfälle von \emph{lokal
  abgeschlossene} Unterschemata.
\begin{defn}[37]
  \mbox{}
  \begin{enumerate}
  \item Sei $X$ ein Schemata. Ein \textbf{Unterschemata} von $X$ ist ein
    Schemata $(Y,\mathcal{O}_{Y})$, so dass $Y\subset X$ eine lokal
    abgeschlossene Teilmenge von $X$ ist, und $Y$ ein abgeschlossenes
    Unterschemata von dem offenen Unterschemata $U=X\backslash(\overline{Y}\backslash Y)\subseteq X$
    ist. Wir haben dann einen natürlichen Morphismus $Y\rightarrow X$
    von Schemata.
  \item Eine \textbf{Einbettung} $i:Y\rightarrow X$ ist ein Morphismus von
    Schemata, dessen unterlegende stetige Abbildung ein Homöomorphismus
    von $Y$ auf eine lokale abgeschlossene Teilmenge von $X$ ist, und
    sodass für alle $y\in Y$ : 
    \[
      i_{y}^{\#}:\mathcal{O}_{X,i(y)}\rightarrow\mathcal{O}_{Y,y}
    \]
    surjektiv ist.
  \end{enumerate}
\end{defn}

\begin{rem}[38]
  \mbox{}
  \begin{enumerate}
  \item Ist $Y$ ein Unterschemata von $X$, dann ist $Y\hookrightarrow X$
    eine Einbettung. Umgekehrt bestimmt jede Einbettung einen Isomorphismus
    seiner Quelle mit einem eindeutigen Unterschemata seines Ziels.
  \item Ist $Y$ ein Unterschemata von $X$, wessen unterliegende Teilmenge
    abgeschlossen in $X$ ist, dann ist $Y$ ein abgeschlossenes Unterschemata
    von $X$.
  \item Das Analogon von $(ii)$ für offene Unterschemata ist i.A. falsch.
  \item Jede Einbettung $i:Y\hookrightarrow X$ faktorisiert als:
    \[
      \xymatrix{Y\ar@{^{(}->}[r]^{i}\ar@{^{(}->}[rd] & X\\
        & U=X\backslash(\overline{i(Y)}\backslash i(Y))\ar@{^{(}->}[u]
      }
    \]
  \end{enumerate}
\end{rem}

\begin{defn}[39]
  Sei $X$ ein Schemata und $Z,Z'$ Unterschemata. Wir sagen $Z'$
  \textbf{majorisiert} $Z$, wenn die Inklusion $Z\hookrightarrow X$
  faktorisiert als:
  \[
    \xymatrix{Z\ar@{^{(}->}[r]\ar[rd] & X\\
      & Z'\ar@{^{(}->}[u]
    }
    .
  \]
\end{defn}

\begin{rem}[40]
  Sei \textbf{P} die Eigenschaft eines Schemata-Morphismus, eine affine
  Einbettung, bzw. abgeschlossene Einbettung, bzw. Einbettung zu sein.
  Dann:
  \begin{enumerate}
  \item Die Eigenschaft \textbf{P} ist lokal auf der Basis, d.h. für $f:Z\rightarrow X$
    Morphismus, $X=\bigcup_{i\in I}U_{i}$ offene Überdeckung hat $f$
    die Eigenschaft \textbf{P} $\Longleftrightarrow\forall i$ hat $f^{-1}(U_{i})\rightarrow U_{i}$
    die Eigenschaft \textbf{P}.
  \item Die Komposition zweier Morphismen mit Eigenschaft \textbf{P} hat Eigenschaft
    \textbf{P}.
  \end{enumerate}
\end{rem}

\begin{example}[41]
  \mbox{}
  \begin{enumerate}
  \item Sei $I\subseteq R[T_{0},\ldots,T_{n}]$ homogenes Ideal. Dann ist
    $V_{+}(I)\subseteq\mathbb{P}_{R}^{n}$ ein abgeschlossenes Unterschemata
    von $\mathbb{P}_{R}^{n}$. (Nach Bemerkung 40.1, denn $V_{+}(I)\cap U_{i}\subseteq U_{i}$
    abgeschlossen.)
  \item Alle Unterschemata eines $k$-Schematas $X$ von endlichem Typ sind
    selbst von endlichem Typ. 
  \end{enumerate}
\end{example}
