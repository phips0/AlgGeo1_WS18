\section{Projektive und quasi-projektive Schemata über einen Körper}
\begin{defn}[42]
Sei $k$ ein Körper.
\begin{enumerate}
\item Ein $k$-Schemata $X$ heißt \textbf{projektiv} wenn es ein $n\geq0$
und eine abgeschlossene Einbettung $X\hookrightarrow\mathbb{P}_{k}^{n}$
gibt.
\item Ein $k$-Schemata $X$ heißt \textbf{quasi-projektiv }wenn es ein
$n\geq0$ und eine Einbettung $X\hookrightarrow\mathbb{P}_{k}^{n}$
gibt.
\end{enumerate}
\end{defn}

\begin{example}[43]
\mbox{}
\begin{enumerate}
\item Für ein homogenes Ideal $I$ sind $V_{+}(I)$ projektive Schemata
(Beispiel 41).
\item Sei $X=\Spec A$ affines $k$-Schemata von endlichem Typ. Dann ist
$X$ quasi-projektiv: $A\cong k[T_{1},\ldots,T_{n}]\backslash\mathfrak{a}$,
\[
\xymatrix{X\ar@{^{(}->}[r]\ar[rd] & \mathbb{A}^{n}\ar[d]^{j}\\
 & \mathbb{P}^{n}
}
\]
\end{enumerate}
\end{example}
