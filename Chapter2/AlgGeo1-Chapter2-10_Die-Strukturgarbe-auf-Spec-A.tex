\chapter*{Das Ringsprektrum als lokal geringter Raum}
\label{chap:ringspektrum-lokal-geringter-raum}
Ziel: volltreuer Funktor
\begin{align*}
  \text{Ringe} & \longrightarrow\text{Kategorie lokal geringter Räume}\\
  A & \longmapsto(\Spec A,\mathcal{O}_{\Spec A})
\end{align*}

\section{Die Strukturgarbe auf Spec A}
\label{sec:strukturgarbe-auf-spec-A}

Sei $X:=\Spec(A)$, $\mathcal{B}=\{D(f)\mid f\in A\}$ Basis der Topologie.

\textbf{Vorgegeben:} Definiere Prägarbe $\mathcal{O}_{X}$ auf $\mathcal{B}$,
die Garbenaxiome bzgl. $\mathcal{B}$ erfüllt.

\textbf{Wähle:} $\mathcal{O}_{X}(X)=A$ (vgl. Prävarietäten) bzw.
$\mathcal{O}_{X}(D(f))=A_{f}$, da
\begin{align*}
  \imath_{j}:A & \longrightarrow A_{f}\\
  a & \longmapsto\frac{a}{1}
\end{align*}

einen Homöomorphismus $D(f)\xrightarrow{\sim}\Spec(A_{f})$ induziert.
(``Funktionen mit möglichen Polen in $V(f)$).

\subsection{Wohldefiniertheit}
\label{subsec:strukturgarbe-wohldefiniertheit}

$D(f)=D(g)\Rightarrow A_{f}=A_{g}$ kanonisch. Dazu: 
\begin{align*}
  D(f)\subset D(g) & \Leftrightarrow\exists n\geq1\text{ d.d. }f^{n}\in A_{g}
\end{align*}


\subsection{Induzierte Abbildung}
\label{subsec:strukturgarbe-induzierte-abbildung}

\[
\mathcal{O}_{X}(D(g))\rightarrow\mathcal{O}_{X}(D(f)),\ \rho_{f,g}=:\text{res}_{D(f)}^{D(g)}
\]

Dies definiert eine Prägarbe auf $\mathcal{B}$.
\begin{thm}[orig. 33]
  Die Prägarbe $\mathcal{O}_{X}$ ist eine Garbe auf $\mathcal{B}$.
  Die induzierte Garbe auf $X$ (Proposition 20) werde auch mit $\mathcal{O}_{X}$
  bezeichnet. Da
  \[
  \mathcal{O}_{X,x}:=\underset{\underset{D(f)\ni x}{\longrightarrow}}{\lim}\mathcal{O}_{X}(D(f))=\underset{\underset{f\in\mathfrak{p}_{x}}{\longrightarrow}}{\lim}A_{f}=A_{\mathfrak{p}_{x}}
  \]
  mit $(X,\mathcal{O}_{X})=(\Spec A,\mathcal{O}_{\Spec A})$ (kurz $\Spec A$)
  ein lokal geringter Raum.
\end{thm}

\begin{proof}
  Sei $D(f)=\bigcup_{i\in I}D(f_{i})$ Überdeckung in $\mathcal{B}$.
  Zu zeigen:
  \begin{enumerate}
  \item $s\in\mathcal{O}_{X}(D(f))$ mit $s|_{D(f_{i})}=0$, $i\in I$.

    $\overset{!}{\Rightarrow}s=0$.
  \item $s_{i}\in\mathcal{O}_{X}(D(f_{i}))$, $i\in I$, mit $s_{i}|_{D(f_{i})\cap D(f_{j})}=s_{j}|_{D(f_{i})\cap D(f_{j})}$
    $\forall i,j\in I$.

    $\overset{!}{\Rightarrow}\exists s\in\mathcal{O}_{X}(D(f))$ mit $s|_{D(f_{i})}=s_{i}$
    $\forall i\in I$.
  \end{enumerate}
  Ohne Einschränkung:
  \begin{itemize}
  \item $I$ endlich, da $D(f)$ quasi-kompakt.
  \item $f=1$, $D(f)=X$ (mit $(A_{f},\mathcal{O}_{X}|_{D(f)})$ statt $(A,\mathcal{O}_{X})$
    betrachtet) 
    \[
    X=\bigcup_{i\in I}D(f_{i})\Leftrightarrow(f_{i}\mid i\in I)=A
    \]
    Es folgt: $b_{i}=b_{i}(n)\in A$ d.d. $\sum_{i\in I}b_{i}f_{i}^{n}=1$
    \textbf{Zerlegung der 1}. (z)
  \item[Zu 1.] Sei $s=a\in A$ d.d. $0=\frac{a}{1}\in A_{f}$, $\forall i\in I$.
    $I$ endlich, also $\exists n\geq1$ d.d. $f_{i}^{n}a=0$ $\forall i\in I$.
    Mit $(z)$ folgt
    \[
    a=\left(\sum_{i\in I}b_{i}f_{i}^{n}\right)a=0
    \]
  \item[Zu 2.] $s_{i}=\frac{a_{i}}{f_{i}^{n}}$ für $n$ geeignet, unabhängig von
    $i\in I$ (endlich). Nach Voraussetzung:
    \[
    \frac{a_{i}}{f_{i}^{n}}=\frac{a_{j}}{f_{j}^{n}}\in A_{f_{i}f_{j}},\quad D(f_{i})\cap D(f_{j})=D(f_{i}f_{j})
    \]
    Es folgt: $\exists m\geq1$ d.d. $(f_{i}f_{j})^{m}(f_{j}^{n}a_{i}-f_{i}^{n}a_{j})=0$
    $\forall i,j$.
    \begin{align*}
      \frac{a_{i}}{f_{i}^{n}} & =\frac{f_{i}^{m}a_{i}}{f_{i}^{n+m}}=:\frac{a_{i}'}{f_{i}^{n'}},\quad n'=n+m
    \end{align*}
    Ohne Einschränkung: $f_{j}^{n}a_{i}=f_{i}^{n}a_{j}$ $\forall i,j\in I$,
    ({*}) denn:
    \begin{align*}
      f_{j}^{m+n}f_{i}a_{i} & =f_{i}^{m+n}f_{j}^{m}a_{j}\\
      f_{j}^{n'}a_{i}' & =f_{i}^{n'}a_{j}'
    \end{align*}
    Setze $s:=\sum_{j\in I}b_{j}a_{j}\in A$ ($(z)$). Es folgt:
    \[
    f_{i}^{n}s=f_{i}^{n}\sum b_{i}a_{j}=\sum b_{j}(f_{i}^{n}a_{j})\overset{(*)}{=}\left(\sum b_{i}f_{i}^{n}\right)a_{i}\overset{(z)}{=}a_{i}
    \]
    also $\frac{s}{1}=\frac{a_{i}}{f^{n}}=s_{i}$.
  \end{itemize}
\end{proof}

\end{document}
