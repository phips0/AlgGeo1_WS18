\section{Lokal geringte Räume}
\label{sec:lokal-geringte-raeume}
\begin{defn}
  Ein geringter Raum ist ein Paar $(X,\mathcal{O}_{X})$ bestehend aus
  einem topologischen Raum $X$ und einer Garbe $\mathcal{O}_{X}$ (kommutativer)
  Ringe. Ein Morphismus $(X,\mathcal{O}_{X})\rightarrow(Y,\mathcal{O}_{Y})$
  geringter Räume ist wiederum ein Paar $(f,f^{\flat})$ bestehend aus
  einer stetigen Abbildung $f:X\rightarrow Y$ und einem Homomorphismus
  $f^{\flat}:\mathcal{O}_{Y}\rightarrow f_{\ast}\mathcal{O}_{X}$ von
  Ringgarben auf $Y$. Dieses Datum ist gleichbedeutend (Proposition
  29) mit $(f,f^{\sharp})$, wobei nun $f^{\sharp}:f^{-1}\mathcal{O}_{Y}\rightarrow\mathcal{O}_{X}$
  ein Garbenhomomorphismus auf $X$ ist.

  Bezeichne: $f$ oder $(f,f^{\flat})$ oder $(f,f^{\sharp})$. Damit
  haben wir eine \textbf{Kategorie der geringten Räume}. $\mathcal{O}_{X}$
  heißt Strukturgarbe\index{Strukturgarbe} von $X$, oft schreiben
  wir $X$ für $\mathcal{O}_{X}$.

  Idee: $\mathcal{O}_{X}$ beschreibt die zulässigen Funktionen auf
  $U\subset X$, d.h. etwa stetige, differenzierbare, holomorphe, rigid
  analytische usw. Funktionen. Solche Funktionen auf $V\subset Y$ sollen
  beim ``Zurückziehen'' unter $f$ in dieselbe Klasse überführt werden.
  Dies wird formal durch das Datum $f^{\flat}$ sichergestellt.
\end{defn}

\begin{notation*}
  Wenn $A$ ein lokaler Ring ist, $\mathfrak{m}_{A}$ das maximale Ideal,
  und $\kappa(A)=A/\mathfrak{m}_{A}$ Restklassenkörper. Ein Homomorphismus
  $\varphi:A\rightarrow B$ lokaler Ringe heißt \textbf{lokal}, falls
  $\varphi(\mathfrak{m}_{A})\subset\mathfrak{m}_{B}$. $(f,f^{\flat})=(f,f^{\sharp})=X\rightarrow Y$
  Morphismus geringter Räume induziert:
  \begin{align*}
    & \mathcal{O}_{Y,f(x)}=(f^{-1}\mathcal{O}_{Y})_{x}\xrightarrow{f_{x}^{\sharp}}\mathcal{O}_{X,x}\\
    \text{oder } & \xymatrix{\mathcal{O}_{Y}(U)\ar[r]^{f_{U}^{\flat}}\ar[d] & \mathcal{O}_{X}(f^{-1}(U))\ar[d] & f(x)\in U\subset Y\text{ offen}\\
      \mathcal{O}_{Y,f(x)}=\lim\mathcal{O}_{Y}(U)\ar@{-->}[r] & \mathcal{O}_{X,x}
    }
  \end{align*}
\end{notation*}
\begin{defn}[orig. 31]
  Ein lokal geringter Raum ist ein geringter Raum $(X,\mathcal{O}_{X})$,
  für der $\mathcal{O}_{X,x}$ für alle $x\in X$ ein \emph{lokaler
    Ring} ist. Ein Morphismus $(X,\mathcal{O}_{X})\rightarrow(Y,\mathcal{O}_{Y})$
  lokal geringter Räume ist ein Morphismus geringter Räume $(f,f^{\flat})$,
  so dass die induzierte Abbildung 
  \[
  f_{x}^{\sharp}:\mathcal{O}_{Y,f(x)}\rightarrow\mathcal{O}_{X,x}
  \]

  ein lokaler Ringhomomorphismus ist für alle $x\in X$. Dies führt
  zu einer Unterkategorie der Kategorie geringter Räume, die im Allgemeinen
  \emph{nicht }voll ist, d.h. es gibt Morphismen $f$ geringter Räume
  zwischen lokal geringten Räume, die nicht lokal sind!
\end{defn}

Bezeichne: 
\begin{itemize}
\item $\mathcal{O}_{X,x}$ der ``lokale Ring von $X$ in $x$'';
\item $\mathfrak{m}_{x}$ maximales Ideal;
\item $\kappa(x):=\mathcal{O}_{X,x}/\mathfrak{m}_{x}$ Restklassenkörper
  (bei $x$). \texttt{
  \[
  \xymatrix@R=0pt{\mathcal{O}_{X}(U)\ar[r] & \mathcal{O}_{X,x}\ar[r] & \kappa(x)\\
    f\ar[rr] &  & f(x)
  }
  \]
}
\end{itemize}
Warum \textbf{lokal} geringte Räume? Heuristik:
\begin{align*}
  \mathcal{O}_{X}(U) & \leftrightarrow\text{Funktionen auf }U\\
  \mathcal{O}_{X,x} & \leftrightarrow\text{Funktionen auf Umgebung }U\text{ von }x
\end{align*}

\emph{Wunsch}: $f(x)\neq0\overset{!}{\Rightarrow}f$ ist invertierbar
auf einer kleinen Umgebung $V$ von $x$, d.h. 
\[
\mathcal{O}_{X,x}\backslash\underbrace{\{f\mid f(x)=0\}}_{=\mathfrak{m}_{x}}\subset\mathcal{O}_{X,x}^{\times},
\]

also $\mathcal{O}_{X,x}$ lokal. \emph{Ferner}: $g\mathcal{O}_{Y,f(x)}$
mit $g(f(x))=0$ sollte implizieren: $(g\circ f)(x)=0$. Übersetzt:
\[
f_{x}^{\sharp}(\mathfrak{m}_{f(x)})\subset\mathfrak{m}_{x},\quad f_{x}^{\sharp}(g)="g\circ f"
\]

\begin{example}[orig. 32]
  $\varphi_{X}$ Garbe der $\mathbb{R}$-wertiger stetiger Funktionen
  auf einem topologischen Raum $X$. $\varphi_{X,x}$ Ring der Keime
  $[s]$ stetiger Funktionen in einer Umgebung von $X$:
  \[
  \mathfrak{m}_{x}=\{[s]\in\varphi_{X,x}\mid0=s(x)\}
  \]

  ist einziges maximale Ideal, d.h. $(X,\varphi_{X})$ ist lokal geringter
  Raum. 

  \emph{Denn}: Sei $[s]\in\varphi_{X,x}\backslash\mathfrak{m}_{x}$
  gegeben.

  $\Rightarrow s(x)\neq0$ für alle $s\in[s]$. 

  $\Rightarrow(s$ stetig) $\exists x\in U\subset X$ offen mit $s(u)\neq0$
  für alle $u\in U$.

  $\Rightarrow\frac{1}{s|_{U}}\in\varphi_{X}(U)$ existiert.

  $\Rightarrow\varphi_{X,x}\backslash\mathfrak{m}_{x}=\varphi_{X,x}^{\times}$
  Einheitengruppe. Es ist: 
  \[
  \varphi_{X,x}\rightarrow\mathbb{R},\ [s]\mapsto s(x)
  \]

  ein surjektiver Ringhomomorphismus mit $\ker=\mathfrak{m}_{x}$.

  $\Rightarrow\kappa(x)\cong\mathbb{R}$. Sei $f:X\rightarrow Y$ stetig,
  $V\subset Y$ offen.
  \begin{align*}
    f_{x}^{\flat}:\varphi_{Y}(V) & \longrightarrow\varphi_{X}(f^{-1}(V))=f_{\ast}\varphi_{X}(V)\\
    t & \longmapsto t\circ f
  \end{align*}

  Es folgt:
  \begin{align*}
    \varphi_{Y,f(x)} & \longrightarrow\varphi_{X,x}\\{}
           [t] & \longmapsto[t\circ f]
  \end{align*}

  ist ein Morphismus lokal geringter Räume. Ebenso lassen sich Prävarietäten
  über lokal geringte Räume interpretieren!
\end{example}
