\section{Separierte Morphismen}

Erinnerung: Für einen topologischen Raum $X$ sind äquivalent:
\begin{enumerate}
\item $X$ ist Hausdorff.
\item $\Delta\subset X\times X$ ist abgeschlossen bzgl. der Produkttopolgie.
\item Für jedes Paar von stetigen Abbildungen $f,g:Y\rightarrow X$ ist
  $\ker(f,g)\subset X$ abgeschlossen.
\item Für jeder stetige Morphismus $f:Y\rightarrow X$ ist $\Gamma_{f}\subset X\times Y$
  abgeschlossen.
\end{enumerate}
Für ein Schema $X$ ist der unterliegende topologische Raum selten
Hausdorff, aber $(2)-(4)$ geben sinnvolle Konzepte für Schemata (im
Allgemeinen ist die Produkttopologie ungleich der Faserprodukttopologie).
\begin{defn}[30]
  Ein Morphismus $v:Y\rightarrow S$ von Schemata heißt separiert,
  falls folgende äquivalente Bedingungen erfüllt sind.
  \begin{enumerate}
  \item $\Delta_{Y/S}$ ist eine \emph{abgeschlossene} Immersion.
  \item Für jedees Paar $f,g:X\rightarrow Y$ ist $\ker(f,g)\subset X$ ein
    abgeschlossenes Unterschema.
  \item Für jeden $S$-Morphismus $f:X\rightarrow Y$ ist $\Gamma_{f}$ eine
    abgeschlossene Immersion.
  \end{enumerate}
  Dann heißt auch $Y$ ist \textbf{separiert über $S$}. Ein Schema
  $Y$ heißt \textbf{separiert}, falls es separiert über $\mathbb{Z}$
  ist.
\end{defn}

\begin{proof}
  Die Äquivalenz folgt nach Proposition 26, und dass ,,abgeschlossene
  Immersion`` stabil unter Basiswechsel ist.
\end{proof}
Nach Proposition 27 ist jeder Morphismus zwischen affinen Schemata
separiert. Insbesondere ist jedes affine Schemata separiert.
\begin{prop}[31]
  Seien $X,Y\in\schs$, $Y$ separiert über $S$, $U\subset X$ offenes
  dichtes Unterschema, $f,g:X\rightarrow Y$ $S$-Morphismus mit $f|_{U}=g|_{U}$.
  Dann ist $f|_{X_{\red}}=g|_{X_{\red}}$.
\end{prop}

\begin{proof}
  Nach Voraussetzung ist $U\subseteq\ker(f,g)$. Da $Y$ separiert ist
  über $S$, ist $\ker(f,g)\subset X$ abgeschlossenes Unterschema.
  Da $U$ dicht ist in $X$, ist der unterliegende topologische Raum
  von $\ker(f,g)$ gleich $X$. $\Longrightarrow X_{\red}\subseteq\ker(f,g)$
  als Schema.
\end{proof}
\begin{example}[32]
  Sei
  \[
    \,\xymatrix{\ar@{-}[r] & :\ar@{-}[r] & \,}
  \]

  affine Gerade mit Doppelpunkt (siehe Beispiel 11, III.5) ist \emph{nicht}
  separiert: $V\subset U$ offen, nicht abgeschlossen.
  \[
    j,j':U\longrightarrow U\cup_{V}U\quad\Rightarrow\quad\ker(j,j')=V\subset U\subset X\text{\,nicht abg.!}
  \]
\end{example}

\begin{rem}[33]
  $\mathbb{P}$ sei eine Eigenschaft von Morphismen, sodass gilt:
  \begin{itemize}
  \item stabil unter Komposition und Basiswechsel;
  \item jede (abgeschlossene) Immersion erfült $\mathbb{P}$.
  \end{itemize}
  Für jedes kommutative Diagramm:
  \[
    \xymatrix{X\ar[r]^{f}\ar[rd]_{u} & Y\ar[d]^{v}\\
      & S
    }
  \]

  mit $u$ erfüllt $\mathbb{P}$ (und $v$ seperariert) $\Longrightarrow f$
  erfüllt $\mathbb{P}$, da:
  \[
    f:X\xrightarrow[\text{(abg.) Imm. erfüllt }\mathbb{P}]{\Gamma_{f}}X\times_{S}Y\xrightarrow[\text{Basisw. erfüllt }\mathbb{P}]{q}Y
  \]

  erfüllt $\mathbb{P}$, wegen stabil unter Komposition.
\end{rem}

\begin{prop}[34]
  \mbox{}
  \begin{enumerate}
  \item Jeder Monomorphismsu von Schemata (insbesondere jede Immersion) ist
    separiert.
  \item Die Eigenschaft ,,separiert`` ist stabil unter Komposition, stabil
    unter Basiswechsel, und lokal bzgl. Ziel.
  \item Ist die Komposition $X\rightarrow Y\rightarrow Z$ zweier Morphismen
    separiert, so auch $X\rightarrow Y$.
  \item $f:X\rightarrow Y$ ist seperariert genau dann, wenn $f_{\red}:X_{\red}\rightarrow Y_{\red}$
    separiert ist.
  \end{enumerate}
\end{prop}

\begin{proof}
  \mbox{}
  \begin{enumerate}
  \item Wenn $f$ Monomorphismus ist (d.h. injektiv auf $T$-wertigen Punkten
    für alle Schemata $T$), dann ist $\Delta_{f}$ Isomorphismus (d.h.
    bijektiv auf allen $T$-wertigen Punkte für alle $T$). Insbesondere
    ist $\Delta_{f}$ eine abgeschlossene Immersion.
  \item Seien $f:X\rightarrow Y$, $g:Y\rightarrow Z$ separierte Schemata-Morphismen,
    $p,q:X\times_{Y}X\rightarrow X$ die zwei Projektionen. Das folgende
    Diagramm ist kommutativ, und das rechte Viereck ist kartesisch (überprüfe
    in $\set$):
    \[
      \xymatrix{X\ar[r]^{\Delta_{f}}\ar[rd]_{\Delta_{g\circ f}} & X\times_{Y}X\ar[r]^(0.6){f\circ p=f\circ q}\ar[d]|-{(p,q)_{Z}} & Y\ar[d]^{\Delta_{g}}\\
        & X\times_{Z}X\ar[r]_{f\times f} & Y\times_{Z}Y.
      }
    \]
    Da $\Delta_{g}$ abgeschlossene Immersion, ist $(p,q)_{Z}$ abgeschlossene
    Immersion $\Longrightarrow$ die Komposition $\Delta_{g\circ f}$
    ist abgeschlossene Immersion $\Longrightarrow g\circ f$ ist separiert.

    $\Delta_{f}$ abgeschlossene Immersion $\Longrightarrow\Delta_{f_{(S')}}$
    ist abgeschlossene Immersion. Dies zeigt das ,,separiert`` abgeschlossen
    ist unter Basiswechsel. Weiter ist ,,separiert`` lokal bzgl. Ziel,
    da dies gilt für ,,abgeschlossene Immersion``.
  \item Folgt aus (1), (2) nach Bemerkung 33. ($u=``\circ"$, $v=Y\rightarrow Z$,
    $f:X\rightarrow Y$)
  \item Sei $f:X\rightarrow S$ Morphismus, $i:X_{\red}\rightarrow X$ kanonische
    Immersion. Dann ist $i$ surjektive Immersion, also ein universeller
    Homöomorphismus. Identifizieren von $X_{\red}\times_{S_{\red}}X_{\red}$
    mit $X_{\red}\times_{S}X_{\red}$ liefert $\Delta_{f}\circ i=(i\times_{S}i)\circ\Delta_{f_{\red}}$.
    $\Longrightarrow\Delta_{f}$ ist abgeschlossene Immersion genau dann
    wenn $\Delta_{f_{\red}}$ abgeschlossene Immersion.
  \end{enumerate}
\end{proof}

\begin{example}[35]
  Sei $S$ beliebiges Schema, $n\in\mathbb{N}$. Dann ist $\mathbb{A}_{S}^{n}$
  separiert über $S$, ebenso jedes Unterschema, denn $\mathbb{A}_{S}^{n}=\mathbb{A}_{\mathbb{Z}}^{n}\times_{\Spec\mathbb{Z}}S$
  und ,,separiert`` ist stabil unter Basiswechsel nach Proposition
  34.
\end{example}

\begin{prop}[36]
  Sei $S=\Spec R$ affin und $X$ ein $S$-Schema. Dann sind äquivalent:
  \begin{enumerate}
  \item $X$ ist separiert.
  \item Für je zwei offene affine $U,V\subseteq X$ ist $U\cap V$ affin,
    und
    \begin{align*}
      \rho_{U,V}:\mathcal{O}_{X}(U)\otimes_{R}\mathcal{O}_{X}(V) & \longrightarrow\mathcal{O}_{X}(U\cap V),\\
      s\otimes t & \longmapsto s|_{U\cap V}\cdot t|_{V\cap U}.
    \end{align*}
  \item Es gibt eine offene affine Überdeckung $X=\bigcup_{i\in I}U_{i}$,
    sodass $\forall i,j\in I$: $\rho_{U,V}$ ist surjektiv.
  \end{enumerate}
\end{prop}

\begin{proof}
  Für alle offene affine $U,V\subseteq X$ gilt:
  \[
    U\cap V=\Delta_{X/S}^{-1}(U\times_{S}V).
  \]

  ,,abgeschlossene Immersion`` ist lokal auf dem Ziel, daher: $\Delta_{X/S}$
  ist abgeschlossene Immersion

  $\Longleftrightarrow$ für alle $U,V\subseteq X$ offen affin ist
  \[
    U\cap V\xrightarrow{\Delta_{X/S}|_{U\cap V}}U\times_{S}V
  \]

  abgeschlossene Immersion.

  $\Longleftrightarrow$ für jede offene affine Überdeckung $X=\bigcup_{i\in I}U_{i}$
  und alle $i,j\in I$ ist
  \[
    U_{i}\cap U_{j}\longrightarrow U_{i}\times_{S}U_{j}
  \]

  abgeschlossene Immersion. Sind $U=\Spec A$, $V=\Spec B$ affin, so
  ist auch
  \[
    U\times_{S}V=\Spec(A\otimes_{R}B)
  \]

  affin. Daher:
  \[
    U\cap V\longrightarrow U\times_{S}V
  \]

  abgeschlossene Immersion. $\Longleftrightarrow\rho_{U,V}$ surjektiv.
\end{proof}
\begin{example}[37]
  Für jedes Schema $S$ und $n\in\mathbb{N}$ ist $\mathbb{P}_{S}^{n}$
  separiert über $S$, denn ,,separiert`` ist lokal auf dem Ziel (Proposition
  34), daher ohne Einschränkung $S=\Spec R$ affin. Sei $\mathbb{P}_{R}^{n}=\bigcup_{i=0}^{n}U_{i}$
  mit $U_{i}=\Spec R\left[\frac{X_{0}}{X_{i}},\ldots,\frac{\widehat{X_{i}}}{X_{i}},\ldots,\frac{X_{n}}{X_{i}}\right]$
  und
  \begin{align*}
    \rho_{U_{i},U_{j}}:R & \left[\frac{X_{0}}{X_{i}},\ldots,\frac{\widehat{X_{i}}}{X_{i}},\ldots,\frac{X_{n}}{X_{i}}\right]\otimes_{R}R\left[\frac{X_{0}}{X_{j}},\ldots,\frac{\widehat{X_{j}}}{X_{j}},\ldots,\frac{X_{n}}{X_{j}}\right]\\
                         & \longrightarrow R\left[\frac{X_{0}}{X_{i}},\ldots,\frac{\widehat{X_{i}}}{X_{i}},\ldots,\frac{X_{n}}{X_{i}}\right]\left[\frac{X_{i}}{X_{j}}\right]
  \end{align*}

  ist surjektiv.
\end{example}

\begin{example}[38]
  Sei $k$ algebraisch abgeschlossener Körper, $X$ Prävarietät über
  $k$, d.h. ganzes Schema von endlichem Typ über $k$. $X$ heißt \textbf{Varietät},
  wenn $X$ separabel ist. Affine Prävarietäten sind also automatisch
  Varietäten. $\mathbb{P}^{n}(k)$ ist Varietät (Beispiel 37) $\Longrightarrow$
  Jede quasi-projektive Prävarietät ist eine Varietät!
\end{example}
