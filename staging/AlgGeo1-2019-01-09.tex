%% LyX 2.3.2 created this file.  For more info, see http://www.lyx.org/.
%% Do not edit unless you really know what you are doing.
\documentclass[oneside,ngerman]{book}
\usepackage[T1]{fontenc}
\usepackage[utf8]{inputenc}
\setcounter{secnumdepth}{3}
\setcounter{tocdepth}{3}
\usepackage{amsmath}
\usepackage{amsthm}
\usepackage{amssymb}
\usepackage[all]{xy}

\makeatletter
%%%%%%%%%%%%%%%%%%%%%%%%%%%%%% Textclass specific LaTeX commands.
\theoremstyle{plain}
\ifx\thechapter\undefined
	\newtheorem{thm}{\protect\theoremname}
\else
	\newtheorem{thm}{\protect\theoremname}[chapter]
\fi
\theoremstyle{definition}
\newtheorem{defn}[thm]{\protect\definitionname}
\theoremstyle{remark}
\newtheorem{rem}[thm]{\protect\remarkname}
\theoremstyle{definition}
\newtheorem{example}[thm]{\protect\examplename}
\theoremstyle{plain}
\newtheorem*{question*}{\protect\questionname}

%%%%%%%%%%%%%%%%%%%%%%%%%%%%%% User specified LaTeX commands.

%% User-specified commands
\DeclareMathOperator{\rad}{rad}
\DeclareMathOperator{\Spec}{Spec}
\DeclareMathOperator{\maxspec}{MaxSpec}
\DeclareMathOperator{\Quot}{Quot}
\DeclareMathOperator{\im}{\mathrm{im}}
\DeclareMathOperator{\Hom}{\mathrm{Hom}}
\DeclareMathOperator{\Mor}{\mathrm{Mor}}
\DeclareMathOperator{\id}{\mathrm{id}}
\DeclareMathOperator{\res}{res}
\DeclareMathOperator{\Abb}{Abb}
\DeclareMathOperator{\nil}{nil}
\DeclareMathOperator{\supp}{supp}
\DeclareMathOperator{\red}{red}

%%sets
\DeclareMathOperator{\CC}{\mathbb{C}}
\DeclareMathOperator{\RR}{\mathbb{R}}
\DeclareMathOperator{\QQ}{\mathbb{Q}}
\DeclareMathOperator{\ZZ}{\mathbb{Z}}
\DeclareMathOperator{\NN}{\mathbb{N}}

%% categories
\DeclareMathOperator{\ouv}{\mathcal{O}uv}
\DeclareMathOperator{\set}{\underline{Set}}
\DeclareMathOperator{\ab}{\underline{Ab}}
\DeclareMathOperator{\cattop}{\underline{Top}}
\DeclareMathOperator{\cring}{\underline{CRing}}
\DeclareMathOperator{\ring}{\underline{Ring}}
\DeclareMathOperator{\psh}{\underline{\mathcal{PS}h}}
\DeclareMathOperator{\sh}{\underline{\mathcal{S}h}}
\DeclareMathOperator{\aff}{\underline{Aff}}
\DeclareMathOperator{\schs}{\underline{Sch/S}}
\DeclareMathOperator{\pres}{\underline{Prevar/S}}
\DeclareMathOperator{\prek}{\underline{Prevar/k}}
\DeclareMathOperator{\affk}{\underline{AffVar/k}}
\DeclareMathOperator{\topcp}{\underline{TopCP}}
\DeclareMathOperator{\Top}{\underline{Top}}

\makeatother

\usepackage{babel}
\providecommand{\definitionname}{Definition}
\providecommand{\examplename}{Beispiel}
\providecommand{\questionname}{Frage}
\providecommand{\remarkname}{Bemerkung}
\providecommand{\theoremname}{Theorem}

\begin{document}
\begin{proof}
Sei $Z$ ein abgeschlossenes Unterschemata, $i:Z\hookrightarrow X$
Inklusion. Definition $\Longrightarrow\mathcal{O}_{X}\twoheadrightarrow i_{\ast}\mathcal{O}_{Z}$
surjektiv. Sei:
\[
\mathcal{I}_{Z}:=\ker(\mathcal{O}_{X}(X)\rightarrow\Gamma(X,i_{\ast}\mathcal{O}_{Z})=\Gamma(Z,\mathcal{O}_{Z}))\unlhd A
\]

Ideal. Falls $Z$ von der Form $V(\mathfrak{a})$ ist (was zu zeigen
ist!) gilt $\mathcal{I}_{Z}=\mathfrak{a}$. Daher reicht z.z. $Z=V(\mathcal{I}_{Z})$.
\textbf{Dazu:
\[
\xymatrix{A\ar[r]^{\varphi}\ar@{->>}[dr] & \Gamma(Z,\mathcal{O}_{Z})\\
 & A/\mathcal{I}_{Z}\ar@{^{(}->}[u]
}
\]
}

faktorisiert per Definition. $\Longrightarrow$ Das Diagramm
\[
\xymatrix{Z\ar@{^{(}->}[r]^{i}\ar[rd] & X\\
 & \Spec(A/\mathcal{I}_{Z})\ar@{^{(}->}[u]
}
\]

kommutiert. Es ist $\Mor(Z,\Spec A)=\Hom(A,\Gamma(Z,\mathcal{O}_{Z}))$,
ohne Einschränkung: $\mathcal{I}_{Z}=0$ (sonst ersetze $A$ durch
$A/\mathcal{I}_{Z}$). Zu zeigen: $Z\hookrightarrow X=V(\mathfrak{a})$
ist ein Isomorphismus.

Wir wissen: die unterliegende stetige Abbildung topologischer Räume
ist injektiv und abgeschlossen. ($A\subset_{\text{abg.}}Z\subset X$
$\Longrightarrow A\subset X$ abg.) Bleibt zu zeigen: surjektiv.

Sei $U\subseteq Z$ offen mit $(U,\mathcal{O}_{X|U})$ affin. So gilt:
\begin{align*}
U\subset U\backslash D(\varphi(s)|_{U}) & =V_{U}(\varphi(s)|_{U})\\
 & =\varphi(s)|_{U}\in\mathcal{O}_{Z}(U)\text{ nilpotent}.
\end{align*}

Endliche Überdeckung von $Z$ durch affine Schemata $\Longrightarrow\varphi(s^{N})=0$.
$\varphi$ injektiv $\Longrightarrow s^{N}=0$ bzw. $V(s)=X$. $Z$
abgeschlossen in $X$ $\Longrightarrow i(Z)=X$.

\textbf{Behauptung: }Der Homomorphismus von Garben $\mathcal{O}_{X}\rightarrow\mathcal{O}_{Z}$
ist bijektiv. Reicht zu zeigen: injektiv (da surjektiv nach Voraussetzung).

Sei $x\in X$ beliebig, $\mathcal{O}_{X,x}=A_{\mathfrak{p}_{x}}$.
Sei $\frac{g}{1}\in\ker(\mathcal{O}_{X,x}\rightarrow\mathcal{O}_{Z,x}$).
Überdecke
\[
Z=U\cup\bigcup_{i\in I}U_{i},\quad\#I<\infty
\]

mit:
\begin{enumerate}
\item $(U,\mathcal{O}_{Z\mid U})$, $(U_{i},\mathcal{O}_{Z\mid U_{i})}$
affin für alle $i\in I$;
\item $x\in U$, $\varphi(g)|_{U}=0$.
\end{enumerate}
Wähle $s\in A$ mit $x\in D(s)\subseteq U$. \textbf{Behauptung: }$\varphi(s^{N}g)=0$
für $N>0$. Mit $\varphi$ injektiv folgt dann $s^{N}g=0$, und $\frac{g}{1}=0$
in $\mathcal{O}_{X,x}$ da $s$ eine Einheit ist in $\mathcal{O}_{X,x}$.
\begin{itemize}
\item Nach (2) ist $\varphi(g)=0$, d.h. $\varphi(s\cdot g)|_{U}=\varphi(s)|_{U}\cdot\underbrace{\varphi(g)|_{U}}_{=0}=0$.
\item $D_{U_{i}}(\varphi(s)|_{U_{i}})=D(s)\cap U_{i}\subseteq U\cap U_{i}$,
also $\varphi(g)|_{D_{U_{i}}(\varphi(s)|_{U_{i}})}=0$, d.h. $\frac{\varphi(g)}{1}=0$
in $\mathcal{O}_{Z}(U_{i})_{\varphi(s)|_{U_{i}}}$. $\Longleftrightarrow\varphi(s)|_{U_{i}}^{N_{i}}\varphi(g)=\varphi(s^{N_{i}}g)=0$
(Die Indexmenge $I$ ist endlich). Setze $N:=\max_{i\in I}\{1,N_{i}\}$.
\end{itemize}
\end{proof}

\section{Unterschemata und Einbettung}

Offene und abgeschlossene Unterschemata sind Spezialfälle von \emph{lokal
abgeschlossene} Unterschemata.
\begin{defn}[37]
\mbox{}
\begin{enumerate}
\item Sei $X$ ein Schemata. Ein \textbf{Unterschemata} von $X$ ist ein
Schemata $(Y,\mathcal{O}_{Y})$, so dass $Y\subset X$ eine lokal
abgeschlossene Teilmenge von $X$ ist, und $Y$ ein abgeschlossenes
Unterschemata von dem offenen Unterschemata $U=X\backslash(\overline{Y}\backslash Y)\subseteq X$
ist. Wir haben dann einen natürlichen Morphismus $Y\rightarrow X$
von Schemata.
\item Eine \textbf{Einbettung} $i:Y\rightarrow X$ ist ein Morphismus von
Schemata, dessen unterlegende stetige Abbildung ein Homöomorphismus
von $Y$ auf eine lokale abgeschlossene Teilmenge von $X$ ist, und
sodass für alle $y\in Y$ : 
\[
i_{y}^{\#}:\mathcal{O}_{X,i(y)}\rightarrow\mathcal{O}_{Y,y}
\]
surjektiv ist.
\end{enumerate}
\end{defn}

\begin{rem}[38]
\mbox{}
\begin{enumerate}
\item Ist $Y$ ein Unterschemata von $X$, dann ist $Y\hookrightarrow X$
eine Einbettung. Umgekehrt bestimmt jede Einbettung einen Isomorphismus
seiner Quelle mit einem eindeutigen Unterschemata seines Ziels.
\item Ist $Y$ ein Unterschemata von $X$, wessen unterliegende Teilmenge
abgeschlossen in $X$ ist, dann ist $Y$ ein abgeschlossenes Unterschemata
von $X$.
\item Das Analogon von $(ii)$ für offene Unterschemata ist i.A. falsch.
\item Jede Einbettung $i:Y\hookrightarrow X$ faktorisiert als:
\[
\xymatrix{Y\ar@{^{(}->}[r]^{i}\ar@{^{(}->}[rd] & X\\
 & U=X\backslash(\overline{i(Y)}\backslash i(Y))\ar@{^{(}->}[u]
}
\]
\end{enumerate}
\end{rem}

\begin{defn}[39]
Sei $X$ ein Schemata und $Z,Z'$ Unterschemata. Wir sagen $Z'$
\textbf{majorisiert} $Z$, wenn die Inklusion $Z\hookrightarrow X$
faktorisiert als:
\[
\xymatrix{Z\ar@{^{(}->}[r]\ar[rd] & X\\
 & Z'\ar@{^{(}->}[u]
}
.
\]
\end{defn}

\begin{rem}[40]
Sei \textbf{P} die Eigenschaft eines Schemata-Morphismus, eine affine
Einbettung, bzw. abgeschlossene Einbettung, bzw. Einbettung zu sein.
Dann:
\begin{enumerate}
\item Die Eigenschaft \textbf{P} ist lokal auf der Basis, d.h. für $f:Z\rightarrow X$
Morphismus, $X=\bigcup_{i\in I}U_{i}$ offene Überdeckung hat $f$
die Eigenschaft \textbf{P} $\Longleftrightarrow\forall i$ hat $f^{-1}(U_{i})\rightarrow U_{i}$
die Eigenschaft \textbf{P}.
\item Die Komposition zweier Morphismen mit Eigenschaft \textbf{P} hat Eigenschaft
\textbf{P}.
\end{enumerate}
\end{rem}

\begin{example}[41]
\mbox{}
\begin{enumerate}
\item Sei $I\subseteq R[T_{0},\ldots,T_{n}]$ homogenes Ideal. Dann ist
$V_{+}(I)\subseteq\mathbb{P}_{R}^{n}$ ein abgeschlossenes Unterschemata
von $\mathbb{P}_{R}^{n}$. (Nach Bemerkung 40.1, denn $V_{+}(I)\cap U_{i}\subseteq U_{i}$
abgeschlossen.)
\item Alle Unterschemata eines $k$-Schematas $X$ von endlichem Typ sind
selbst von endlichem Typ. 
\end{enumerate}
\end{example}


\section{Projektive und quasi-projektive Schemata über einen Körper}
\begin{defn}[42]
Sei $k$ ein Körper.
\begin{enumerate}
\item Ein $k$-Schemata $X$ heißt \textbf{projektiv} wenn es ein $n\geq0$
und eine abgeschlossene Einbettung $X\hookrightarrow\mathbb{P}_{k}^{n}$
gibt.
\item Ein $k$-Schemata $X$ heißt \textbf{quasi-projektiv }wenn es ein
$n\geq0$ und eine Einbettung $X\hookrightarrow\mathbb{P}_{k}^{n}$
gibt.
\end{enumerate}
\end{defn}

\begin{example}[43]
\mbox{}
\begin{enumerate}
\item Für ein homogenes Ideal $I$ sind $V_{+}(I)$ projektive Schemata
(Beispiel 41).
\item Sei $X=\Spec A$ affines $k$-Schemata von endlichem Typ. Dann ist
$X$ quasi-projektiv: $A\cong k[T_{1},\ldots,T_{n}]\backslash\mathfrak{a}$,
\[
\xymatrix{X\ar@{^{(}->}[r]\ar[rd] & \mathbb{A}^{n}\ar[d]^{j}\\
 & \mathbb{P}^{n}
}
\]
\end{enumerate}
\end{example}


\section{Reduzierte Unterschemata}

\[
\Spec K[X,Y]\supset\Spec(K[X,Y]/Y^{2})\supset\Spec(K[X,Y]/Y)
\]

\begin{question*}
Gibt es ein ,,kleinstes`` Unterschemata?
\end{question*}
Setze $\mathcal{N}_{X}\subset\mathcal{O}_{X}$, Garbifizierung der
Prägarben:
\[
U\mapsto\nil(\Gamma(U,\mathcal{O}_{X})),\quad U\subseteq X\text{ offen}
\]

Definiere $X_{\red}:=(X,\mathcal{O}_{X}/\mathcal{N}_{X})$.
\end{document}
