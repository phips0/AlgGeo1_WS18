%% LyX 2.3.2 created this file.  For more info, see http://www.lyx.org/.
%% Do not edit unless you really know what you are doing.
\documentclass[oneside,ngerman]{book}
\usepackage[T1]{fontenc}
\usepackage[utf8]{inputenc}
\setcounter{secnumdepth}{3}
\setcounter{tocdepth}{3}
\usepackage{amsmath}
\usepackage{amsthm}
\usepackage{amssymb}
\usepackage{stmaryrd}
\usepackage[all]{xy}

\makeatletter
%%%%%%%%%%%%%%%%%%%%%%%%%%%%%% Textclass specific LaTeX commands.
\theoremstyle{plain}
\ifx\thechapter\undefined
	\newtheorem{thm}{\protect\theoremname}
\else
	\newtheorem{thm}{\protect\theoremname}[chapter]
\fi
\theoremstyle{plain}
\newtheorem{prop}[thm]{\protect\propositionname}
\theoremstyle{plain}
\newtheorem*{lem*}{\protect\lemmaname}
\theoremstyle{definition}
\newtheorem*{example*}{\protect\examplename}
\theoremstyle{plain}
\newtheorem{lem}[thm]{\protect\lemmaname}
\theoremstyle{definition}
\newtheorem{defn}[thm]{\protect\definitionname}
\theoremstyle{remark}
\newtheorem{rem}[thm]{\protect\remarkname}

%%%%%%%%%%%%%%%%%%%%%%%%%%%%%% User specified LaTeX commands.
%% User-specified commands
\DeclareMathOperator{\rad}{rad}
\DeclareMathOperator{\Spec}{Spec}
\DeclareMathOperator{\maxspec}{MaxSpec}
\DeclareMathOperator{\Quot}{Quot}
\DeclareMathOperator{\im}{\mathrm{im}}
\DeclareMathOperator{\Hom}{\mathrm{Hom}}
\DeclareMathOperator{\Mor}{\mathrm{Mor}}
\DeclareMathOperator{\id}{\mathrm{id}}
\DeclareMathOperator{\res}{res}
\DeclareMathOperator{\Abb}{Abb}
\DeclareMathOperator{\supp}{supp}
\DeclareMathOperator{\red}{red}
\DeclareMathOperator{\op}{op}
\DeclareMathOperator{\obj}{Obj}
\DeclareMathOperator{\nil}{nil}

%% sets
\DeclareMathOperator{\CC}{\mathbb{C}}
\DeclareMathOperator{\RR}{\mathbb{R}}
\DeclareMathOperator{\QQ}{\mathbb{Q}}
\DeclareMathOperator{\ZZ}{\mathbb{Z}}
\DeclareMathOperator{\NN}{\mathbb{N}}

%% categories
\DeclareMathOperator{\ouv}{\mathcal{O}uv}
\DeclareMathOperator{\set}{\underline{Set}}
\DeclareMathOperator{\ab}{\underline{Ab}}
\DeclareMathOperator{\cattop}{\underline{Top}}
\DeclareMathOperator{\cring}{\underline{CRing}}
\DeclareMathOperator{\ring}{\underline{Ring}}
\DeclareMathOperator{\psh}{\underline{\mathcal{PS}h}}
\DeclareMathOperator{\sh}{\underline{\mathcal{S}h}}
\DeclareMathOperator{\aff}{\underline{Aff}}
\DeclareMathOperator{\sch}{\underline{Sch}}
\DeclareMathOperator{\schk}{\underline{Sch/k}}
\DeclareMathOperator{\schs}{\underline{Sch/S}}
\DeclareMathOperator{\schsn}{\underline{Sch/S_{0}}}
\DeclareMathOperator{\pres}{\underline{Prevar/S}}
\DeclareMathOperator{\prek}{\underline{Prevar/k}}
\DeclareMathOperator{\affk}{\underline{AffVar/k}}
\DeclareMathOperator{\topcp}{\underline{TopCP}}
\DeclareMathOperator{\Top}{\underline{Top}}
\DeclareMathOperator{\grp}{\underline{Grp}}
\DeclareMathOperator{\func}{\underline{Func}}

\makeatother

\usepackage{babel}
\providecommand{\definitionname}{Definition}
\providecommand{\examplename}{Beispiel}
\providecommand{\lemmaname}{Lemma}
\providecommand{\propositionname}{Satz}
\providecommand{\remarkname}{Bemerkung}
\providecommand{\theoremname}{Theorem}

\begin{document}
Sei $X,X'\in\schs$, $f:X'\rightarrow X$ Morphismus in $\schs$,
$g:=f\times_{S}\id_{Y}$.
\[
\xymatrix{Z'=X'\times_{S}Y\ar[r]^{g}\ar[d]_{p'} & Z=X\times_{S}Y\ar[r]^{q}\ar[d]_{p'} & Y\ar[d]\\
X'\ar[r]^{f} & X\ar[r] & S
}
\]

kommutiert. Da $q\circ g=q'$ Projektion auf $Y$ ist, ist das große
und damit auch beide Diagramme kartesisch. (Proposition 10)
\begin{prop}[14]
$f$ induziert einen Homömorphismus von $X'$ auf $f(X')$ und:
\begin{enumerate}
\item $f_{x'}^{\#}:\mathcal{O}_{X,f(x')}\rightarrow\mathcal{O}_{X',x'}$
sei surjektiv $\forall x'\in X'$ und es existiert eine offene affine
Umgebung $U'$ von $f(x')$, sodass $f^{-1}(U')$ quasi-kompakt ist,
oder
\item $f_{x'}^{\#}$ ist bijektiv $\forall x'\in X'$.
\end{enumerate}
Dann gilt:
\begin{enumerate}
\item $g$ ist ein Homöomorphismus von $Z'$ auf $g(Z')=p^{-1}(f(X'))$.
\item $\forall z'\in Z'$ haben wir das induzierte Diagramm für lokale Ringe:
\[
\xymatrix{\mathcal{O}_{Z',z'}\ar[d] & \mathcal{O}_{Z,g(z')}\ar[l]^{g_{z'}^{\#}}\ar[d]^{p_{g(z)}^{\#}}\\
\mathcal{O}_{X',p'(z')} & \mathcal{O}_{X,p(g(z'))}\ar[l]^{f_{p'(z')}}
}
\]
\end{enumerate}
\begin{itemize}
\item $g_{z'}^{\#}$ ist surjektiv;
\item $\ker(g_{z'}^{\#})$ ist von $p_{g(z')}^{\#}(\ker f_{p'(z')}^{\#})$
erzeugt.
\end{itemize}
\end{prop}

\begin{proof}
(1), (2) lassen sich lokal bzgl. $S,Y,X$ verifizieren. Ohne Einschränkung
sei $S=\Spec R$, $X=\Spec A$, $Y=\Spec B$ affin, $X'$ quasi-kompakt.
\[
f\leftrightarrow\xymatrix{A\ar[r]^{\varphi}\ar@{->>}[rd]_{\varphi_{1}} & \Gamma(X,\mathcal{O}_{X})\\
 & A/\ker\varphi\ar@{^{(}->}[u]^{\varphi_{2}}
}
\quad R\text{-Algebren}
\]

$f$ faktorisiert sich als
\[
\xymatrix{X'\ar[r]^{f_{1}} & \Spec(A/\ker\varphi)\ar[r]^{f_{2}} & \Spec(A)=X}
.
\]

Dann ist $f_{2}$ eine abgeschlossene Immersion, also surjektiv auf
Halmen $(f_{p})_{2}$, und ein Homeomorphismus auf einer abgeschlossenen
Teilmenge in $X$. In Situation $I$ erfüllt daher mit $f$ auch $f_{1}$
Voraussetzung (1). Daher reicht es die folgenden 2 Fälle zu beweisen.
\begin{enumerate}
\item $f$ ist eine abgeschlossene Immersion ($\hat{=}$ $f_{2}$ Voraussetzung
(1))
\item $f_{x'}^{\#}$ ist bijektiv für alle $x'\in X'$ ($\hat{=}$ $f_{1}$
Voraussetzung (1) + Voraussetzung (2)).
\end{enumerate}
\begin{lem*}
Sei:
\[
\xymatrix{A\ar@{^{(}->}[r] & \Gamma(X,\mathcal{O}_{X})\\
X'\ar[r]^{f} & \Spec(A)
}
,\quad X'\text{ quasi-kompakt}
\]

Dann ist $f_{x'}^{\#}$ injektiv für alle $x\in X'$.
\end{lem*}
\textbf{Vorüberlegung. }Sei $Z$ ein Schemata und $t\in\Gamma(Z,\mathcal{O}_{Z})$,
$Z_{T}:=\{z\mid t(z)\neq0\}\subset Z$ offen. Die Einschränkung
\[
\xymatrix{\Gamma(Z,\mathcal{O}_{Z})\ar[r]\ar[d] & \Gamma(Z_{t},\mathcal{O}_{Z})\\
\Gamma(Z,\mathcal{O}_{Z})_{t}\ar@{^{(}-->}[ur]_{\rho_{t}}
}
\]

definiert einen Homomorphismus $\rho_{t}$. Dieser ist injektiv, falls
$Z$ quasi-kompakt, \textbf{denn} $Z=\bigcup U_{i}$ ist endliche
offene affine Überdeckung. Sei $C_{i}=\mathcal{O}_{Z}(U_{i})$, $t_{i}=t|_{U_{i}}$.
$\Longrightarrow(\prod_{i}C_{i})_{t}=\prod_{i}(C_{i})_{t_{i}}$ da
$i$ endlich. Wir erhalten das kommutative Diagramm:
\[
\xymatrix{\mathcal{O}_{Z}(Z)\ar[r]\ar@{^{(}-}[d]_{\text{Garbe}} & \Gamma(Z,\mathcal{cO}_{Z})_{t}\ar[r]^{\rho_{t}}\ar@{^{(}-}[d] & \Gamma(Z_{t},\mathcal{O}_{Z})\ar@{^{(}-}[d]^{\text{Garbe}}\\
\prod_{i}C_{i}\ar[r] & \prod_{i}(C_{i})_{t_{i}}\ar[r]_{\cong} & \prod_{i}\Gamma(D(t_{i}),\mathcal{O}_{U_{i}})
}
\]
\begin{proof}[Beweis (Lemma)]
Sei $\mathfrak{p}\subset A\cong f(x')$. Für alle $s\in A\backslash\mathfrak{p}$
sei
\[
\varphi_{s}:A_{s}\longrightarrow\Gamma(X',\mathcal{O}_{X'})_{\varphi(s)}
\]

der injektive Homomorphismus aus $\varphi$ durch Lokalisierung in
$s$, und sei $\psi_{s}$ die injektive Komposition
\[
\xymatrix{\psi_{s}:A_{s}\ar[r]^{\varphi_{s}} & \Gamma(X',\mathcal{O}_{X'})_{\varphi(s)}\ar[r]^{\rho_{\varphi(s)}} & \Gamma(X'_{\varphi(s)},\mathcal{O}_{X'}).}
\]

Dann ist $X'_{\varphi(s)}=f^{-1}(D(s))$. Da für $s\in A\backslash\mathfrak{p}$
die $D(s)$ eine offene Umgebungsbasis von $f(x)'$, und da $f$ ein
Homeomorphismus auf sein Bild ist, bilden die $X'_{\varphi(s)}$ eine
offene Umgebungsbasis von $x'$. $\Longrightarrow\underset{\underset{s}{\longrightarrow}}{\lim}\ \Gamma(X'_{\varphi(s)},\mathcal{O}_{X'})=\mathcal{O}_{X',x'}$
und $\underset{\underset{s}{\longrightarrow}}{\lim}\psi_{s}=f_{x'}^{\#}$
$\Longrightarrow f_{x'}^{\#}$ injektiv.
\end{proof}
Zu 1.) Sei $\xymatrix{X'=\Spec(A/\mathfrak{a})\ar[r]^{f} & \Spec(A)=X}
,$ $\mathfrak{a}\subset A$ Ideal. Proposition 11 $\Longrightarrow Z,Z'$
affin, und $g$ entspricht $R$-Algebren
\[
\xymatrix{A\otimes_{R}B\ar@{->>}[r]^{``g``} & A/\mathfrak{a}\otimes_{R}B\\
A\ar[u]^{``p``}\ar[r]_{``f``} & A/\mathfrak{a}\ar[u]_{p'}
}
\]

und $``p``(\ker``f``)\subset A\otimes_{R}B=\ker``g``$ $\Longrightarrow g$
ist Homöomorphismus auf $g(Z')=p^{-1}(f(x'))$, $x'\in X'$. $\checkmark$\medskip{}

Zu 2.) Es ist $f^{\#}:f^{-1}\mathcal{O}_{X}\rightarrow\mathcal{O}_{X'}$
ein Isomorphismus bzgl. $(X',\mathcal{O}_{X'})\cong(f(x'),\mathcal{O}_{X}|_{f(x')})$
Isomorphismus lokal geringter Räume. Leicht zu verifizieren: $(p^{-1}(f(x')),\mathcal{O}_{Z}|_{p^{-1}(f(x'))})$
ist ein Faserprodukt von $X'$ mit $Z$ über $X$ in der Kategorie
lokal geringter Räume, also erst recht in $\sch$. (vgl. Zusatz in
Theorem (Existenz $X\times_{S}Y$)). $\Longrightarrow g$ Isomorphismus
\[
\xymatrix{(Z',\mathcal{O}_{Z})\ar[r]^{\cong} & (p^{-1}(f(x')),\mathcal{O}_{Z}|_{p^{-1}(f(x'))})}
.
\]
\end{proof}
\begin{example*}
Proposition 14 gilt in folgenden Situationen.
\begin{enumerate}
\item $f$ ist eine Immersion von Schemata.
\item $f$ ist der kanonische Morphismus $\Spec\mathcal{O}_{X,x}\rightarrow X$
für ein $x\in X$, vgl. (5.4), (2.11).
\item $f$ ist der kanonische Morphismus $\Spec\kappa(x)\rightarrow X$
für ein $x\in X$.
\item Komposition von Morphismen, die Proposition 14 erfüllen (und Proposition
10).
\end{enumerate}
\end{example*}

\section{Beispiele}

\paragraph{Produkte affiner Räume}

Sei $R$ ein Ring, und $\mathbb{A}_{R}^{n}=\Spec(R[T_{1},\ldots,T_{n}])$
der affine Raum über $R$. Für $n,m\geq0$ haben wir 
\[
R[T_{1},\ldots,T_{n}]\otimes_{R}R[T_{n+1},\ldots,T_{n+m}]\cong R[T_{1},\ldots,T_{n+m}]
\]

und deshalb nach Proposition 11
\[
\mathbb{A}_{R}^{n}\times_{R}\mathbb{A}_{R}^{m}\cong\mathbb{A}_{R}^{n+m}.
\]


\paragraph{Produkte von Prävarietäten}

Sei $k$ ein algebraisch abgeschlossener Körper, und $X$ ein $k$-Schema
endlichen Typs. Nach 3.14 ist $X_{k}(k)=X_{0}$ (abgeschlossene Punkte
von $X$).
\begin{align*}
x:\Spec k & \longrightarrow X\longrightarrow\text{Bild}\\
\text{\{Integ. Sch. v.e.T./}k\} & \longleftrightarrow\text{\{Präv./}k\}\\
X & \longmapsto\{X_{0},\mathcal{O}_{X}|_{X_{0}}\}
\end{align*}

\begin{lem}[15]
Sei $k$ ein Körper und seien $X,Y$ integre $k$-Schemata. Dann
ist $X\times_{k}Y$ ein integres $k$-Schemata.
\end{lem}

Beweis: später. Falls $X,Y$ integral von endlichem Typ über $k$
sind, dann ist auch $X\times_{k}Y$ integral von endlichem Typ über
$k$. Denn: $X=\bigcup_{\text{endl.}}X_{i}$, $Y=\bigcup_{\text{endl.}}Y_{j}$
$\Longrightarrow X\times_{k}Y=\bigcup_{i,j}X_{i}\times_{k}Y_{i}$.
$\Longrightarrow X=\Spec A$, $Y=\Spec B$ mit $A,B$ endlich erzeugte
$k$-Algebras. $\Longrightarrow X\times_{k}Y=\Spec A\otimes_{k}B$
endlich erzeugte $k$-Algebra.\medskip{}

Seien $X_{0}$ und $Y_{0}$ die Prävarietäten zu $X$ bzw. $Y$, und
$Z_{0}$ die Prävarietät zu $X\times_{k}Y$. Dann gilt nach der universellen
Eigenschaft des Faserprodukts:
\[
Z_{0}=(X\times_{k}Y)_{k}(k)=X_{k}(k)\times Y_{k}(k)=X_{0}\times Y_{0},
\]

d.h. das Faserprodukt von 2 Prävarietäten $X_{0},Y_{0}$ ist wieder
eine Prävarietät $Z_{0}$ (als volle Unterkategorie von $\schk$)
mit $Z_{0}=X_{0}\times Y_{0}$ (als Mengen). Die Projektionen $Z_{0}\rightarrow X_{0}$
und $Z_{0}\rightarrow Y_{0}$ sind stetig, aber im Allgemeinen ist
die Topologie auf $Z_{0}$ \textbf{feiner }als die Produkttopologie
von $X_{0}$ und $Y_{0}$.

\section{Basiswechsel}

Sei $\mathcal{C}$ eine beliebige Kategorie mit Faserprodukten (z.B.
$\sch$), $u:S'\rightarrow S$ ein Morphismus in $\mathcal{C}$, $X\rightarrow S$
ein $S$-Objekt. $\Longrightarrow q:X\times_{S}S'\rightarrow S'$
ist ein $S'$-Objekt. Bezeichne $u^{\ast}(X)=:q$ oder $X_{(s')}$
\textbf{Urbild} oder \textbf{Basiswechsel} von $X$ bzgl. $u$.\medskip{}

Sei $f:X\rightarrow Y$ Morphismus von $S$-Objekten. $\Longrightarrow f\times_{S}\id_{S'}:X\times_{S}S'\rightarrow Y\times_{S}S'$
ist ein Morphismus von $S'$-Objekten. Bezeichne $f\times_{S}\id_{S'}=:u^{*}(f)=:f_{(s')}$
der \textbf{Basiswechsel von $f$ bzgl. $u$}. Wir erhalten einen
kontravarianten Funktor
\[
u^{\ast}:\mathcal{C}/S\longrightarrow\mathcal{C}/S'
\]

der Kategorie von $S$-Objekten in $\mathcal{C}$ zu der Kategorie
der $S'$-Objekten in $\mathcal{C}$. Nenne $u^{\ast}$ den \textbf{Basiswechsel
bzgl. $u$}.

\paragraph{Transitivität des Basiswechsels}

Sei $u':S''\rightarrow S'$ ein weiterer Morphismus in $\mathcal{C}$.
Nach Proposition 10 ist $(u\circ u')^{\ast}\cong u'^{\ast}\circ u^{\ast}$
ein Isomorphismus von Funktoren. Sei
\[
\xymatrix{T\ar[r]^{h}\ar[rd] & S'\ar[d]^{u}\\
 & S
}
\in\mathcal{C}/S'.
\]

Wir können $T$ als $S$-Objekt auffassen durch $u\circ h$. Sei $p:X_{(S')}\rightarrow X$
die erste Projektion. Dann erhalten wir zueinander inverse Bijektionen,
funktoriell in $T$ und $X$:
\begin{align*}
t & \longmapsto p\circ\\
\hom_{S'}(T,X_{(S')}) & \longleftrightarrow\hom_{S}(T,X)\\
(t,h)_{S'} & \longmapsfrom t
\end{align*}

\begin{defn}[16]
Sei $\mathbb{P}$ eine Eigenschaft von Morphismen in $\mathcal{C}$,
sodass $\id_{X}$ $\mathbb{P}$ erfüllt für alle $X\in\mathcal{C}$.
\begin{enumerate}
\item $\mathbb{P}$ heißt \textbf{stabil}
\begin{enumerate}
\item \textbf{unter Komposition}, wenn mit $f:X\rightarrow Y$ und $g:Y\rightarrow Z$
auch $g\circ f$ $\mathbb{P}$ erfüllt.
\item \textbf{unter Basiswechsel}, wenn mit $f:X\rightarrow S$ auch $f_{(S')}:X_{(S')}\rightarrow S'$
für alle Morphismen $S'\rightarrow S$, $\mathbb{P}$ erfüllt.
\end{enumerate}
\item Wir sagen, dass $f:X\rightarrow S$ $\mathbb{P}$ \textbf{universell}
erfüllt, falls $f_{(S')}$ $\mathbb{P}$ erfüllt für alle $S'\rightarrow S$.
\end{enumerate}
\end{defn}

\begin{rem}[17]
Sei $\mathbb{P}$ stabil unter Komposition. Dann sind äquivalent:
\begin{enumerate}
\item $\forall S\in\mathcal{C}$, $\forall S$-Morphismen $f:X'\rightarrow X$,
$g:Y'\rightarrow Y$, die $\mathbb{P}$ erfüllen, erfüllt auch $f\times_{S}g$
$\mathbb{P}$.
\item $\mathbb{P}$ ist stabil unter Basiswechsel.
\end{enumerate}
\end{rem}

\begin{proof}
\mbox{}
\begin{itemize}
\item[$(i)\Rightarrow(ii)$.] $f_{(S')}=f\times_{S}\id_{S'}$.
\item[$(ii)\Rightarrow(i)$.] Seien $f,g$ Morphismen (wie in 1) die $\mathbb{P}$ erfüllen. Da
$f\times_{S}g=(f\times_{S}\id_{Y})\circ(\id_{X}\times_{S}g)$ sei
ohne Einschränkung $g=\id_{Y}$.
\begin{align*}
f_{(X\times_{S}Y)}=f\times_{S}\id_{Y}: & X'\times_{S}Y=X'\underbrace{\times_{X}}_{\text{bzgl. }f}(X\times_{S}Y)\rightarrow X\times_{S}Y
\end{align*}
erfüllt $\mathbb{P}$.
\end{itemize}
In $\sch$ sind fast alle betrachteten Eigenschaften von Morphismen
stabil unter Komposition, aber nicht unbedingt unter Basiswechsel,
z.B. injektiv oder abgeschlossen.
\end{proof}
\begin{example*}
Es ist
\begin{align*}
f:X=\Spec\mathbb{Q}(\xi_{p}) & \longrightarrow\Spec\mathbb{Q}=S\\
u:S' & \longrightarrow\Spec\mathbb{Q}
\end{align*}

Homöomorphismus, d.h. injektiv, aber
\begin{align*}
f_{(S')}:X\times_{S}S' & \longrightarrow\underbrace{S'=\Spec\mathbb{Q}(\xi_{p})}_{1\text{ Punkt}}
\end{align*}

ist nicht injektiv:
\[
\Spec(\mathbb{Q}(\xi_{p})\otimes\mathbb{Q}_{p})\cong\underbrace{\prod^{p-1}\mathbb{Q}(\xi_{p})}_{p-1\text{ Punkte}}.
\]
\end{example*}
\textbf{Warnung.} Absolute Eigenschaften von Schemata sind oft nicht
kompatibel mit Basiswechsel.

Sei $k=\mathbb{F}_{p}(t)$ (nicht perfekt!), $K=\bigcup_{n\geq1}\mathbb{F}_{p}(t^{\frac{1}{p^{n}}})$
perfekter Abschluss von $k$, $A:=K\otimes_{k}K$. Man kann zeigen:
$\nil(A)$ ist \emph{nicht} endlich erzeugt, d.h. $\Spec(A)$, $A$
ist nicht reduzibel und nicht noethersch.
\end{document}
